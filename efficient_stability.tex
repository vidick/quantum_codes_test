\documentclass[11pt]{article}
\usepackage{booktabs}
\usepackage{fullpage}
\usepackage{titlesec}
%\newcommand{\sectionbreak}{\clearpage}
\usepackage{amsmath,amsfonts,amsthm,mathrsfs,xspace,graphicx}
\usepackage[backref,colorlinks,citecolor=blue,bookmarks=true]{hyperref}
\usepackage{mathpazo}
\usepackage{endnotes}
\usepackage{color}
\usepackage{float}
\usepackage{xcolor}
\usepackage{mdframed}
\usepackage{bbm}
\usepackage{suffix} % for *-version commands
\usepackage{times}
\usepackage{tabularx}
\usepackage{makecell}
\usepackage{amssymb,latexsym}
%\usepackage{IEEEtrantools}
\usepackage[capitalize]{cleveref}
\usepackage{enumitem}
\usepackage{tikz}
\usepackage{tikz-cd}
\usepackage{multirow}
\usepackage[section]{placeins}
\usepackage[affil-it]{authblk}


\mdfdefinestyle{figstyle}{ %
  linecolor=black!7, %
  backgroundcolor=black!7, %
  innertopmargin=10pt, %
  innerleftmargin=25pt, %
  innerrightmargin=25pt, %
  innerbottommargin=10pt %
}

\newtheorem{theorem}{Theorem}[section]
\newtheorem{proposition}[theorem]{Proposition}
\newtheorem{conjecture}[theorem]{Conjecture}
\newtheorem{lemma}[theorem]{Lemma}
\newtheorem{claim}[theorem]{Claim}
\newtheorem{fact}[theorem]{Fact}
\newtheorem{corollary}[theorem]{Corollary}

\newtheorem{remark}[theorem]{Remark}

\theoremstyle{definition}
\newtheorem{definition}[theorem]{Definition}
\newtheorem{example}[theorem]{Example}

\newcommand{\beq}{\begin{eqnarray}}
\newcommand{\eeq}{\end{eqnarray}}

\newcommand{\code}{\mathcal{C}}
\newcommand{\strategy}{\mathscr{S}}
\newcommand{\algebra}{\mathscr{A}}

\newcommand{\ket}[1]{|#1\rangle}
\newcommand{\bra}[1]{\langle#1|}
\newcommand{\ketbra}[2]{\ket{#1}\!\bra{#2}}
\newcommand{\ip}[2]{\langle #1 \! | #2 \rangle}
\newcommand{\proj}[1]{\ket{#1}\!\bra{#1}}
\newcommand{\Tr}{\mbox{\rm Tr}}
\newcommand{\Id}{\ensuremath{I}}
\DeclareMathOperator*{\Expectation}{\mathbb{E}}
\newcommand{\Es}[1]{\Expectation_{#1}}

\newcommand{\reg}[1]{{\textsf{#1}}}
\newcommand{\ol}[1]{\overline{#1}}

\newcommand{\field}{\mathbb{F}_2}
\newcommand{\C}{\ensuremath{\mathbb{C}}}
\newcommand{\N}{\ensuremath{\mathbb{N}}}
\newcommand{\bbN}{\ensuremath{\mathbb{N}}}
\newcommand{\complex}{\ensuremath{\mathbb{C}}}
\newcommand{\real}{\ensuremath{\mathbb{R}}}
%\newcommand{\natural}{\ensuremath{\mathbb{N}}}

\newcommand{\bij}{\pi}
\newcommand{\qp}{\tau}
\newcommand{\dlS}{\ensuremath{\rm dlS}}

\newcommand{\F}{\ensuremath{\mathbb{F}}}
\newcommand{\M}{\ensuremath{\mathbb{M}}}
\newcommand{\ot}{\otimes}
\newcommand{\Fp}{\F_p}
\newcommand{\Fq}{\field}
\newcommand{\BH}{\textsc{BH}}
\newcommand{\ld}{\textsc{ld}}
\newcommand{\com}{\textsc{com}}
\newcommand{\sq}{\textsc{sq}}
\newcommand{\downsize}{\kappa}
\newcommand{\tobin}{\flat}
\newcommand{\downsizem}{\chi}

\newcommand{\K}{\ensuremath{\mathbb{K}}}
\newcommand{\R}{\ensuremath{\mathbb{R}}}
\newcommand{\Z}{\ensuremath{\mathbb{Z}}}

\newcommand{\mA}{\ensuremath{\mathcal{A}}}
\newcommand{\mB}{\ensuremath{\mathcal{B}}}
\newcommand{\mC}{\ensuremath{\mathcal{C}}}
\newcommand{\mE}{\ensuremath{\mathcal{E}}}
\newcommand{\mD}{\ensuremath{\mathcal{D}}}
\newcommand{\mF}{\ensuremath{\mathcal{F}}}
\newcommand{\mG}{\ensuremath{\mathcal{G}}}
\newcommand{\mH}{\ensuremath{\mathcal{H}}}
\newcommand{\mK}{\ensuremath{\mathcal{K}}}
\newcommand{\mM}{\ensuremath{\mathcal{M}}}
\newcommand{\mI}{\ensuremath{\mathcal{I}}}
\newcommand{\mJ}{\ensuremath{\mathcal{J}}}
\newcommand{\cM}{\ensuremath{\mathcal{M}}}
\newcommand{\mP}{\ensuremath{\mathcal{P}}}
\newcommand{\mQ}{\ensuremath{\mathcal{Q}}}
\newcommand{\mR}{\ensuremath{\mathcal{R}}}
\newcommand{\mS}{\ensuremath{\mathcal{S}}}
\newcommand{\mT}{\ensuremath{\mathcal{T}}}
\newcommand{\mU}{\ensuremath{\mathcal{U}}}
\newcommand{\mX}{\ensuremath{\mathcal{X}}}
\newcommand{\mY}{\ensuremath{\mathcal{Y}}}

\newcommand{\Inv}{\ensuremath{\textsc{Inv}}}
\newcommand{\GEN}{\ensuremath{\textsc{GEN}}}
\newcommand{\SAMP}{\ensuremath{\textsc{SAMP}}}
\newcommand{\epr}{\ensuremath{\textsc{epr}}}
\newcommand{\RM}{\ensuremath{\textsc{RM}}}
\newcommand{\bRM}{\ensuremath{\textsc{RM2}}}
\newcommand{\Had}{\ensuremath{\textsc{Had}}}
\newcommand{\HRM}{\ensuremath{\textsc{HRM}}}


\newcommand{\Alg}{\mathcal{A}}
\newcommand{\ind}{\ensuremath{\mathrm{ind}}}


\newcommand{\setft}[1]{\mathrm{#1}}
\newcommand{\Density}{\setft{D}}
\newcommand{\Pos}{\setft{Pos}}
\newcommand{\Proj}{\setft{Proj}}
\newcommand{\Channel}{\setft{C}}
\newcommand{\Unitary}{\setft{U}}
\newcommand{\Herm}{\setft{Herm}}
\newcommand{\Obs}{\setft{Obs}}
\newcommand{\Lin}{\setft{L}}
\newcommand{\Trans}{\setft{T}}
\DeclareMathOperator{\poly}{poly}
\DeclareMathOperator{\negl}{negl}
\newcommand{\dset}{G}

\newcommand{\val}{\ensuremath{\mathrm{val}}}
\newcommand{\valco}{\ensuremath{\mathrm{val}^{\mathrm{co}}}}
\newcommand{\ia}{\Id_\alice}
\newcommand{\ib}{\Id_\bob}

\newcommand{\desc}[1]{\overline{\cal{#1}}}
\newcommand{\supp}{\textsc{Supp}}
\newcommand{\Gen}{\textsc{Gen}}
\newcommand{\Enc}{\textsc{Enc}}
\newcommand{\Dec}{\textsc{Dec}}

\newcommand{\GenTrap}{\textsc{GenTrap}}
\newcommand{\Invert}{\textsc{Invert}}
\newcommand{\lossy}{\textsc{lossy}}

\newcommand{\rand}{\textrm{rand}}
\newcommand{\had}{\textsc{Had}}


\newcommand{\eps}{\varepsilon}
\newcommand{\ph}{\ensuremath{\varphi}}


\newcommand{\ac}{\textsc{ac}}
\newcommand{\GX}{\textsc{Gap-Maxcut}}
\newcommand{\GNI}{\textsc{Graph Non-Isomorphism}}


\newcommand{\Acc}{\textsc{Acc}}
\newcommand{\Samp}{\textsc{Samp}}
\newcommand{\Ext}{\ensuremath{\text{Ext}}}

\newcommand{\BD}{\mathbb{QB}}
\newcommand{\DD}{\mathbb{D}}
\newcommand{\DDb}{\mathbb{D'}}
\newcommand{\Pot}{\Phi}
\newcommand{\inj}{J}
\newcommand{\mZ}{\mathcal{Z}}
\newcommand{\mN}{\mathcal{N}}
\newcommand{\vs}{\vspace{2mm}~\newline\noindent}
\newcommand{\vb}{\vspace{3mm}\noindent}
\newcommand{\sX}{\mathcal{X}}
\newcommand{\sA}{\mathcal{A}}
\newcommand{\sB}{\mathcal{B}}
\newcommand{\sY}{\mathcal{Y}}
\newcommand{\sR}{\mathcal{R}}


\newcommand{\trnq}[1]{\left[ {#1} \right]_q}

\DeclareMathOperator{\polylog}{polylog}
\newcommand{\mx}[1]{\mathbf{{#1}}}
\newcommand{\vc}[1]{\mathbf{{#1}}}
\newcommand{\abs}[1]{\left\vert {#1} \right\vert}
\newcommand{\norm}[1]{\left\| {#1} \right\|}
\newcommand{\for}{\text{for }}

\DeclareMathOperator{\arcsinh}{arcsinh}
\DeclareMathOperator{\tr}{tr}
\DeclareMathOperator{\sgn}{sgn}


\newcommand{\E}{\mathop{\mathbb{E}}\displaylimits} % Expectation

\newcommand{\unif}{\mathcal{U}}
\newcommand{\pt}{\textrm{pt}}
\newcommand{\sample}{\textrm{sample}}
\newcommand{\test}{\textrm{test}}
\newcommand{\free}{\mathcal{F}}
\newcommand{\plane}{\mathcal{P}}
\newcommand{\lines}{\mathcal{L}}
\newcommand{\clines}{\mathcal{CL}}
\newcommand{\pl}{\mathbf{p}}
\newcommand{\individual}{\textrm{individual}}
\newcommand{\blocks}{\textrm{blocks}}
\newcommand{\liness}{\textrm{lines}}
\newcommand{\lp}{\mathcal{LP}}
\newcommand{\Pl}{\ensuremath{\mathrm{Pl}}}
\newcommand{\Ln}{\ensuremath{\mathrm{Lines}}}
\newcommand{\mode}{\mathfrak{m}}
\newcommand{\ECC}{\ensuremath{\textsc{ECC}}}
\newcommand{\EC}{\ensuremath{\textsc{EC}}}
\newcommand{\ENC}{\ensuremath{\textsc{ENC}}}
\newcommand{\cktval}{\ensuremath{\textsc{CKTVAL}}}


\newcommand{\GL}{\mathrm{GL}}
\newcommand{\Matrix}{\mathrm{M}}
\newcommand{\End}{\mathrm{End}}
\newcommand{\Aut}{\mathrm{Aut}}

\newcommand{\game}{\mathfrak{G}}
\newcommand{\sampler}{\mathcal{S}}
\newcommand{\decider}{\mathcal{D}}
\newcommand{\verifier}{\mathcal{V}}


\newcommand{\type}{\mathcal{T}}
\newcommand{\lt}{\mathcal{L}}
\newcommand{\rt}{\mathcal{R}}
\newcommand{\checker}{\mathcal{C}}


\newcommand{\gamestyle}[1]{\ensuremath{\textsc{#1}}\xspace}
\newcommand{\qld}{\gamestyle{QLD}}
\newcommand{\ms}{\gamestyle{MS}}
\newcommand{\pauli}{\gamestyle{P}}
%\newcommand{\bp}{\gamestyle{BP}}
\newcommand{\ora}{\gamestyle{Orac}}
\newcommand{\pcp}{\gamestyle{PCP}}
\newcommand{\ar}{\gamestyle{AR}}
\newcommand{\intro}{\gamestyle{Intro}}

\newcommand{\labelstyle}[1]{\ensuremath{\textsc{#1}}\xspace}
\newcommand{\EPR}{\labelstyle{EPR}}
\newcommand{\aux}{\labelstyle{aux}}
\newcommand{\ancilla}{\labelstyle{anc}}
\newcommand{\msc}{\labelstyle{MC}}
\newcommand{\msv}{\labelstyle{MV}}
\newcommand{\vertex}[1]{\labelstyle{V#1}}
\newcommand{\edge}[1]{\labelstyle{N#1}}
\newcommand{\basis}{\labelstyle{W}}
\newcommand{\xpt}{\labelstyle{X}}
\newcommand{\zpt}{\labelstyle{Z}}
\newcommand{\rxpt}{\labelstyle{R}_\xpt}
\newcommand{\rzpt}{\labelstyle{R}_\zpt}
\newcommand{\dir}[1]{\labelstyle{V#1}}
\newcommand{\coord}{\labelstyle{I}}
\newcommand{\intercept}{\labelstyle{U}}
\newcommand{\plf}{\labelstyle{Pl}}
\newcommand{\lnf}{\labelstyle{Ln}}
\newcommand{\ptf}{\labelstyle{Pt}}
\newcommand{\full}{\labelstyle{full}}
\newcommand{\opt}{\labelstyle{opt}}
\newcommand{\partition}{\mathcal{B}}

\newcommand{\tvarstyle}[1]{\mathsf{#1}}
\newcommand{\tvar}{\ensuremath{\tvarstyle{t}}}
\newcommand{\lvar}{\ensuremath{\tvarstyle{u}}}
\newcommand{\rvar}{\ensuremath{\tvarstyle{v}}}
\newcommand{\pvar}{\ensuremath{\tvarstyle{p}}}
\newcommand{\ovar}{\ensuremath{\tvarstyle{o}}}
\newcommand{\trole}{\ensuremath{v}} % used in intro types

\newcommand{\types}{\labelstyle{T}}

\newcommand{\decode}{\labelstyle{Decode}}

%\newcommand{\alice}{\labelstyle{Alice}}
%\newcommand{\bob}{\labelstyle{Bob}}
\newcommand{\alice}{\labelstyle{A}}
\newcommand{\bob}{\labelstyle{B}}

\newcommand{\oracle}{\labelstyle{Oracle}}
\newcommand{\ab}{\{\alice, \bob\}}

\newcommand{\typestyle}[1]{\ensuremath{\textsc{#1}}\xspace}
\newcommand{\Type}{\typestyle{Type}}
\newcommand{\Plane}{\typestyle{Plane}}
\renewcommand{\line}{\mathbf{\ell}}
\newcommand{\Llane}{\typestyle{Line}}
\newcommand{\Point}{\typestyle{Point}}
\newcommand{\HPoint}{\typestyle{HPoint}}
\newcommand{\Line}{\typestyle{Line}}
\newcommand{\ALine}{\typestyle{ALine}}
\newcommand{\DLine}{\typestyle{DLine}}
\newcommand{\Pair}{\typestyle{Pair}}
\newcommand{\Constraint}{\typestyle{Constraint}}
\newcommand{\Variable}{\typestyle{Variable}}
\newcommand{\Pauli}{\typestyle{Pauli}}
\newcommand{\Sample}{\typestyle{Sample}}
\newcommand{\Read}{\typestyle{Read}}
\newcommand{\MeasureX}{\typestyle{MeasureX}}
\newcommand{\Hide}[1]{\typestyle{Hide}_{#1}}
\newcommand{\HideX}[1]{\typestyle{HideX}_{#1}}
\newcommand{\Target}[1]{\typestyle{Target}_{#1}}
\newcommand{\Oracle}{\typestyle{Oracle}}
\newcommand{\Introspect}{\typestyle{Intro}}
\newcommand{\Intro}{\typestyle{Intro}}
\newcommand{\Simple}{\typestyle{Simple}}
\newcommand{\Eval}{\typestyle{Eval}}
\newcommand{\Agg}{\typestyle{Agg}}
\newcommand{\Input}{\typestyle{Input}}
\newcommand{\Skip}{\typestyle{Skip}}
\newcommand{\Alice}{\typestyle{Alice}}
\newcommand{\Bob}{\typestyle{Bob}}
\newcommand{\Edge}{\typestyle{Alice}}
\newcommand{\Vertex}{\typestyle{Bob}}
\newcommand{\Anchor}{\typestyle{Anchor}}
\renewcommand{\Game}{\typestyle{Game}}
\newcommand{\AB}{\{\alice, \bob\}}
\newcommand{\ctrl}{\labelstyle{c}}
\newcommand{\target}{\labelstyle{t}}

\newcommand{\abc}[1][\delta]{\otimes I_\bob \simeq_{#1} I_\alice \otimes}

\newcommand{\ldc}{k} % number of copies of classical ld tests

\newcommand{\class}[1]{\ensuremath{\mathsf{#1}}\xspace}
\newcommand{\NP}{\class{NP}} %
\newcommand{\IP}{\class{IP}} %
\newcommand{\EXP}{\class{EXP}} %
\newcommand{\NEXP}{\class{NEXP}} %
\newcommand{\QMA}{\class{QMA}} %
\newcommand{\QMIP}{\class{QMIP}} %
\WithSuffix\newcommand\QMIP*{\ensuremath{\class{QMIP}^*}} %
\newcommand{\PSPACE}{\class{PSPACE}} %
\newcommand{\PCP}{\class{PCP}} %
\newcommand{\MIP}{\class{MIP}} %
\newcommand{\MIPco}{\class{MIP}^{\mathrm{co}}} %
\newcommand{\RE}{\class{RE}} %
\newcommand{\coRE}{\class{coRE}}
\newcommand{\NEEXP}{\class{NEEXP}} %
\newcommand{\NEEEXP}{\class{NEEEXP}}
\WithSuffix\newcommand\MIP*{\ensuremath{\class{MIP}^*}} %
\newcommand{\QIP}{\class{QIP}} %


\newcommand{\Ent}{\mathscr{E}}
\newcommand{\compr}{\textsc{Compr}}
\newcommand{\halt}{\textsc{Halt}}
\newcommand{\machine}{\cal{M}}
\renewcommand{\cal}[1]{\mathcal{#1}}
\newcommand{\Kleene}{\cal{K}}
\newcommand{\qldparams}{\mathsf{qldparams}}
\mathchardef\mhyphen="2D
\newcommand{\Fqldparams}{\F_2\mhyphen\mathsf{qldparams}}
\newcommand{\introparams}{\mathsf{introparams}}
\newcommand{\ldparams}{\mathsf{ldparams}}
\newcommand{\tmldparams}{\mathsf{tmldparams}}
\newcommand{\pcpparams}{\mathsf{pcpparams}}

\newcommand{\TMtoSAT}{\mathrm{TMtoSAT}}
\newcommand{\TMtoLD}{\mathrm{TMtoLD}}
\newcommand{\BoundedHalting}{\mathrm{BH}}
\newcommand{\timecomplexity}{\mathsf{TIME}}
\newcommand{\TIME}{\mathsf{TIME}}
\newcommand{\answer}{\mathsf{ANS}}
\newcommand{\MS}{\mathrm{MS}}

\newcommand{\accept}{\typestyle{Accept}}
\newcommand{\reject}{\typestyle{Reject}}

\newcommand{\anch}{\gamestyle{Anch}}
\newcommand{\ans}{\gamestyle{ANS}}
%%%%%%%self testing macros%%%%%%%%%%

\newcommand{\local}{\mathrm{local}}
%\newcommand{\aux}{\mathrm{aux}}


\newcommand{\G}{\mG}
\newcommand{\XZ}{\mathcal{B}}
\newcommand{\hilb}{\mathcal{H}}


%\newcommand{\tmstyle}[1]{\ensuremath{\textsf{#1}}}
\newcommand{\tmstyle}[1]{\ensuremath{\mathsf{#1}}}
\newcommand{\Compress}{\tmstyle{Compress}}
\newcommand{\ComputeRepetitions}{\tmstyle{ComputeRepetitions}}
\newcommand{\ComputeSampler}{\tmstyle{ComputeSampler}}
\newcommand{\RawIntroSampler}{\tmstyle{RawIntroSampler}}
\newcommand{\ComputeIntroSampler}{\tmstyle{IntroSampler}}
\newcommand{\RawIntroDecider}{\tmstyle{RawIntroDecider}}
\newcommand{\ComputeIntroDecider}{\tmstyle{IntroDecider}}
\newcommand{\ComputeIntroVerifier}{\tmstyle{IntroVerifier}}
\newcommand{\ComputeOracleVerifier}{\tmstyle{OracleVerifier}}
\newcommand{\ComputeAnsVerifier}{\tmstyle{AnsRedVerifier}}
\newcommand{\ComputeParrepVerifier}{\tmstyle{RepeatedVerifier}}
\newcommand{\ComputePCPVerifier}{\tmstyle{PCPVerifier}}
\newcommand{\ComputeFixedPoint}{\tmstyle{ComputeFixedPoint}}
\newcommand{\detype}{\tmstyle{Detype}}

\newenvironment{gamespec}{
  \begin{mdframed}[style=figstyle]}{
  \end{mdframed}}

\newcommand{\zero}{\mathrm{zero}}

%%%%%%%From NW19:%%%%%%%%%%
\newcommand{\polymeas}[3]{\mathrm{PolyMeas}(#1,#2,#3)}
\newcommand{\simulpolymeas}[4]{\mathrm{PolyMeas}(#1,#2,#3, #4)}

\newcommand{\eval}{\mathrm{eval}}

%\newcommand{\coin}{o}
\newcommand{\succinctdecider}{\ensuremath{\mathsf{SuccinctDecider}}}
\newcommand{\circuit}{\mathcal{C}}
\newcommand{\formula}{\mathcal{F}}
\newcommand{\bin}{\mathrm{binary}}
\newcommand{\pcpeval}{\Xi}
\newcommand{\pcpverifier}{\mathcal{M}_\ar}
\newcommand{\qlen}{Q}
\DeclareMathOperator{\ev}{eval}

\newcommand{\coded}{\mathrm{Dec}}
\newcommand{\hx}{\hat{x}}
\newcommand{\hz}{\hat{z}}
\newcommand{\htvar}{\hat{\tvar}}
\newcommand{\soundness}{\mathrm{sound}}

\newcommand{\rep}{\gamestyle{Rep}}
\newcommand{\sep}{\gamestyle{Sep}}

\newcommand{\binary}[1]{\mathrm{binary}_{#1}}
\newcommand{\num}[1]{\mathrm{number}_{#1}}
\newcommand{\canbasis}[1]{\mathrm{basis}(#1)}
\newcommand{\canH}[3]{H_{\mathrm{canon}, #1, #2, #3}}
\newcommand{\canlilh}[3]{h_{\mathrm{canon}, #1, #2, #3}}
\newcommand{\canin}[3]{\pi_{\mathrm{canon},#1,#2,#3}}
\newcommand{\canenc}[4]{g_{\mathrm{canon},#1,#2,#3,#4}}


% \usepackage{showlabels}
% \renewcommand{\showlabelfont}{\tiny\ttfamily\color{red}}

\bibliographystyle{alpha}

\newif\ifnotes\notestrue
%\newif\ifnotes\notesfalse


% MARGIN NOTES

\ifnotes
\usepackage{color}
\definecolor{mygrey}{gray}{0.50}
\newcommand{\notename}[2]{{\textcolor{mygrey}{\footnotesize{\bf (#1:} {#2}{\bf ) }}}}
\newcommand{\noteswarning}{{\begin{center} {\Large WARNING: NOTES ON}\endnote{Warning: notes on}\end{center}}}
\newcommand{\notesendofpaper}{{\theendnotes}}

\newcommand{\pnote}[1]{\textcolor{blue}{\small {\textbf{(MLN:} #1\textbf{)
      }}}}
\newcommand{\tnote}[1]{\textcolor{magenta}{\small {\textbf{(Thomas:} #1\textbf{)
      }}}}
\newcommand{\mnote}[1]{\textcolor{red}{\small {\textbf{(Michael:} #1\textbf{) }}}}
\newcommand{\hnote}[1]{\textcolor{olive}{\small {\textbf{(Henry:} #1\textbf{) }}}}
\newcommand{\ftnote}[1]{\footnote{\textcolor{magenta}{\small {\textbf{(Thomas:} #1\textbf{) }}}}}
\newcommand{\tdnote}[1]{\textcolor{blue}{\small {\textbf{(TODO:} #1\textbf{) }}}}

\else

\newcommand{\notename}[2]{{}}
\newcommand{\noteswarning}{{}}
\newcommand{\notesendofpaper}{}
\newcommand{\pnote}[1]{}

\newcommand{\tnote}[1]{}
\newcommand{\jnote}[1]{}
\newcommand{\anote}[1]{}
\newcommand{\znote}[1]{}
\newcommand{\hnote}[1]{}
%\newcommand{\ftnote}[1]{\footnote{\textcolor{magenta}{\small {\textbf{(Thomas:} #1\textbf{) }}}}}
%\newcommand{\tdnote}[1]{\textcolor{blue}{\small {\textbf{(TODO:} #1\textbf{) }}}}

\fi


\begin{document}

\title{Efficient stability for the Pauli group}

\author[1]{Michael Chapman}
\author[2]{Thomas Vidick}
\author[3]{Henry Yuen}
\affil[1]{}
\affil[2]{}
\affil[3]{}

\date{\today}
\maketitle

\noteswarning


\begin{abstract}

\end{abstract}


\section{Introduction}

\tnote{Make the connection with LCS}


	\section{Preliminaries}

\subsection{Notation}

When we write $\Es{i\in \mX}$ where $\mX$ is a finite set, we mean the expectation over $i$ chosen uniformly at random from $\mX$, i.e.\ $\frac{1}{|\mX|} \sum_{i\in \mX}$. For a vector $u \in \mX^n$ and a subset $S \subseteq [n]$, we write $u_S$ to denote the vector in $\mX^S$ which is the restriction of $u$ to $S$.

\subsection{Algebra}

  A \emph{tracial von Neumann algebra} is a pair $(\mM,\tau)$ of a von Neumann algebra $\mM$ together with a normal faithful tracial state $\tau$ on $\mM$, which we often refer to as the \emph{trace}. The main example of interest is $\mM=M_n(\C)$, the algebra $n\times n$ complex matrices, with $\tau$ the dimension-normalized trace, which we denote $\tr(M)=\frac{1}{n}\Tr(M)$. 	We write $\|x\|_2=\tau(x^*x)^{1/2}$ for the $2$-norm on $\mM$.
	
	Let $B(\ell_2)$ be the von Neumann algebra of bounded operators on $\ell_2$, the Hilbert space of convergent sequences in $\C^\Z$ equipped with the usual Euclidean norm (for which we let $(e_i)_{i \in \Z}$ denote the standard basis). We denote $\mM_\infty = \mM \overline{\otimes} B(\ell_2)$, where the overline denotes closure for the operator topology. $\mM_\infty$ is a von Neumann algebra equipped with the (infinite) trace $\tau_\infty = \tau \otimes \Tr$, with $\Tr(x)=\sum_{i\in \Z} e_i^T X e_i$ the trace on $B(\ell_2)$. We generally identify $\mM$ with the ``corner'' $\mM\otimes e_{1,1}\subset \mM_\infty$. 

	We let $\F$ denote a finite field, and $\field$ the field with two elements. For $u\in \F^n$ for some $n$, we write $|u|$ for the Hamming weight of $u$, i.e.\ the number of nonzero coordinates. For $a,b \in \F^k$, we write $a \cdot b$ to denote the inner product $\sum_{i=1}^k a_i b_i$. 
	
	
	\subsection{Measurements}
	\label{sec:measurements}
	
	A POVM in $\mM$ with outcome set $\mA$ is a finite collection of positive semidefinite operators $\{P_a\}_{a\in \mA}$ such that $\sum_a P_a = \Id_\mM$. A POVM is \emph{projective} if for all $a$, $P_a$ is a projection. 
	Given a projective measurement $\{P_a\}_{a\in \field^k}$ and $b\in \field^k$ we define the corresponding \emph{observable} 
	\[ \widehat{P}(b) = \sum_a (-1)^{a\cdot b} P_a\;,\]
	which is self-adjoint and unitary. If $k=1$, we often use the shorthand $\widehat{P}$ for $\widehat{P}(1) = P_0-P_1$.
	
	We define a specific family of projective measurements on $M_{2^k}(\C)$ which are derived from the \emph{Pauli observables}. Define
	\[ \sigma^X = \begin{pmatrix} 0 & 1 \\ 1 & 0 \end{pmatrix}\;,\qquad \sigma^Z = \begin{pmatrix} 1 & 0 \\ 0 & -1\end{pmatrix}\;,\]
	and more generally for $a,b\in \F_2^k$ let $\sigma^X(a) = \bigotimes_{i=1}^t (\sigma^X)^{a_i}$ and $\sigma^Z(b) = \bigotimes_{i=1}^t (\sigma^Z)^{b_i}$, which are observables in $M_{2^k}(\C)$. These are self-adjoint unitary operators called Pauli observables. Each observable $\sigma^X(a)$ (resp. $\sigma^Z(b)$) corresponds to the \emph{Pauli measurement} $\{ \sigma^X_a \}_{a \in \F_2^k}$ (resp. $\{ \sigma^Z_b \}_{b \in \F_2^k}$) where (in a slight abuse of notation)  
	%	We slightly abuse notation and write 
	\[ \sigma^X_a = \Es{\alpha\in\F_2^k} (-1)^{a\cdot \alpha} \sigma^X(\alpha)\qquad\text{and}\qquad\sigma^Z_b = \Es{\beta\in\F_2^k} (-1)^{b\cdot\beta} \sigma^Z(\beta).\]
	It is easy to verify that $\{\sigma^X_a\}_a$ and $\{\sigma^Z_b\}_b$ are projections summing to identity.	
	
	
	We will make use of the following. It is an application of \emph{orthonormalization}, which transforms a nearly-orthogonal measurement to a nearby orthogonal measurement. See e.g.~\cite{kempe2011parallel,ji2020quantum} or~\cite[Theorem 1.2]{de2021orthogonalization} for the version that we use here. 
	
\begin{lemma}\label{lem:pull-back}
Let  $(\mM,\tau^\mM)$ be a tracial von Neumann algebra, $P\in\mM_\infty$ a projection of finite trace, $\mN=P\mM_\infty P$ and $\tau^\mN=\tau_\infty/\tau_\infty(P)$, and $w\in P \mM_\infty \Id_\mM$ a partial isometry. Let 
\[ \eps = \max\big\{ \tau^\mM\big(\Id_\mM - w^* w\big)\,,\;\tau^\mN\big( P- w w^*\big)\big\}\;.\] 
 Then for any projective measurement $\{P_a\}_{a \in \mA}$ on $\mN$, there is a projective measurement $\{Q_a\}_{a \in \mA}$ on $\mM$ such that 
\begin{equation}
\label{eq:pull-back} \sum_{a \in \mA} \big\| Q_a - w^* P_a w\big\|_2^2 \,\leq \ 56\eps\;.
\end{equation}
\end{lemma}	

\begin{proof}
If $\eps\geq \frac{1}{2}$ the conclusion is trivial (for a suitably large implicit constant in the $O(\cdot)$ notation in \eqref{eq:pull-back}), so assume $\eps<\frac{1}{2}$. 
Define 
\[\tilde{Q}_a = w^* P_a w  + \frac{1}{|\mA|}\big(\Id_\mM - w^* w\big) \in \mM\;.\]
Then $\{\tilde{Q}_a\}$ is a POVM on $\mM$. Moreover, 
\begin{align*}
\sum_a \tau^\mM \big( \tilde{Q}_a^2 \big) &\geq \sum_a \tau^\mM \big( \big(w^* P_a w \big)^2 \big) \\
&= \sum_a \tau^\mM \big(  w^* P_a w w^*P_a w \big)\\
&= \sum_a \tau^\mM \big(  w^* P_a  P P_a w \big) - \sum_a \tau^\mM \big( w^* P_a  ( P - w w^*) P_a w \big)\\
&\geq 1 - \eps -  \sum_a \tau_\infty \big( w^* P_a  ( P - w w^*) P_a w \big)\\
\intertext{
\hnote{I don't get this next line... how did $\tau_\infty$ show up?}\tnote{By definition, $\mM$ is seen as a sub-algebra of $\mM_\infty$, and $\tau_\infty$ restricted to that sub-algebra is $\tau^\cM$. So, whenever $M\in \mM$, $\tau^\mM(M)=\tau_\infty(\mM)$ is a legal ``shortcut'' in the writing. It's this sentence about the ``corner'' in the prelims...should we add something more explicit? Or explain this line here more, since it's the first time we make the manipulation? }\hnote{I think it would be good to add at least one line of explanation...}\tnote{added (extra line above and justification below)}
}
&\geq 1 - \eps -  \tau_\infty\Big(\big( P - w w^*\big)\Big(\sum_a  P_a w w^* P_a\Big)\Big)\\ 
&\geq 1- \eps- \tau_\infty\big( P- w w^*\big)\;,
\end{align*}
where the third line uses that $P_aPP_a=P_a$, $\sum_a P_a = \Id_\mN$ and the definition of $\eps$ for the first term, and for the second the fact that for $A\in\mM$, $\tau^\mM(A)=\tau_\infty(A)$ by definition of $\tau_\infty$ and the identification of $\mM$ with a ``corner'' in $\mM_\infty$, the fourth line uses cyclicity of the trace for the second, and the last uses $\|ww^*\|,\|\sum_a P_a\|_\infty\leq 1$. By assumption, 
\begin{align*}
\tau_\infty\big( P- w w^*\big) \,\leq\, \eps\, \tau_\infty(P)\,\leq \frac{\eps}{1-\eps}\;.
\end{align*}
where the last inequality is because by definition, $\tau^N(P)=1$, thus
\[1-\eps \,\leq\, \tau^\mN(ww^*) \,=\, \frac{\tau_\infty(ww^*)}{\tau_\infty(P)}\,=\, \frac{\tau_\infty(w^*w)}{\tau_\infty(P)}  \,\leq\, \frac{1}{\tau_\infty(P)}\]
since $\tau_\infty(w^* w) = \tau^\mM(w^* w)$ and $w^*w\leq I_\mM$. Overall, 
\[ \sum_a \tau^\mM \big( \tilde{Q}_a^2 \big) \,\geq\, 1-\eps-\frac{\eps}{1-\eps}\,\geq\, 1-3\eps\;.\]
To conclude we apply~\cite[Theorem 1.2]{de2021orthogonalization} to obtain a projective measurement $\{Q_a\}$ on $\mM$ such that 
\begin{equation*}
\sum_a \big\|{Q}_a - \tilde{Q}_a \big\|^2_2 \,=\, 27\eps\;.
\end{equation*}
Finally,
\begin{align*}
\sum_a \big\|{Q}_a - w^*{P}_a w\big\|^2_2 &= \sum_a \Big\|{Q}_a - \tilde{Q}_a  + \frac{1}{|\mA|}\big(I_\mM - w^* w\big) \Big\|^2_2\\
&\leq  \sum_a 2\big\|{Q}_a - \tilde{Q}_a\big\|_2^2  + 2\frac{1}{|\mA|}\big\|I_\mM - w^* w\big\|_2^2 \\
&\leq 54 \eps + 2 \tau^\mM( (I_\mM - w^* w)^2 ) \\
&\leq 54 \eps + 2 \tau^\mM(I_\mM - w^* w ) \\
&\leq 56 \eps\;,
\end{align*}
where the second line is by the triangle inequality, the fourth line is due to the fact that $I_\mM - w^* w$ is positive and has operator norm at most $1$, and the last line is by $\tau^\mM(I_\mM - w^* w ) \leq \eps$.
\end{proof}

	
\subsection{Nonlocal games}

We give standard definitions on nonlocal games. 

\begin{definition}[Game]
A game is a tuple $(\mX,\mu,\mA,D)$ where $\mX$ is a finite set, $\mu$ a distribution on $\mX\times \mX$, $\mA=(\mA(x))_{x\in\mX}$ a collection of finite sets, and 
\[ D: \big\{ (x,y,a,b) : (x,y)\in\text{supp}(\mu),a\in\mA(x),b\in\mA(y)\big\} \;\to\;\{0,1\}\]
such that $D$ is symmetric, i.e. $D(x,y,a,b)=D(y,x,b,a)$ whenever both terms are defined. We often abuse notation and write $\mu$ for the symmetrized marginal of $\mu$, i.e.\ 
\[\mu(x) := \sum_{x'\in \mX} \frac{1}{2}\big(\mu(x,x')+\mu(x',x')\big)\;.\]
\end{definition}

The interpretation of $G=(\mX,\mu,\mA,D)$ as a nonlocal game is the following. In the ``game'', a referee is imagined to sample a pair of ``questions'' $(x,y)\sim \mu$. The question $x$ is sent to a first player, ``Alice,'' and the question $y$ is sent to a second player, ``Bob.'' Each player is tasked with responding with an answer $a\in \mA(x)$ for Alice, and $b\in \mA(y)$ for Bob. The referee accepts the players' answers if and only if $D(x,y,a,b)=1$. 

Nonlocal games provide a framework to study different kinds of bipartite correlations: depending on the level of coordination allowed between Alice and Bob, they may have varying chances of success in the game. 

In quantum mechanics, a local strategy for the players (meaning that each player is required to determine their answer locally, without exchanging information with the other player) is specified by the following. 

\begin{definition}[Synchronous strategy]
If $G=(\mX,\mu,\mA,D)$ is a game and $(\mM,\tau)$ a tracial von Neumann algebra, a \emph{synchronous strategy $\strategy$ for $G$ on $(\mM,\tau)$} is, for every $x\in \mX$, a projective measurement $(P^x_a)_{a\in \mA(x)}$ on $\mM$. The value of a strategy $\strategy$ in $G$ is 
\[ \omega(G;\strategy)\,=\, \sum_{(x,y)\in\mX\times\mX}\frac{1}{2}\big(\mu(x,y)+\mu(y,x)\big) \sum_{(a,b)\in\mA(x)\times\mA(y)} D(x,y,a,b)\, \tau\big(P^x_a \,P^y_b\big) \;.\footnote{Note the symmetrization of $\mu$. This is to avoid explicitly requiring $\mu$ to be permutation-invariant in the definition of a game.}\]
We say that $\strategy$ is \emph{perfect} if $\omega(G;\strategy)=1$.
\end{definition}
	
The name \emph{synchronous} stems from the fact that whenever an identical pair $(x,x)$ is chosen, $\tau(P^x_a P^x_b)=0$ for $a\neq b$ due to the requirement that $\{P^x_a\}_a$ is a projective measurement. Thus a synchronous strategy always returns the same answer to the same question. More general strategies, which allow different operators $\{P^x_a\}$ and $\{Q^y_b\}$, do not automatically enforce the synchronicity condition, but we do not consider such strategies here. 
	
		
\section{Stability}

\subsection{Approximate homomorphisms}

Given a set $S$, we let $\mF(S)$ denote the free group generated by $S$. We freely identify functions from $S$ to $H$, where $H$ is any group, with homomorphisms from $\mF(S)$ to $H$. If $R$ is a subset of $\mF(S)$ then the quotient of $\mF(S)$ by the normal subgroup generated by $R$ is denoted $\langle S:R\rangle$. 

\tnote{not sure we need this:}For convenience we adopt the following notation from~\cite{slofstra2019set}. A \emph{group over $\Z_2$} is a pair $(G,J)$ of a finitely presented group $G$ and a central element $J$ of $G$ of order $2$. Any such group has a presentation $G=\langle S:R\rangle$ where $J\in S$ and $R$ includes the relations $J^2=e$ and $[J,s]=e$ for every $e\in S\backslash\{J\}$. We use the notation 
\[ \langle S:R\rangle_{\Z_2} \,=\, \langle S\cup\{J\} : R\cup\{[J,s]=e:s\in S\}\cup\{J^2=e\}\rangle\;.\]

In~\cite[Section 2]{hadwin2018stability} a notion of $\eps$-\emph{almost homomorphism} from a finitely presented group to a unital tracial $C^*$-algebra $\mA$ is introduced. Informally, an $\eps$-almost homomorphism of $G=\langle S:R\rangle$ is a map from $S$ to $\mU(\mA)$ that approximately respects all relations in $R$. We give a variant of their definition that quantifies the error in an average sense. 

\begin{definition}
Let $G = \langle S:R\rangle $ be a finitely presented group, $\mu$ a distribution on $R$, and $(\mM,\tau)$ a tracial von Neumann algebra. An $(\eps,\mu)$-almost homomorphism of $G$ on $(\mM,\tau)$ is a homomorphism $\phi:\mF(S)\to\mU(\mM)$ such that
\[ \Es{r\sim \mu} \big\|  \phi(r) - \Id \|_\tau^2 \,\leq\, \eps\;.\]
\end{definition}

We note that this notion depends on the presentation of $G$, not only on the group itself. 
When the distribution $\mu$ is uniform over the set $R$, we simply write $\eps$-homomorphism. The definition is consistent with the usual notion of a (unitary) homomorphism, which is recovered when $\eps=0$. 

A stability result states that $\eps$-homomorphisms are close to homomorphisms. To measure the distance between homomorphisms into different algebras we make the following definition. 


\begin{definition}[Closeness]\label{def:close}
Let $\{U_i\}\subseteq \mM$ and $\{V_i\}\subseteq \mN$ be two families of unitaries on  tracial algebras $(\mM,\tau^\mM)$ and $(\mN,\tau^\mN)$ respectively, indexed by the same set $i\in \mI$. For $\delta\geq0$ and $\mu$ a measure on $\mI$ we say that $\{U_i\}$ and $\{V_i\}$ are $(\delta,\mu)$-close if there exists a projection $P\in\mM_\infty$ of finite trace such that $\mN=P\mM_\infty P$ and $\tau^\mN=\tau_\infty/\tau_\infty(P)$, and a partial isometry $w\in P \mM_\infty \Id_\mM$ such that 
\[ \Es{i\sim\mu} \big\| U_i - w^* V_i w \big\|_{\tau^\mM}^2 \,\leq\,\delta\]
and 
\[\max\big\{ \tau^\mM(\Id_\mM-w^*w)\,,\; \tau^\mN(P-ww^*)\big\} \,\leq\, \delta\;.\]
If the measure $\mu$ is omitted then it is understood to be the uniform measure on $\mI$.
\end{definition}

We now give our definition of stability. 

\begin{definition}\label{def:eff-stab}
Let $G = \langle S:R\rangle $ be a finitely presented group. Let $\mC$ be a class of tracial von Neumann algebras. Let $\mu_S$ be a distribution on $S$ and $\mu_R$ a distribution on $R$. For $\delta:[0,1]\to[0,1]$ such that $\lim_{t\to 0}\delta(t)=0$ we say that $G=\langle S:R\rangle$ is $(\delta,\mu_S,\mu_R,\mC)$-stable if for every $(\cM,\tau)$ in $\mC$, every $(\eps,\mu_R)$-almost homomorphism of $G$ is $(\delta(\eps),\mu_S)$-close to a unitary representation of $G$ on some $(\mN,\tau^\mN)\in \mC$. 
\end{definition}

\subsection{General results}

We do not know many general results about stability. We give two. First of all, 
for a finite group $G$ we can always write $G=\langle S:R\rangle$ where $S = G$ and $R=\{ g\cdot h \cdot (gh)^{-1} =e \}$. We refer to this presentation as the \emph{exhaustive presentation}. If we let $\mu_S$ and $\mu_R$ be the uniform distribution on $S$ and on $R$ respectively then Definition~\ref{def:eff-stab} reduces to a widely used notion of \emph{flexible (Hilbert-Schmidt) stability}. In particular, for finite groups the following result is known~\cite{gowers2017inverse,de2019operator}. We adopt the formulation from~\cite[Theorem 1.4]{de2022spectral}.

\begin{theorem}\label{thm:gh}
Let $G$ be a finite group and $\mC$ the class of all tracial von Neumann algebras. Let $U_S$ and $U_R$ be the uniform distribution on $S=G$ and $R=\{ g\cdot h \cdot (gh)^{-1}=e \}$ respectively. Then $G=\langle S:R\rangle$ is $(O(\eps),U_R,U_C,\mC)$-stable. 
\end{theorem}

Next we state a result from de la Salle~\cite{de2022spectral} which allows to combine stability results. The results in~\cite{de2022spectral} are general, and in particular allow to show that the direct product of stable groups is stable. Here, we will only use the following specialization to the case of $\Z_2^k$. For a measure $\mu$ on $\Z_2^k$, define its spectral gap 
\[ \kappa = \max_{a\neq 0} \frac{1}{1-\Es{b\sim\mu}(-1)^{a\cdot b}}\;.\] 

\begin{theorem}[\cite{de2022spectral} Corollary 2.6]\label{thm:dls-gap}
Let $\mu$ be a measure on $\Z_2^k$ with spectal gap $\kappa$. Let $\phi_X,\phi_Z: \Z_2^k \to \mU(\mM)$ be two homomorphisms such that
\[ \Es{a,b\sim \mu} \big\| \phi_X(a)\phi_Z(b)-(-1)^{a\cdot b} \phi_Z(b)\phi_X(a)\big\|_\tau^2 \,\leq\,\eps\;.\]
Then there is an $\mN=P\mM_\infty P$ and homomorphisms $U_X,U_Z:\Z_2^k\to\mU(\mN)$ and $\delta=O(\kappa^2\eps)$ such that $\phi_X$ and $U_X$ are $\delta$-close, $\phi_Z$ and $U_Z$ are $\delta$close, and moreover $U_X(a)U_Z(b)=(-1)^{a\cdot b}U_Z(b)U_X(a)$ for all $a,b\in\Z_2^k$.
\end{theorem}


\subsection{Examples}

As a first example we spell out the application of Theorem~\ref{thm:gh} to the case of $G=\Z_2^k$. 

\begin{corollary}\label{cor:lin-test} \tnote{Stated this for finite dimensions. Not sure what we want in the end}
Let $d\geq 1$ and $\phi:\Z_2^k \to \mU(\C^d)$ be such that 
\[ \Es{x,y\in \Z_2^k} \big\| \phi(x)\phi(y)-\phi(x+y) \big\|_{\tau}^2 \,\leq\,\eps\;.\]
Then there is a %$d'=(1+O(\eps))d$, an isometry $w:\C^d \to \C^{d'}$ and a 
projective measurement $\{P_u\}_{u\in \Z_2^k}$ on $\mA$ such that 
\[ \Es{x\in \Z_2^k} \Big\| \phi(x) -\Big(\sum_u (-1)^{u\cdot x} P_u\Big)  \Big\|_{\tau_d}^2 \,=\, O(\eps)\;.\]
\end{corollary} 

\begin{proof}
Any $\phi$ as in the corollary statement is an $(\eps,U_R)$-almost homomorphism of $\Z_2^k$ into $(\mA,\tau_d)$ for the exhaustive presentation. Applying Theorem~\ref{thm:gh}, $\phi$ is $O(\eps)$-close to a homomorphism to $(\mN,\tau^\mN)$. Because $\Z_2^k$ is Abelian, such a homomorphism is given by commuting unitaries $(U_x)_{x\in\Z_2^k}$ on $\C^{d'}$. Moreover, since $\Z_2^k$ is a $2$-group, each $U_x$ satisfies $U_x^2=\Id$, hence $U_x=U_x^*$.

For every $u\in  \Z_2^k$ let $Q_u = \Es{x} (-1)^{u\cdot x} U_x$. Then each $Q_u$ is a projection such that $\sum_u Q_u=\Id$. Using Lemma~\ref{lem:pull-back}, we find a projective measurement $\{P_u\}$ on $\mA$ that satisfies the conclusions of the corollary. \tnote{more details needed} 
\end{proof}

There is a more efficient presentation of $\Z_2^k$, given by

\begin{equation}\label{eq:z2-efficient}
 \Z_2^k = \langle x_1,\ldots,x_k : [x_i,x_j]=e, x_i^2=e \; \forall i\neq j \rangle\;.
\end{equation}

We call this presentation ``efficient'' because it has a number of generators and relations that is far smaller than those of the exhaustive presentation; in this case, polylogarithmic. This presentation is also stable. 



\begin{lemma}[Lemma 3.8 in~\cite{slofstra2019set}]\label{lem:eff-z2}
For every $k$, there is a $\delta_k = O_k(\eps)$ such that the presentation~\eqref{eq:z2-efficient} is $\delta_k$-stable. Furthermore, the close representation can be taken on the same algebra. 
\end{lemma}

As one would naturally expect, the dependence of $\delta_k$ on $k$ depends on the girth of $\Z_2^k$ for the presentation~\eqref{eq:z2-efficient}. By expressing each element of $\Z_2^k$ as a product of generators in the natural way, applying Corollary~\ref{cor:lin-test} it is possible to show that $\delta_k=O(k^2\eps)$ in Lemma~\ref{lem:eff-z2}. With more work one can get $\delta_k=O(k\eps)$, see~\cite[Theorem 3.2]{chao2017overlapping}. This is nearly tight, as shown in~\cite{chao2017overlapping}.

We end with a non-Abelian group that plays an important role in applications in quantum information, the \emph{Pauli group}. For an integer $k\geq 1$,  the Pauli group $\pauli_k$ can be defined as follows. Let $\gamma: \Z_2^k\times \Z_2^k \to \{-1,1\}$ be given by $\gamma(a,b)=(-1)^{a\cdot b}$. Then $\pauli_k$ is the central extension of $\Z_2^k\times \Z_2^k$ by $\{-1,1\}$ given by $\gamma$. Alternatively, $\pauli_k$ is the group generated by the Pauli matrices 
\[ \sigma_X(a) = , \sigma_Z(b) =\]
for $a,b\in \Z_2^k$. It is also known as the Heisenberg group $H_{2k+1} = \{\begin{pmatrix} 1 & a & c \\ 0 & 1_k & b \\ 0 & 0 & 1 \end{pmatrix} \}\subseteq GL_{k+2}(\F_2)$. Concretely, we label elements of the Pauli group as triples $(J,a,b)$ where $J$ is a special label and $a,b\in\Z_2^k$, and consider the following presentation for it:
\[ \pauli_k \,=\, \langle (a,b), a,b\in \Z_2^k : (a,b)(a',b') = J^{a\cdot b'} (a',b')(a,b) \rangle_{\Z_2}\;.\] 
This presentation is not quite the exhaustive presentation (it has $2^{2k}+1$ generators, when $|\pauli_k|=2^{2k+1}$), but it is not far from it.  
Applying Theorem~\ref{thm:gh} we obtain the following consequence, which we again state only for the case of a subclass of approximate homomorphisms and the class of finite-dimensional matrix algebras\tnote{again, not sure what we want ultimately}. 

\begin{corollary}[Pauli braiding test]
Let $\phi_X,\phi_Z:\Z_2^k \to \Obs(\C^d)$ be such that for all $W\in \{X,Z\}$,
\[ \Es{a,b\in \Z_2^k} \big\| \phi_W(a)\phi_W(b)-\phi_W(a+b) \big\|_F^2 \,\leq\,\eps\;,\]
and
\[ \Es{a,b\in \Z_2^k} \big\| \phi_X(a)\phi_Z(b)- (-1)^{a\cdot b} \phi_Z(b)\phi_X(a) \big\|_{\tau_d}^2 \,\leq\,\eps\;.\]
Then there is a $d'=(1+O(\eps))d2^{-k}$ and an isometry $w:\C^d \to (\C^2)^{\otimes k} \otimes \C^{d'}$ such that for all $W\in \{X,Z\}$, 
\[ \Es{W\in \Z_2^k} \big\| \phi_W(a) - w^* \big(\sigma_W(a)\otimes \Id\big) w \big\|_{\tau_d}^2 \,=\, O(\eps)\;.\]
\end{corollary}

\begin{proof}
\tnote{todo}
\end{proof}

Similarly to Lemma~\ref{lem:eff-z2} we can state an ``efficient'' version of the preceding corollary, which applies to the presentation
\begin{equation}\label{eq:pauli-efficient}
 \pauli_k = \langle x_1,\ldots,x_k,z_1,\ldots,z_k : [x_i,x_j]=[z_i,z_j]=[x_i,z_j]=e, x_i^2=z_i^2=e, \{x_i,z_i\}=e \; \forall i\neq j \rangle\;.
\end{equation}

\begin{lemma}\label{lem:eff-pauli}
For every $k$, there is a $\delta'_k = O_k(\eps)$ such that the presentation~\eqref{eq:pauli-efficient} is $\delta_k$-stable.
\end{lemma}

\begin{proof}
This follows from Lemma~\ref{lem:eff-z2} and Theorem~\ref{thm:dls-gap}.
\end{proof}

\section{Presentations from codes}

The goal of this section is to reformulate, in the language of this paper, a result that plays an important role in~\cite{ji2020mip}. To state the result we first introduce a specific linear error-correcting code. 

\subsection{Presentations from codes}
\label{sec:pres-code}

For $q$ a prime power and $n,k,d$ integer, a linear $[n,k,d]_q$ code is a $k$-dimensional subspace $\code$ of $\F_q^n$ such that for all $x\in \code$ such that $x\neq 0$, the Hamming weight $|x|\geq d$. A code can be specified by a \emph{parity check matrix} $h\in \F_q^{m\times n}$ by setting $\code = \ker h$. 

We now specialize to the case where $q=2$. 
A parity check matrix for a code of dimension $k$ implies a presentation in the following way. Introduce $n$ generators $S=\{x_1,\ldots,x_n\}$. Each of the generators is required to be an involution: $x_i^2=e$. For each row $i\in \{1,\ldots,m\}$ of the parity check matrix, introduce a relation 
\[ R_i\,:\; \prod_{1\leq j \leq k} x_j^{h_{ij}}=e \]
that ``verifies'' the parity check associated with the $i$-th row of $j$. Finally, to guarantee that $R_i$ is independent of the order in which the $x_j$ are multiplied, whenever $j\neq j'$ are such that $h_{ij}$ and $h_{ij'}$ are both nonzero we require that $x_j$ and $x_j'$ commute. This can be written succinctly using a relation 
\[ R'_{ijj'}\,:\; [x_j,x_{j'}]^{h_{ij} h_{ij'}}=e\;.\]
The presentation obtained in this way defines a group $G=G(h)$ as 
\begin{equation}\label{eq:def-gh-pres}
 G(h) \,=\, \big\langle x_1,\ldots,x_n \,:\, x_i^2=e\,,\; R_i\,,\; R'_{ijj'}\,,\quad \forall 1\leq i\leq n,\, 1\leq j< j' \leq m\big\rangle\;.
\end{equation}
Note that we made the dependence of $G(h)$ on $h$ explicit. This is because in general, the group defined in this way may depend on $h$. If however we further impose pairwise commutation relations then it is not hard to see the following. 

\begin{lemma}\label{lem:com-code}
Let $R''_{jj'}$ be the commutation relation $[x_j,x_{j'}]=e$. Then
\[ \Z_2^k \,=\, \big\langle x_1,\ldots,x_n \,:\, x_i^2=e\,,\; R_i\,,\; R''_{jj'}\,,\quad \forall 1\leq i\leq n,\, 1\leq j< j' \leq m\big\rangle\;.\]
\end{lemma}

\begin{proof}
The group defined by the right-hand side is obviously Abelian and a $2$-group, so it is of the form $\Z_2^{k'}$ for some $k'$. In fact, it is equal to the quotient of $Z_2^n$ by the subgroup generated by the $\prod_{1\leq j \leq k} x_j^{h_{ij}}$. So it is $\Z_2^k$ where $k$ is the dimension of the kernel of $h$.  
\end{proof}

\begin{remark}
There exists matrices $h$ such that $G(h)$ is not $\Z_2^k$, and in fact is not Abelian. For an example, see e.g.~\cite[Example 2.16]{paddock2022arkhipov}.  
\end{remark}


For readability it is convenient to reformulate the parity check matrix as a \emph{tester} for the code. This allows us to give a more succinct, ``algorithmic'' definition of a parity check matrix for a given code. Informally, the tester takes as input a word $w\in \F_2^n$ and determines if $w\in \mC$ by selecting a parity check at random and evaluating it. 
 Specifically we give the following definition. (For the sake of later use, we state the definition for the case of a general prime power $q$.)

\begin{definition}[$r$-local linear tester]\label{def:code-test}
Let $\code$ be an $[n,k,d]_q$ linear code and $r\in \N$.
An \emph{$r$-local linear tester for $\code$} is a pair $M = ((v_S)_{S \subseteq [n]},\nu)$ where $\nu$ is a distribution over subsets $S\subseteq [n]$ of size at most $r$, and for each such subset a vector $v_S\in\F_q^S$, such that $u\in \code$ if and only if for all $S$ in the support of $\nu$, $v_S\cdot u_S=0$.\footnote{Generally one imposes stronger ``soundness'' conditions on the tester. However, we will not need to be more specific than this.}
\end{definition}

We conclude with an example, the \emph{Hadamard code}. This code can be defined for any  $t\geq 1$ and it is an $[T,t,T/2]_2$ linear code, where $T=2^t$. For simplicity we write  $\code_\had$ to denote this code, omitting $t$. As a linear map, for $i\in\{1,\ldots,t\}$, $\code_\had(e_i) = (x_i)_{x\in \field^t} \in \field^T$, where we identify the index set $\{1,\ldots,T\}$ with the set $\field^t$ in an arbitrary way.  

A parity check matrix for $\code_\had$ is the matrix $h_\had\in \F_2^{T^2\times T}$ defined as follows. Identify the rows of $h_\had$ with pairs $(x,y)\in \F_2^t\times \F_2^t$, and the columns of $h_\had$ with $\F_2^t$. Then the $(x,y)$-th row of $h$ has nonzero entries at positions $x,y$ and $x+y$ only. The corresponding $3$-local linear tester is $M_\had=((v_S),\nu)$ where $\nu$ is uniform over triples $\{x,y,x+y\}$ for $(x,y)\in \F_2^t\times \F_2^t$ and $v_S = (1,1,1)^T$ for all $S$ in the support of $\nu$. This tester can be described more algorithmically as in Figure~\ref{fig:test-had}. 

\begin{figure}[!htbp]
  \centering
  \begin{gamespec}
	Given access to some $g\in \F_2^n$, where $n=2^t$, identify $g$ with a function $g:\F_2^t\to\F_2$. Perform the following. 
\begin{enumerate}
\item Select $(x,y)\in \F_2^t \times \F_2^t$ uniformly at random. 
\item Accept if and only if $g(x)+g(y)+g(x+y)=0$.  	
    \end{enumerate}
  \end{gamespec}
  \caption{A $3$-local linear tester for $\code_{\had}$}
  \label{fig:test-had}
\end{figure}

\begin{remark}
Because each pair of coordinates $(x,y)$ appears together in at least one parity check, we can apply Lemma~\ref{lem:com-code} to deduce that $G(h_\had)=\Z_2^t$. 
\end{remark}

We state our first stability result for a code-based presentation, the presentation $G(h_\had)$ defined as~\eqref{eq:def-gh-pres} where $h_\had$ is defined above. To state the result we need to specify distributions $\mu_S$ and $\mu_R$. We let $\mu_S$ be uniform over the $2^t$ generators, and $\mu_R$ uniform over the relations $R_i$, for $i\in \{1,\ldots,T^2\}$. Note that here, we do not need to place any weight on the commutation relations.  

\begin{lemma}\label{lem:had-stab}
Let $\mC$ be the class of all tracial von Neumann algebras. 
The presentation $\Z_2^t= G(h_\had)$, together with the distributions $\mu_S$ and $\mu_R$ defined above, is $(\delta,\mC)$ stable with $\delta(\eps)=O(\eps)$. 
\end{lemma}

\begin{proof}
The result is an immediate application of Theorem~\ref{thm:gh}.

\end{proof}



\subsection{The Reed-Muller code over $\F_q$}
\label{sec:rmq}

Fix integers $m,t \in \N$ and let $q=2^t$ and $M = 2^m$. Let $\mP(q,m,d)$ be the vector space over $\F_q$ that consists of all $m$-variate polynomials $f$ over $\F_q$ of individual degree at most $d$, that is all functions of the form
\[
	f(x_1,\ldots,x_m) = \sum_{\alpha \in \{0,1,\ldots,d\}^m} c_\alpha\,
  x_1^{\alpha_1} \cdots x_m^{\alpha_m}\;,
\]
where $\{c_\alpha\}$ is a collection of coefficients in $\F_q$. It is easy to verify that $\mP(q,m,d)$ has dimension $k = (d+1)^m$ over $\F_q$. It follows that the map $\mC_\RM: (c_\alpha) \mapsto (f(x))_{x\in \F_q^m}$ defines a $[q^m,(d+1)^m,D]_q$ linear code over $\F_q$, where $D\geq (1-md/q)q^m$ follows from the Schwartz-Zippel lemma.

\begin{lemma}[Schwartz-Zippel lemma~\cite{Sch80,Zip79}]
  \label{lem:schwartz-zippel}
  Let $f, g: \F_q^m \to \F_q$ be two unequal polynomials with total degree at most $d$. Then
  \begin{equation*}
    \Pr_{x \sim \F_q^m}\big(f(x) = g(x)\big) \leq \frac{d}{q}\;.
  \end{equation*}
\end{lemma}


We define a local linear tester $M_{\RM}$ for the code $\mC_\RM$ over $\F_q$. The tester is introduced as an algorithmic procedure in Figure~\ref{fig:RM-tester}. From this description it is straightforward to deduce a description of the tester as a distribution $\nu$ on subsets of $\F_q^n$, where $n=q^m$, together with vectors $v_S\in F_q^S$ for every $S$ in the support of $\nu$. 

To describe the tester we need to introduce interpolation coefficients, as follows. For $u\in \F_q$ and $i\in\{1,\ldots,d\}$ let 
\[ \alpha_{u,i} = \frac{\prod_{\substack{i'=1\\i'\neq i}}^{d+1} i'}{\prod_{\substack{i'=1\\i'\neq i}}^{d+1} (i'-i)}\;.\]
These are defined so that for any $f:\F_q\to\F_q$ of degree at most $d$, $f(u)=\sum_{i=1}^{d+1} \alpha_{u,i} f(u+i)$. The tester verifies this relation along a randomly chosen direction. 
\tnote{Here I chose to specialize the ``line'' to $(d+2)$ consecutive points on it, which is notationally marginally simpler than $(d+2)$ uniformly random points on it. Hopefully it's fine}


%The second test applied by the tester, the subcube commutation test, may seem superfluous, because it always accepts. However, the test is important to show that the code is robust. Note that including the test imposes a non-trivial constraint of pairwise approximate commutation on representations of $\mC_\RM$, and hence also on $\eps$-local presentations. Both tests are easily seen to be linear; in particular, testing that $q$ entries match the truth table of a polynomial of degree at most $d$ can be done by checking $(n-d-1)$ linear equalities, because the set of evaluation vectors  of polynomials of  degree at most $d<m$ at any $m$ distinct points  is a linear subspace of dimension $(d+1)$ of $\F_q^m$. 

\begin{figure}[!htbp]
  \centering
  \begin{gamespec}
Given access to some $g\in \F_q^n$, where $n=q^m$, identify $g$ with a function $g:\F_q^m\to F_q$. Perform the following.
\begin{enumerate}
	\item Sample	$u \sim \F_q^m$ and $j\sim \{1,\ldots,m\}$ uniformly at random. 
	\item Let $e_j=(0,\ldots,0,1,0,\ldots,0)\in \F_q^m$, where the unique $1$ is in the $j$-th position. Accept if and only if $g(u) = \sum_{i=1}^{d+1} \alpha_{u,i} g(u+ie_j)$.  
    \end{enumerate}
  \end{gamespec}
  \caption{A local test for $\code_{\RM}$}
  \label{fig:RM-tester}
\end{figure}



%\begin{theorem}\label{thm:mrm-sound}
%$M_\RM$ has quantum soundness $\delta(\eps)=\poly(m,d)\cdot \poly(\eps,n^{-1})$.
%\end{theorem}

%\begin{proof}
%In~\cite{ji2022quantum} it is shown that the game $G_{\code_\RM,M_\RM}$ (played using $\F_q$ as the base field) is $(\delta,\nu)$-robust, where $\nu$ is the uniform distribution over $\F_q^m \subseteq \mX$ and $\delta$ satisfies $\delta(\eps)=\poly(m,d)\cdot \poly(\eps,n^{-1})$. The theorem follows by the second item of Proposition~\ref{prop:sound-game}. 
%\end{proof}


\subsection{Code composition}
\label{sec:code-comp}

 The Reed-Muller code from the previous section is defined over $\F_q$, for $q$ a prime power. We can transform any $q$-ary code, for $q=2^t$, into a binary code using the idea of \emph{code composition} which we now describe. 

For $q=2^t$ and $a\in \field$ we let $\kappa(a)\in\F_2^t$ denote the binary representation of $a$, taken in a fixed but usually left implicit self-dual basis of $\F_2^t$ over $\F_2$. We extend $\kappa$ to vectors over $\field$ coordinate-wise. We let $\tr(\cdot):\F_q\to\F_2$ denote the trace over $\F_2$. Because we chose a self-dual basis for the binary representation, the trace satisfies $\tr(ab)=\kappa(a)\cdot\kappa(b)$. 

Let $q=2^t$ and $\code$ an $[n,k,d]_q$ linear code. Let $\code_{\Had}$ be the Hadamard code over $\F_2^t$ (introduced at the end of Section~\ref{sec:pres-code}). Let $T=2^t$.
Let $\mC'$ be the $[Tn,tk,d']$ linear code over $\F_2$ defined as follows. Given $a\in (\F_2^t)^{k}$, first map $a\mapsto a'=\kappa^{-1}(a) \in \F_q^{k}$. Then encode $a'$ to $b'=\code_\RM(a')\in \F_q^n$. Finally, return $b=\code_\Had(\kappa(b'))\in(\F_2^T)^n$, where $\code_\Had$ is applied component-wise. Using that $\code_\Had$ has relative distance $\frac{1}{2}$, it is easy to verify that this code has distance $d'\geq dT/2$.

Given an $r$-local $\delta$-tester $M$ for $\code$, there is a natural $rq$-local tester $M'$ for $\code'$ which can be described as follows. Index coordinates of $\code'$ by pairs $(i,\alpha)\in [n]\times\F_2^t$, fixing a bijection between $[Tn]$ and $[n]\times \F_2^t$. Then $\nu'$ is the uniform mixtures of two distributions, $\nu'_1$ and $\nu'_2$. 

To sample from $\nu'_1$, sample $S\sim\nu$ and $\alpha\in \F_2^t$ uniformly at random. If $M$ performed the check $v_S \cdot u=0$, then perform the check 
\begin{align*}
0 &= \tr\big(( v_S \cdot u) \alpha\big)\\
&= \sum_{j\in S} \tr( v_{S,j} u_j \alpha) \\
&= \sum_{j\in S} \kappa(\alpha v_{S,j}) \cdot \kappa(u_j)\;.
\end{align*}
This check can be performed by evaluating $u'$ at the coordinates $(j,\kappa(\alpha v_{S,j}))$ for $j\in S$ and summing the results in $\F_2$. \tnote{Sometimes there is a step of self-correction included in the composition. Here, the marginal distribution on each vector $\kappa(\alpha v_{S,j}) $ is uniform so it seems fine(?)}

To sample from $\nu'_2$, sample $i\sim[n]$ uniformly at random and $x,y\in\F_2^t$ uniformly at random, and return $\{i\}\times\{x,y,x+y\}$. Execute the tester $M_\had$ for the Hadamard code, i.e.\ check if the three corresponding entries of $u'$ sum to zero. 


\subsection{The code game}
\tnote{This section is a dump---what do we need?}

To make the connection with the literature on non-local games we associate a game $G_{\code,M}$ to any code $\code$ and local tester $M$ for it. In the game, one player is asked to provide an assignment to all variables in the set $S$ queried by the tester, that will satisfy its test, while the other player is asked to provide an assignment to a single $i\in S$, and checked for consistency with the first player. The formal definition follows.  

\begin{definition}
Let $\code$ be an $[n,k,d]_q$ linear code and $M$ an $r$-local tester for $\code$. The game $G_{\code,M}$ is defined as follows. We set 
\[\mX = \{ S\subseteq \{1,\ldots,n\},|S|\leq r\} \sqcup\{1,\ldots,n\}\quad\text{and}\quad \mu(S,i)=\frac{1_{i\in S}}{|S|}\nu(S)\;,\]
and for any $S,i\in\mX$, $\mA(S)=\F_q^S$ and $\mA(i)=\F_q$, and finally $D(S,i,a,b)=M(S,a)1_{a_i=b}$. 
\end{definition}

We show the following. It is a variation on~\cite[Proposition 3.4]{slofstra2019set}.

\begin{proposition}\label{prop:sound-game}
Let $\code$ be an $[n,k,d]_q$ linear code and $M$ an $r$-local tester for $\code$ with distribution $\nu$. Let $\delta,\delta':[0,1]\to[0,1]$. Then if $\phi:\code\to \mU(\mA)$ is an $\eps$-homomorphism, there is a strategy $\strategy$ for $G_{\code,\mM}$ on $(\mN,\tau^\mN)$ such that the second prover's projective measurements are $\{\phi(i)\}_{1\leq i \leq n}$ and $\omega^*(\G_{\code,M},\strategy)\geq 1-\eps$. 
\end{proposition}

We note that a converse to the proposition also holds. The converse is easier and we will not need it, so we skip it. 

\begin{proof}
Let $\phi$ be an $\eps$-homomorphism of $G=\langle S:R\rangle$ on $(\mM,A)$. For each $S$ in the support of $\mu$ and $i\neq j \in S$ the constraint $[x_i,x_j]=e$ appears in $R$. Therefore, the restriction of $\phi$ to $S$ is an $\eps_S$-homomorphism of $\Z_2^S$, where $\eps_S$ is such that $\Es{S\sim \mu} \eps_S = O(\eps)$. Applying Lemma~\ref{lem:eff-z2} we deduce that there is a family $\{P^S_a\}_{a\in \F_2^S}$ of projective measurements on $(\mM,A)$ that satisfy 
\[ \Es{S \sim \nu} \Es{i\in S} \sum_{a \in \F_q} \big\| \phi(i)_a - P^S_a \big\|_\tau^2  \,\leq\,O(\eps)\;.\]
\end{proof}

Later we also use the following proposition, which relates the code games obtained before and after code composition. 


\begin{proposition}\label{prop:q-to-2}
Let $\code$ be an $[n,k,d]_q$ code and $M$ an $r$-local tester for it. Let $M'$ be the $rq$-local tester for $\code'$ obtained as above. Let $\strategy'$ be such that $\omega^*(G_{\code',M'},\strategy')\geq 1-\eps$. Then there is a strategy $\strategy$ for $G_{\code,M}$ such that \tnote{some closeness condition} 
\end{proposition}

\begin{proof}
Let $\strategy'$ be such that $\omega^*(G_{\code',M'},\strategy')\geq 1-\eps$. Restricting to questions in the support of $\nu'_2$, for each $i$ we obtain an $\eps_i$ and a strategy $\strategy_i$ for $G_{\had,M_\had}$ that succeeds with probability $1-\eps_i$, and $\Es{i} \eps_i \leq 2\eps$. 

By quantum soundness of $\code_\Had$ (Theorem~\ref{thm:had-qsound}\tnote{adapt}), for each $i$ there exists commuting $\{\hat{A}^{(i,\alpha)}\}$ that are $O(\eps_i)$-close to the $\{A^{(i,\alpha)}\}$ and moreover are a representation of $\code_{\Had}$. Let $w^{(i)}$ be the implied isometry. For every $a\in \F_2^t$, define $\hat{A}^{i}_{a}=\Es{\alpha}(-1)^{a\cdot \alpha} \hat{A}^{(i,\alpha)}$. Then by linearity this is a projective measurement:
\begin{align*}
\big(\hat{A}^{i}_{a}\big)^2 &=\Big(\Es{\alpha}(-1)^{a\cdot \alpha} \hat{A}^{(i,\alpha)}\Big)^2\\
&= \Es{\alpha,\alpha' }(-1)^{a\cdot (\alpha+\alpha')} \hat{A}^{(i,\alpha)}\hat{A}^{(i,\alpha')}\\
&=\Es{\alpha,\alpha' }(-1)^{a\cdot (\alpha+\alpha')} \hat{A}^{(i,\alpha+\alpha')}\\
&=\hat{A}^{i}_{a}\;,
\end{align*}
where the third line uses that  $\{A^{(i,\alpha)}\}$ are a representation of $\code_{\Had}$. Moreover, $\sum_a \hat{A}^{i}_{a} = \hat{A}^{(i,0)}=\Id$. Hence using Lemma~\ref{lem:pull-back}, for every $i$ we obtain a projective measurement $\{\tilde{A}^{i}_{a}\}$ on $\cM$ such that
\begin{equation}\label{eq:qto2-1a}
 \sum_a \big\| \tilde{A}^i_a - (w^{(i)})^* \hat{A}^i_a (w^{(i)}) \big\|_2^2 \,=\, O(\eps_i)\;.
\end{equation}
%Moreover,
%\begin{equation}\label{eq:qto2-1}
%\Es{i} \Es{\alpha}  \Big\| A^{(i,\alpha)} - \sum_a (-1)^{a\cdot\alpha}(w^{(i)})^* \hat{A}^i_a (w^{(i)})\Big\|^2
%\,=\, \Es{i} \Es{\alpha}  \big\| A^{(i,\alpha)} - (w^{(i)})^* \hat{A}^{i,\alpha} (w^{(i)})\big\|^2 \,=\, O(\eps)\;,
%\end{equation}
%by closeness. Combining~\eqref{eq:qto2-1a} with~\eqref{eq:qto2-1} through the triangle inequality gives
We then get
\begin{align}
 \Es{i} \sum_a \Big\| \Es{\alpha} (-1)^{a\cdot \alpha} A^{(i,\alpha)} -  \tilde{A}^i_a \Big\|_2^2 
&\leq  2\Es{i} \sum_a \Big\| \Es{\alpha} (-1)^{a\cdot \alpha} A^{(i,\alpha)} -  (w^{(i)})^* \hat{A}^i_a (w^{(i)}) \Big\|_2^2 + O(\eps)\notag\\
&=  2\Es{i} \sum_a \Big\| \Es{\alpha} (-1)^{a\cdot \alpha} A^{(i,\alpha)} -   \Es{\alpha} (-1)^{a\cdot \alpha} (w^{(i)})^* \hat{A}^{i,\alpha} (w^{(i)}) \Big\|_2^2+ O(\eps) \notag\\
&= 2\Es{i} \Es{\alpha} \big\| A^{(i,\alpha)} -    (w^{(i)})^* \hat{A}^{i,\alpha} (w^{(i)}) \big\|_2^2+ O(\eps)\notag\\
&= O(\eps)\;, \label{eq:qto2-2}
\end{align}
where the first line uses the triangle inequality and~\eqref{eq:qto2-1a}, the second line uses the definition of $\hat{A}^i_a$, the third line is Parseval's identity and the last is by closeness. 

Now for $b\in \F_q$ define $\tilde{B}^{S,i}_b= \Es{\alpha} (-1)^{b\cdot \alpha} B^{S\times \F_2^t,(i,\alpha)}$, which for the same reasons as earlier is a projective measurement. 
To conclude we show that $\{\tilde{A}^{i}\}$ and $\{\tilde{B}^{S,i}\}$ form a strategy in $G_{\code, M}$ that succeeds with probability $1-O(\eps)$. This is because
\begin{align*}
\Es{S}\Es{i} \sum_b \big\| \tilde{A}^i_b - \tilde{B}^{S,i}_b \big\|_2^2
&\leq 2\Es{S}\Es{i} \sum_b \Big\| \Es{\alpha} (-1)^{b\cdot \alpha} {A}^{(i,\alpha)} - \Es{\alpha} (-1)^{b\cdot \alpha} B^{S\times \F_2^t,(i,\alpha)}\Big\|_2^2 + O(\eps)\\
&= 2\Es{S}\Es{i} \Es{\alpha}\big\|  {A}^{(i,\alpha)} -  B^{S\times \F_2^t,(i,\alpha)}\big\|_2^2 + O(\eps)\\
&\leq 4\eps + O(\eps)\;,
\end{align*}
where the first inequality is by~\eqref{eq:qto2-2} and the triangle inequality, the second line by Parseval's formula and the last is by assumption. 
\end{proof}

\section{An efficient presentation for $\Z_2^k$}

Fix integers $m,t,d \in \N$ and let $q=2^t$. Let $\code_{\bRM}$ be the $[q^{m+1},tk,D']_2$ code obtained by applying the composition procedure from Section~\ref{sec:code-comp} to the $[q^m,(d+1)^m,D]_q$ Reed-Muller code $\code_\RM$ from Section~\ref{sec:rmq}. Let $N=q^{m+1}$ and $h_{\bRM}\in \F_2^{M\times N}$ the parity check matrix for $\code_{\bRM}$ obtained from the composition of the $(d+2)$-local tester for $\code_\RM$ with the $3$-local tester for $\code_\had$. Then $h_\bRM$ has $M= m\cdot q^m \cdot (1+q^2)$ rows, indexed by $(u,j)\in \F_1^m\times [m]$ and $\F_q^m\times [m] \times \F_q^2$ respectively, such that each row has at most $3(d+2)$ nonzero entries. 

Let $G_\bRM = G(h_\bRM)$ be the group that is presented from $h_\bRM$. Explicitly, $h_\bRM$ is the parity check matrix that arises from the tester described in Figure~\ref{fig:bRM-tester}. 


\begin{figure}[!htbp]
  \centering
  \begin{gamespec}
Given access to some $g\in (\F_2^t)^n$, where $n=q^m$, identify $g$ with a function $g:(\F_2^t)^m \times \F_2^t \to F_2$. Perform one of the following tests with probability~$\tfrac{1}{2}$ each. \tnote{Needs to be filled in}
\begin{enumerate}
	\item \textbf{Low-degree test:}
		Let $u \sim \F_q^m$ be a uniformly random point and $j\sim \{1,\ldots,m\}$ chosen uniformly at random. Let $e_j=(0,\ldots,0,1,0,\ldots,0)\in \F_q^m$, where the unique $1$ is in the $j$-th position. 
	\item \textbf{Hadamard test:} 	
    \end{enumerate}
  \end{gamespec}
  \caption{A local test for $\code_{\bRM}$}
  \label{fig:bRM-tester}
\end{figure}

We do not know if $G_\bRM = \Z_2^K$, with $K=t(d+1)^m$. Instead we modify the presentation $G(h_\bRM)$ by adding pairwise commutation relations in a similar manner as Lemma~\ref{lem:com-code}. Let 
\[ G(h_\bRM) \,=\, \big\langle x_1,\ldots,x_N \;:\; \{R^\sq_k\}\,,\; \{R^\ld_k\}\,,\; \{R^\had_{k}\} \big\rangle\;,\]
where  $R^\sq_k$ ranges overal all relations of the form $x_i^2=e$ for $i\in \{1,\ldots,N\}$, $R^\ld_k$ ranges over all relations implied by the ``low-degree test'' in Figure~\ref{fig:bRM-tester} and $R^\had_{k}$ ranges over all the relations implied by the ``Hadamard test.'' For $k=(i,j)\in\{1,\ldots,N\}$ such that $i<j$ let $R^\com_k$ be the relation $[x_i,x_j]=e$.
Then we define
\begin{equation}\label{eq:z2k-eff}
 G' \,=\,\big\langle x_1,\ldots,x_N \;:\; \{R^\sq_k\}\,,\; \{R^\ld_k\}\,,\; \{R^\had_{k}\}\, , \; \{ R^\com_k\}\big\rangle\;.
\end{equation}
From Lemma~\ref{lem:com-code} it follows that $G'=\Z_2^K$. Our main result is an efficient stability result for this presentation. To state this we need to introduce distributions $\mu_S$ and $\mu_R$ on the generators and relations of $G'$. The distribution $\mu_S$ is taken to be the uniform distribution over $[N] = (\F_2^t)^m \times \F_2^t$. The distribution $\mu_R$ is obtained as follows. With probability $1/4$ each, a relation from $\{R^\sq_k\}$, $\{R^\ld_k\}$ or $\{R^\had_k\}$ is chosen uniformly at random. With probability $1/4$, a random commutation relation from $R^\com_k$ is chosen according to the following distribution. First select $j\in\{1,\ldots,m\}$ and $u_{m-j+2},\ldots,u_m \in \F_2^t$ uniformly at random. Then select $v,v' \in (\F_2^t)^m$ uniformly at random, conditioned on the last $(j-1)$ coordinates matching $u_{m-j+2},\ldots,u_m$. Finally, select $\alpha,\beta\in \F_2^t$ uniformly at random. Check commutation between $x_{u,\alpha}$ and $x_{u,\beta}$. 



\begin{theorem}\label{thm:z2-stab}
Let $\mC$ be the class of all tracial von Neumann algebras. 
The presentation of  $\Z_2^K$ given in~\eqref{eq:z2k-eff}, together with the distributions $\mu_S$ and $\mu_R$ defined above, is $(\delta,\mC)$ stable with $\delta(\eps)=$. 
\end{theorem}


\begin{proof}\tnote{in progress}
Let $(\mM,\tau)$ be a tracial von Neumann algebra and $\phi$ be an $\eps$-homomorphism of $\langle S:R\rangle$ on $(\mU(\mM),\tau)$. 
 Here, $S = \{s_{u,\alpha}: u\in (\F_2^t)^m, \alpha\in \F_2^t\}$ and $R$ is the set of all relations in~\eqref{eq:z2k-eff}, i.e.\ $R= \{R^\ld_k\}\cup\{R^\had_k\}\cup\{R^\com_k\}$. We first show that we may assume without loss of generality that $\phi$ sends each element of $S$ to a Hermitian involution. 

\begin{claim}\label{claim:z2-stab-1}
There is an $\eps^{(1)}=O(\eps)$ and an $\eps^{(1)}$-homomorphism $\phi^{(1)}$ of $\langle S:R\rangle$ on $(\mU(\mM),\tau)$ such that $\phi^{(1)}(x)$ is a Hermitian involution for all $x\in S$, and furthermore
\[ \Es{x\sim\mu} \big\| \phi(x) - \phi^{(1)}(x) \big\|_\tau^2 \,\leq\, \eps^{(1)}\;.\] 
\end{claim}

\begin{proof}
Using elementary calculations (see e.g.~\cite[Lemma 3.6]{slofstra2019set}) we see that for any complex $\alpha$, 
\[ \big| \sgn\Re\alpha-\alpha\big| \,\leq\, \Big(1+\frac{1}{\sqrt{2}}\Big) \big|\alpha^2 -1 \big|\;.\]
For any $i\in \{1,\ldots,N\}$ let $\phi^{(1)}(x_i) = \sgn\Re (\phi(x_i))$. Then the claim follows since 
\[ \Es{i\in\{1,\ldots,N\}} \big\| \phi(x_i)^2-\Id \big\|_\tau^2 \,\leq\, 4\eps\;,\]
by assumption and the definition of $\mu_R$. 
\end{proof}

For ease of notation we relabel $\phi^{(1)}$ and $\eps^{(1)}$ as $\phi$ and $\eps$ respectively. We first exploit the relations $\{R^\had_k\}$ to show the following. 

\begin{claim}\label{claim:z2-stab-2}
Let $q=2^t$. 
For every $u\in \F_q^m$ there is a projective measurement $\{P^u_\alpha\}_{\alpha\in \F_q}$ on $\mM$ such that 
\[ \Es{u\in \F_q^m} \Es{\alpha\in \F_2^t} \big\| \phi(s_{u,\alpha}) - \sum_{\beta\in\F_2^t} (-1)^{\alpha \cdot \beta} P^u_\beta \big\|_\tau^2 \,=\, O(\eps)\;. \]
\end{claim}

\begin{proof}
Since $\mu_R$ places weight $1/4$ on relations $\{R^\had_k\}$ we deduce that 
\begin{equation}\label{eq:stab-rm-1}
\Es{u\in \F_q^m} \Es{\alpha,\beta\in \F_2^t} \big\|\phi(s_{u,\alpha})\phi(s_{u,\beta})\phi(s_{u,\alpha+\beta})-\Id\big\|_\tau^2 \,\leq\, 4\eps\;. 
\end{equation}
Fix an $u\in \F_q^m$ and apply Lemma~\ref{lem:had-stab} for that $u$. This gives a partial isometry $w\in P\mM_\infty\Id_\mM$ and a family of commuting unitaries $\{U_\alpha\}$ on $\mN=P\mM_\infty P$ such that 
\[ \Es{\alpha \in \F_2^t} \big\| \phi(s_{u,\alpha}) - w^* U_\alpha w \big\|_\tau^2 \,=\, O(\eps_u)\;,\]
where $\eps_u\geq 0$ is such that $\Es{u\in \F_q^m} \eps_u = 4\eps$. Because $\{U_\alpha\}$ are a representation of $\F_2^t$, there is a projective measurement $\{Q^u_\alpha\}_{\alpha\in \F_2^t}$ on $\mN$ such that $U_\alpha = \sum_\beta (-1)^{\alpha\cdot \beta} Q^u_\beta$. ($\{Q^u_\alpha\}$ can be found explicitly by applying the Fourier transform.) Applying Lemma~\ref{lem:pull-back}, we deduce a projective measurement $\{P^u_\alpha\}_{\alpha\in \F_2^t}$  on $\mM$ such that (by the triangle inequality)
\[ \Es{\alpha \in \F_2^t} \big\| \phi(s_{u,\alpha}) - \sum_{\beta\in\F_2^t} (-1)^{\alpha \cdot \beta} P^u_\beta \big\|_\tau^2 \,=\, O(\eps_u)\;.\]
\end{proof}

The following claim will prove useful.

\begin{claim}\label{claim:z2-stab-2b}\tnote{This will require a specific distribution on the commutation relations}
\[ \Es{u\in \F_q^m} \Es{j\in\{1,\ldots,m\}} \Es{i\neq i' \in \{0,\ldots,d+1\}} \big\| [P^{u+ie_j}_\alpha, P^{u+i'
 e_{j}}_\beta ]\big\|_\tau^2 = O(\eps)\;. \]
\end{claim}

\begin{proof}
From Claim~\ref{claim:z2-stab-2} we have that 
\[ \Es{u\in \F_q^m} \Es{\alpha\in \F_2^t} \big\|\Es{\alpha\in \F_2^t} (-1)^{\alpha \cdot \beta} \phi(s_{u,\alpha}) -  P^u_\beta \big\|_\tau^2 \,=\, O(\eps)\;. \]
The claim therefore follows from the commutation relations. 
\end{proof}

Next we exploit the relations $\{R^\ld_k\}$ to obtain the following. 

\begin{claim}\label{claim:z2-stab-3}
For every axis-parallel line $\ell\subseteq \F_q^m$ there is a projective measurement $\{Q^\ell_g\}$ on $\mM$ such that 
\[ \Es{u\in \F_q^m} \sum_f \tau\big( P^u_{f(u)} Q^\ell_g\big) \,\geq\, 1-\poly(\eps)\;. \]
\end{claim}

\begin{proof}
Fix a direction $j\in \{1,\ldots,m\}$, an $u\in \F_q^m$, and let $u_i = u+ i e_j$ for $i\in \{0,\ldots,d+1\}$. Let $\{P^{u_i}_\alpha\}$ be the projective measurement obtained from Claim~\ref{claim:z2-stab-2}, and $U_i = \sum_\alpha \omega_q^\alpha P^u_\alpha$, where $\omega_q = e^{2i\pi/q}$. Then $U_i \in \mU(\mM)$. Furthermore, from Claim~\ref{claim:z2-stab-2b} we immediately get that
\begin{equation}\label{eq:z2-stab-3-1}
\Es{i\neq j \in \{0,\ldots,d+1\}} \big\| [U_i,U_j] \big\|_\tau^2 = O(\eps)\;.
\end{equation}
We define $\psi: \Z_q^{d+2} \to \mU(\mM)$ by 
\[\psi(x_1,\ldots,x_{d+2}) \,=\, U_1^{x_1} \cdots U_{d+1}^{x_{d+1}}\;,\]
which is well-defined since each $U_i$ has order $q$. Using~\eqref{eq:z2-stab-3-1}, the map $\psi$ is an $O(d\eps)$-approximate homomorphism of $\Z_q^{d+2}$ on $\mU(\mA)$. 

This allows us to apply Theorem~\ref{thm:gh} to obtain a partial isometry $w\in P\mM_\infty\Id_\mM$ and a family of commuting unitaries $\{V_i\}$ of order $q$ each on $\mN=P\mM_\infty P$ such that 
\[ \Es{i\in \{0,\ldots,d+1\}} \big\| U_i - w^* V_i \big\|_\tau^2 \,=\, O(d\eps)\;.\]
\tnote{Actually this is not quite the right distance measure}
Because $\{V_i\}$ are a representation of $\F_q^{d+1}$, there is a projective measurement $\{Q_\alpha\}_{\alpha\in \F_q^{d+1}}$ on $\mN$ such that $U_\alpha = \sum_\beta (-1)^{\alpha\cdot \beta} Q^u_\beta$. ($\{Q^u_\alpha\}$ can be found explicitly by applying the Fourier transform.) Applying Lemma~\ref{lem:pull-back}, we deduce a projective measurement $\{P^u_\alpha\}_{\alpha\in \F_2^t}$  on $\mM$ such that (by the triangle inequality)
\[ \Es{\alpha \in \F_2^t} \big\| \phi(s_{u,\alpha}) - \sum_{\beta\in\F_2^t} (-1)^{\alpha \cdot \beta} P^u_\beta \big\|_\tau^2 \,=\, O(\eps_u)\;.\]



Let $V_1,\ldots,V_{d+1}$ be commuting unitaries. POVM. Pull-back. 



\end{proof}



\end{proof}








\section{Applications: nonlocal games}
	


\subsection{The Pauli braiding test}

\subsection{Qubit tests}

\subsection{Dimension bounds}


Formulate the PBT as an LCS. Get soundness as a corollary. 





\bibliography{qld}

\notesendofpaper

\end{document}
