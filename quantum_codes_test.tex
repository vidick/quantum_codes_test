\documentclass[11pt]{article}
\usepackage{booktabs}
\usepackage{fullpage}
\usepackage{titlesec}
%\newcommand{\sectionbreak}{\clearpage}
\usepackage{amsmath,amsfonts,amsthm,mathrsfs,xspace,graphicx}
\usepackage[backref,colorlinks,citecolor=blue,bookmarks=true]{hyperref}
\usepackage{mathpazo}
\usepackage{endnotes}
\usepackage{color}
\usepackage{float}
\usepackage{xcolor}
\usepackage{mdframed}
\usepackage{bbm}
\usepackage{suffix} % for *-version commands
\usepackage{times}
\usepackage{tabularx}
\usepackage{makecell}
\usepackage{amssymb,latexsym}
%\usepackage{IEEEtrantools}
\usepackage[capitalize]{cleveref}
\usepackage{enumitem}
\usepackage{tikz}
\usepackage{tikz-cd}
\usepackage{multirow}
\usepackage[section]{placeins}
\usepackage[affil-it]{authblk}


\mdfdefinestyle{figstyle}{ %
  linecolor=black!7, %
  backgroundcolor=black!7, %
  innertopmargin=10pt, %
  innerleftmargin=25pt, %
  innerrightmargin=25pt, %
  innerbottommargin=10pt %
}

\newtheorem{theorem}{Theorem}[section]
\newtheorem{proposition}[theorem]{Proposition}
\newtheorem{conjecture}[theorem]{Conjecture}
\newtheorem{lemma}[theorem]{Lemma}
\newtheorem{claim}[theorem]{Claim}
\newtheorem{fact}[theorem]{Fact}
\newtheorem{corollary}[theorem]{Corollary}

\newtheorem{remark}[theorem]{Remark}

\theoremstyle{definition}
\newtheorem{definition}[theorem]{Definition}
\newtheorem{example}[theorem]{Example}

\newcommand{\beq}{\begin{eqnarray}}
\newcommand{\eeq}{\end{eqnarray}}

\newcommand{\code}{\mathcal{C}}
\newcommand{\strategy}{\mathscr{S}}
\newcommand{\algebra}{\mathscr{A}}

\newcommand{\ket}[1]{|#1\rangle}
\newcommand{\bra}[1]{\langle#1|}
\newcommand{\ketbra}[2]{\ket{#1}\!\bra{#2}}
\newcommand{\ip}[2]{\langle #1 \! | #2 \rangle}
\newcommand{\proj}[1]{\ket{#1}\!\bra{#1}}
\newcommand{\Tr}{\mbox{\rm Tr}}
\newcommand{\Id}{\ensuremath{I}}
\DeclareMathOperator*{\Expectation}{\mathbb{E}}
\newcommand{\Es}[1]{\Expectation_{#1}}

\newcommand{\reg}[1]{{\textsf{#1}}}
\newcommand{\ol}[1]{\overline{#1}}

\newcommand{\field}{\mathbb{F}_2}
\newcommand{\C}{\ensuremath{\mathbb{C}}}
\newcommand{\N}{\ensuremath{\mathbb{N}}}
\newcommand{\bbN}{\ensuremath{\mathbb{N}}}
\newcommand{\complex}{\ensuremath{\mathbb{C}}}
\newcommand{\real}{\ensuremath{\mathbb{R}}}
%\newcommand{\natural}{\ensuremath{\mathbb{N}}}

\newcommand{\bij}{\pi}
\newcommand{\qp}{\tau}
\newcommand{\dlS}{\ensuremath{\rm dlS}}

\newcommand{\F}{\ensuremath{\mathbb{F}}}
\newcommand{\M}{\ensuremath{\mathbb{M}}}
\newcommand{\ot}{\otimes}
\newcommand{\Fp}{\F_p}
\newcommand{\Fq}{\field}
\newcommand{\BH}{\textsc{BH}}
\newcommand{\ld}{\textsc{ld}}
\newcommand{\downsize}{\kappa}
\newcommand{\tobin}{\flat}
\newcommand{\downsizem}{\chi}

\newcommand{\K}{\ensuremath{\mathbb{K}}}
\newcommand{\R}{\ensuremath{\mathbb{R}}}
\newcommand{\Z}{\ensuremath{\mathbb{Z}}}

\newcommand{\mA}{\ensuremath{\mathcal{A}}}
\newcommand{\mB}{\ensuremath{\mathcal{B}}}
\newcommand{\mC}{\ensuremath{\mathcal{C}}}
\newcommand{\mE}{\ensuremath{\mathcal{E}}}
\newcommand{\mD}{\ensuremath{\mathcal{D}}}
\newcommand{\mF}{\ensuremath{\mathcal{F}}}
\newcommand{\mG}{\ensuremath{\mathcal{G}}}
\newcommand{\mH}{\ensuremath{\mathcal{H}}}
\newcommand{\mK}{\ensuremath{\mathcal{K}}}
\newcommand{\mM}{\ensuremath{\mathcal{M}}}
\newcommand{\mI}{\ensuremath{\mathcal{I}}}
\newcommand{\mJ}{\ensuremath{\mathcal{J}}}
\newcommand{\cM}{\ensuremath{\mathcal{M}}}
\newcommand{\mP}{\ensuremath{\mathcal{P}}}
\newcommand{\mQ}{\ensuremath{\mathcal{Q}}}
\newcommand{\mR}{\ensuremath{\mathcal{R}}}
\newcommand{\mS}{\ensuremath{\mathcal{S}}}
\newcommand{\mT}{\ensuremath{\mathcal{T}}}
\newcommand{\mX}{\ensuremath{\mathcal{X}}}
\newcommand{\mY}{\ensuremath{\mathcal{Y}}}

\newcommand{\Inv}{\ensuremath{\textsc{Inv}}}
\newcommand{\GEN}{\ensuremath{\textsc{GEN}}}
\newcommand{\SAMP}{\ensuremath{\textsc{SAMP}}}
\newcommand{\epr}{\ensuremath{\textsc{epr}}}
\newcommand{\RM}{\ensuremath{\textsc{RM}}}
\newcommand{\Had}{\ensuremath{\textsc{Had}}}
\newcommand{\HRM}{\ensuremath{\textsc{HRM}}}


\newcommand{\Alg}{\mathcal{A}}
\newcommand{\ind}{\ensuremath{\mathrm{ind}}}


\newcommand{\setft}[1]{\mathrm{#1}}
\newcommand{\Density}{\setft{D}}
\newcommand{\Pos}{\setft{Pos}}
\newcommand{\Proj}{\setft{Proj}}
\newcommand{\Channel}{\setft{C}}
\newcommand{\Unitary}{\setft{U}}
\newcommand{\Herm}{\setft{Herm}}
\newcommand{\Lin}{\setft{L}}
\newcommand{\Trans}{\setft{T}}
\DeclareMathOperator{\poly}{poly}
\DeclareMathOperator{\negl}{negl}
\newcommand{\dset}{G}

\newcommand{\val}{\ensuremath{\mathrm{val}}}
\newcommand{\valco}{\ensuremath{\mathrm{val}^{\mathrm{co}}}}
\newcommand{\ia}{\Id_\alice}
\newcommand{\ib}{\Id_\bob}

\newcommand{\desc}[1]{\overline{\cal{#1}}}
\newcommand{\supp}{\textsc{Supp}}
\newcommand{\Gen}{\textsc{Gen}}
\newcommand{\Enc}{\textsc{Enc}}
\newcommand{\Dec}{\textsc{Dec}}

\newcommand{\GenTrap}{\textsc{GenTrap}}
\newcommand{\Invert}{\textsc{Invert}}
\newcommand{\lossy}{\textsc{lossy}}

\newcommand{\rand}{\textrm{rand}}
\newcommand{\had}{\textsc{Had}}


\newcommand{\eps}{\varepsilon}
\newcommand{\ph}{\ensuremath{\varphi}}


\newcommand{\ac}{\textsc{ac}}
\newcommand{\GX}{\textsc{Gap-Maxcut}}
\newcommand{\GNI}{\textsc{Graph Non-Isomorphism}}


\newcommand{\Acc}{\textsc{Acc}}
\newcommand{\Samp}{\textsc{Samp}}
\newcommand{\Ext}{\ensuremath{\text{Ext}}}

\newcommand{\BD}{\mathbb{QB}}
\newcommand{\DD}{\mathbb{D}}
\newcommand{\DDb}{\mathbb{D'}}
\newcommand{\Pot}{\Phi}
\newcommand{\inj}{J}
\newcommand{\mZ}{\mathcal{Z}}
\newcommand{\mN}{\mathcal{N}}
\newcommand{\vs}{\vspace{2mm}~\newline\noindent}
\newcommand{\vb}{\vspace{3mm}\noindent}
\newcommand{\sX}{\mathcal{X}}
\newcommand{\sA}{\mathcal{A}}
\newcommand{\sB}{\mathcal{B}}
\newcommand{\sY}{\mathcal{Y}}
\newcommand{\sR}{\mathcal{R}}


\newcommand{\trnq}[1]{\left[ {#1} \right]_q}

\DeclareMathOperator{\polylog}{polylog}
\newcommand{\mx}[1]{\mathbf{{#1}}}
\newcommand{\vc}[1]{\mathbf{{#1}}}
\newcommand{\abs}[1]{\left\vert {#1} \right\vert}
\newcommand{\norm}[1]{\left\| {#1} \right\|}
\newcommand{\for}{\text{for }}

\DeclareMathOperator{\arcsinh}{arcsinh}
\DeclareMathOperator{\tr}{tr}

\newcommand{\E}{\mathop{\mathbb{E}}\displaylimits} % Expectation

\newcommand{\unif}{\mathcal{U}}
\newcommand{\pt}{\textrm{pt}}
\newcommand{\sample}{\textrm{sample}}
\newcommand{\test}{\textrm{test}}
\newcommand{\free}{\mathcal{F}}
\newcommand{\plane}{\mathcal{P}}
\newcommand{\lines}{\mathcal{L}}
\newcommand{\clines}{\mathcal{CL}}
\newcommand{\pl}{\mathbf{p}}
\newcommand{\individual}{\textrm{individual}}
\newcommand{\blocks}{\textrm{blocks}}
\newcommand{\liness}{\textrm{lines}}
\newcommand{\lp}{\mathcal{LP}}
\newcommand{\Pl}{\ensuremath{\mathrm{Pl}}}
\newcommand{\Ln}{\ensuremath{\mathrm{Lines}}}
\newcommand{\mode}{\mathfrak{m}}
\newcommand{\ECC}{\ensuremath{\textsc{ECC}}}
\newcommand{\EC}{\ensuremath{\textsc{EC}}}
\newcommand{\ENC}{\ensuremath{\textsc{ENC}}}
\newcommand{\cktval}{\ensuremath{\textsc{CKTVAL}}}


\newcommand{\GL}{\mathrm{GL}}
\newcommand{\Matrix}{\mathrm{M}}
\newcommand{\End}{\mathrm{End}}
\newcommand{\Aut}{\mathrm{Aut}}

\newcommand{\game}{\mathfrak{G}}
\newcommand{\sampler}{\mathcal{S}}
\newcommand{\decider}{\mathcal{D}}
\newcommand{\verifier}{\mathcal{V}}


\newcommand{\type}{\mathcal{T}}
\newcommand{\lt}{\mathcal{L}}
\newcommand{\rt}{\mathcal{R}}
\newcommand{\checker}{\mathcal{C}}


\newcommand{\gamestyle}[1]{\ensuremath{\textsc{#1}}\xspace}
\newcommand{\qld}{\gamestyle{QLD}}
\newcommand{\ms}{\gamestyle{MS}}
\newcommand{\pauli}{\gamestyle{Pauli}}
%\newcommand{\bp}{\gamestyle{BP}}
\newcommand{\ora}{\gamestyle{Orac}}
\newcommand{\pcp}{\gamestyle{PCP}}
\newcommand{\ar}{\gamestyle{AR}}
\newcommand{\intro}{\gamestyle{Intro}}

\newcommand{\labelstyle}[1]{\ensuremath{\textsc{#1}}\xspace}
\newcommand{\EPR}{\labelstyle{EPR}}
\newcommand{\aux}{\labelstyle{aux}}
\newcommand{\ancilla}{\labelstyle{anc}}
\newcommand{\msc}{\labelstyle{MC}}
\newcommand{\msv}{\labelstyle{MV}}
\newcommand{\vertex}[1]{\labelstyle{V#1}}
\newcommand{\edge}[1]{\labelstyle{N#1}}
\newcommand{\basis}{\labelstyle{W}}
\newcommand{\xpt}{\labelstyle{X}}
\newcommand{\zpt}{\labelstyle{Z}}
\newcommand{\rxpt}{\labelstyle{R}_\xpt}
\newcommand{\rzpt}{\labelstyle{R}_\zpt}
\newcommand{\dir}[1]{\labelstyle{V#1}}
\newcommand{\coord}{\labelstyle{I}}
\newcommand{\intercept}{\labelstyle{U}}
\newcommand{\plf}{\labelstyle{Pl}}
\newcommand{\lnf}{\labelstyle{Ln}}
\newcommand{\ptf}{\labelstyle{Pt}}
\newcommand{\full}{\labelstyle{full}}
\newcommand{\opt}{\labelstyle{opt}}
\newcommand{\partition}{\mathcal{B}}

\newcommand{\tvarstyle}[1]{\mathsf{#1}}
\newcommand{\tvar}{\ensuremath{\tvarstyle{t}}}
\newcommand{\lvar}{\ensuremath{\tvarstyle{u}}}
\newcommand{\rvar}{\ensuremath{\tvarstyle{v}}}
\newcommand{\pvar}{\ensuremath{\tvarstyle{p}}}
\newcommand{\ovar}{\ensuremath{\tvarstyle{o}}}
\newcommand{\trole}{\ensuremath{v}} % used in intro types

\newcommand{\types}{\labelstyle{T}}

\newcommand{\decode}{\labelstyle{Decode}}

%\newcommand{\alice}{\labelstyle{Alice}}
%\newcommand{\bob}{\labelstyle{Bob}}
\newcommand{\alice}{\labelstyle{A}}
\newcommand{\bob}{\labelstyle{B}}

\newcommand{\oracle}{\labelstyle{Oracle}}
\newcommand{\ab}{\{\alice, \bob\}}

\newcommand{\typestyle}[1]{\ensuremath{\textsc{#1}}\xspace}
\newcommand{\Type}{\typestyle{Type}}
\newcommand{\Plane}{\typestyle{Plane}}
\renewcommand{\line}{\mathbf{\ell}}
\newcommand{\Llane}{\typestyle{Line}}
\newcommand{\Point}{\typestyle{Point}}
\newcommand{\HPoint}{\typestyle{HPoint}}
\newcommand{\Line}{\typestyle{Line}}
\newcommand{\ALine}{\typestyle{ALine}}
\newcommand{\DLine}{\typestyle{DLine}}
\newcommand{\Pair}{\typestyle{Pair}}
\newcommand{\Constraint}{\typestyle{Constraint}}
\newcommand{\Variable}{\typestyle{Variable}}
\newcommand{\Pauli}{\typestyle{Pauli}}
\newcommand{\Sample}{\typestyle{Sample}}
\newcommand{\Read}{\typestyle{Read}}
\newcommand{\MeasureX}{\typestyle{MeasureX}}
\newcommand{\Hide}[1]{\typestyle{Hide}_{#1}}
\newcommand{\HideX}[1]{\typestyle{HideX}_{#1}}
\newcommand{\Target}[1]{\typestyle{Target}_{#1}}
\newcommand{\Oracle}{\typestyle{Oracle}}
\newcommand{\Introspect}{\typestyle{Intro}}
\newcommand{\Intro}{\typestyle{Intro}}
\newcommand{\Simple}{\typestyle{Simple}}
\newcommand{\Eval}{\typestyle{Eval}}
\newcommand{\Agg}{\typestyle{Agg}}
\newcommand{\Input}{\typestyle{Input}}
\newcommand{\Skip}{\typestyle{Skip}}
\newcommand{\Alice}{\typestyle{Alice}}
\newcommand{\Bob}{\typestyle{Bob}}
\newcommand{\Edge}{\typestyle{Alice}}
\newcommand{\Vertex}{\typestyle{Bob}}
\newcommand{\Anchor}{\typestyle{Anchor}}
\renewcommand{\Game}{\typestyle{Game}}
\newcommand{\AB}{\{\alice, \bob\}}
\newcommand{\ctrl}{\labelstyle{c}}
\newcommand{\target}{\labelstyle{t}}

\newcommand{\abc}[1][\delta]{\otimes I_\bob \simeq_{#1} I_\alice \otimes}

\newcommand{\ldc}{k} % number of copies of classical ld tests

\newcommand{\class}[1]{\ensuremath{\mathsf{#1}}\xspace}
\newcommand{\NP}{\class{NP}} %
\newcommand{\IP}{\class{IP}} %
\newcommand{\EXP}{\class{EXP}} %
\newcommand{\NEXP}{\class{NEXP}} %
\newcommand{\QMA}{\class{QMA}} %
\newcommand{\QMIP}{\class{QMIP}} %
\WithSuffix\newcommand\QMIP*{\ensuremath{\class{QMIP}^*}} %
\newcommand{\PSPACE}{\class{PSPACE}} %
\newcommand{\PCP}{\class{PCP}} %
\newcommand{\MIP}{\class{MIP}} %
\newcommand{\MIPco}{\class{MIP}^{\mathrm{co}}} %
\newcommand{\RE}{\class{RE}} %
\newcommand{\coRE}{\class{coRE}}
\newcommand{\NEEXP}{\class{NEEXP}} %
\newcommand{\NEEEXP}{\class{NEEEXP}}
\WithSuffix\newcommand\MIP*{\ensuremath{\class{MIP}^*}} %
\newcommand{\QIP}{\class{QIP}} %


\newcommand{\Ent}{\mathscr{E}}
\newcommand{\compr}{\textsc{Compr}}
\newcommand{\halt}{\textsc{Halt}}
\newcommand{\machine}{\cal{M}}
\renewcommand{\cal}[1]{\mathcal{#1}}
\newcommand{\Kleene}{\cal{K}}
\newcommand{\qldparams}{\mathsf{qldparams}}
\mathchardef\mhyphen="2D
\newcommand{\Fqldparams}{\F_2\mhyphen\mathsf{qldparams}}
\newcommand{\introparams}{\mathsf{introparams}}
\newcommand{\ldparams}{\mathsf{ldparams}}
\newcommand{\tmldparams}{\mathsf{tmldparams}}
\newcommand{\pcpparams}{\mathsf{pcpparams}}

\newcommand{\TMtoSAT}{\mathrm{TMtoSAT}}
\newcommand{\TMtoLD}{\mathrm{TMtoLD}}
\newcommand{\BoundedHalting}{\mathrm{BH}}
\newcommand{\timecomplexity}{\mathsf{TIME}}
\newcommand{\TIME}{\mathsf{TIME}}
\newcommand{\answer}{\mathsf{ANS}}
\newcommand{\MS}{\mathrm{MS}}

\newcommand{\accept}{\typestyle{Accept}}
\newcommand{\reject}{\typestyle{Reject}}

\newcommand{\anch}{\gamestyle{Anch}}
\newcommand{\ans}{\gamestyle{ANS}}
%%%%%%%self testing macros%%%%%%%%%%

\newcommand{\local}{\mathrm{local}}
%\newcommand{\aux}{\mathrm{aux}}


\newcommand{\G}{\mG}
\newcommand{\XZ}{\mathcal{B}}
\newcommand{\hilb}{\mathcal{H}}


%\newcommand{\tmstyle}[1]{\ensuremath{\textsf{#1}}}
\newcommand{\tmstyle}[1]{\ensuremath{\mathsf{#1}}}
\newcommand{\Compress}{\tmstyle{Compress}}
\newcommand{\ComputeRepetitions}{\tmstyle{ComputeRepetitions}}
\newcommand{\ComputeSampler}{\tmstyle{ComputeSampler}}
\newcommand{\RawIntroSampler}{\tmstyle{RawIntroSampler}}
\newcommand{\ComputeIntroSampler}{\tmstyle{IntroSampler}}
\newcommand{\RawIntroDecider}{\tmstyle{RawIntroDecider}}
\newcommand{\ComputeIntroDecider}{\tmstyle{IntroDecider}}
\newcommand{\ComputeIntroVerifier}{\tmstyle{IntroVerifier}}
\newcommand{\ComputeOracleVerifier}{\tmstyle{OracleVerifier}}
\newcommand{\ComputeAnsVerifier}{\tmstyle{AnsRedVerifier}}
\newcommand{\ComputeParrepVerifier}{\tmstyle{RepeatedVerifier}}
\newcommand{\ComputePCPVerifier}{\tmstyle{PCPVerifier}}
\newcommand{\ComputeFixedPoint}{\tmstyle{ComputeFixedPoint}}
\newcommand{\detype}{\tmstyle{Detype}}

\newenvironment{gamespec}{
  \begin{mdframed}[style=figstyle]}{
  \end{mdframed}}

\newcommand{\zero}{\mathrm{zero}}

%%%%%%%From NW19:%%%%%%%%%%
\newcommand{\polymeas}[3]{\mathrm{PolyMeas}(#1,#2,#3)}
\newcommand{\simulpolymeas}[4]{\mathrm{PolyMeas}(#1,#2,#3, #4)}

\newcommand{\eval}{\mathrm{eval}}

%\newcommand{\coin}{o}
\newcommand{\succinctdecider}{\ensuremath{\mathsf{SuccinctDecider}}}
\newcommand{\circuit}{\mathcal{C}}
\newcommand{\formula}{\mathcal{F}}
\newcommand{\bin}{\mathrm{binary}}
\newcommand{\pcpeval}{\Xi}
\newcommand{\pcpverifier}{\mathcal{M}_\ar}
\newcommand{\qlen}{Q}
\DeclareMathOperator{\ev}{eval}

\newcommand{\coded}{\mathrm{Dec}}
\newcommand{\hx}{\hat{x}}
\newcommand{\hz}{\hat{z}}
\newcommand{\htvar}{\hat{\tvar}}
\newcommand{\soundness}{\mathrm{sound}}

\newcommand{\rep}{\gamestyle{Rep}}
\newcommand{\sep}{\gamestyle{Sep}}

\newcommand{\binary}[1]{\mathrm{binary}_{#1}}
\newcommand{\num}[1]{\mathrm{number}_{#1}}
\newcommand{\canbasis}[1]{\mathrm{basis}(#1)}
\newcommand{\canH}[3]{H_{\mathrm{canon}, #1, #2, #3}}
\newcommand{\canlilh}[3]{h_{\mathrm{canon}, #1, #2, #3}}
\newcommand{\canin}[3]{\pi_{\mathrm{canon},#1,#2,#3}}
\newcommand{\canenc}[4]{g_{\mathrm{canon},#1,#2,#3,#4}}


% \usepackage{showlabels}
% \renewcommand{\showlabelfont}{\tiny\ttfamily\color{red}}

\bibliographystyle{alpha}

\newif\ifnotes\notestrue
%\newif\ifnotes\notesfalse


% MARGIN NOTES

\ifnotes
\usepackage{color}
\definecolor{mygrey}{gray}{0.50}
\newcommand{\notename}[2]{{\textcolor{mygrey}{\footnotesize{\bf (#1:} {#2}{\bf ) }}}}
\newcommand{\noteswarning}{{\begin{center} {\Large WARNING: NOTES ON}\endnote{Warning: notes on}\end{center}}}
\newcommand{\notesendofpaper}{{\theendnotes}}

\newcommand{\pnote}[1]{\textcolor{blue}{\small {\textbf{(MLN:} #1\textbf{)
      }}}}
\newcommand{\tnote}[1]{\textcolor{magenta}{\small {\textbf{(Thomas:} #1\textbf{)
      }}}}
\newcommand{\mnote}[1]{\textcolor{red}{\small {\textbf{(Michael:} #1\textbf{) }}}}
\newcommand{\hnote}[1]{\textcolor{olive}{\small {\textbf{(Henry:} #1\textbf{) }}}}
\newcommand{\ftnote}[1]{\footnote{\textcolor{magenta}{\small {\textbf{(Thomas:} #1\textbf{) }}}}}
\newcommand{\tdnote}[1]{\textcolor{blue}{\small {\textbf{(TODO:} #1\textbf{) }}}}

\else

\newcommand{\notename}[2]{{}}
\newcommand{\noteswarning}{{}}
\newcommand{\notesendofpaper}{}
\newcommand{\pnote}[1]{}

\newcommand{\tnote}[1]{}
\newcommand{\jnote}[1]{}
\newcommand{\anote}[1]{}
\newcommand{\znote}[1]{}
\newcommand{\hnote}[1]{}
%\newcommand{\ftnote}[1]{\footnote{\textcolor{magenta}{\small {\textbf{(Thomas:} #1\textbf{) }}}}}
%\newcommand{\tdnote}[1]{\textcolor{blue}{\small {\textbf{(TODO:} #1\textbf{) }}}}

\fi


\begin{document}

\title{Qubit tests from classical codes}

\author{}
\date{\today}
\maketitle

\noteswarning


\begin{abstract}

\end{abstract}

	\section{Preliminaries}

\subsection{Notation}

When we write $\Es{i\in \mX}$ where $\mX$ is a finite set, we mean the expectation over $i$ chosen uniformly at random from $\mX$, i.e.\ $\frac{1}{|\mX|} \sum_{i\in \mX}$. 

\subsection{Algebra}

  A \emph{tracial von Neumann algebra} is a pair $(\mM,\tau)$ of a von Neumann algebra $\mM$ together with a normal faithful tracial state $\tau$ on $\mM$, which we often refer to as the \emph{trace}. The main example of interest is $\mM=M_n(\C)$, the algebra $n\times n$ complex matrices, with $\tau$ the dimension-normalized trace, which we denote $\tr(M)=\frac{1}{n}\Tr(M)$. 	We write $\|x\|_2=\tau(x^*x)^{1/2}$ for the $2$-norm on $\mM$.
	
	Let $B(\ell_2)$ be the von Neumann algebra of bounded operators on $\ell_2$, the Hilbert space of convergent sequences in $\C^\Z$ equipped with the usual Euclidean norm. We denote $\mM_\infty = \mM \overline{\otimes} B(\ell_2)$, where the overline denotes closure for the operator topology. $\mM_\infty$ is a von Neumann algebra equipped with the (infinite) trace $\tau_\infty = \tau \otimes \Tr$, with $\Tr(x)=\sum_{i\in \Z} e_i^T X e_i$ the trace on $B(\ell_2)$. We generally identify $\mM$ with the ``corner'' $\mM\otimes e_{1,1}\subset \mM_\infty$. 

	We let $\F$ denote a finite field, and $\field$ the field with two elements. For $u\in \F^n$ for some $n$, we write $|u|$ for the Hamming weight of $u$, i.e.\ the number of nonzero coordinates. For $a,b \in \F^k$, we write $a \cdot b$ to denote the inner product $\sum_{i=1}^k a_i b_i$. 
	
	
	\subsection{Measurements}
	
	A POVM on $\mM$ is a finite collection of positive semidefinite operators $\{P_i\}_{i\in \mI}$ such that $\sum_i P_i = \Id_\mM$. A POVM is \emph{projective} if for all $i$, $P_i$ is a projection. 
	Given a projective measurement $\{P_a\}_{a\in \field^k}$ and $b\in \field^k$ we let $\widehat{P}(b) = \sum_a (-1)^{a\cdot b} P_a$ be the associated observable. If $k=1$, we often use the shorthand $\widehat{P}$ for $\widehat{P}(1) = P_0-P_1$.
	
	A specific family of measurements on $M_{2^k}(\C)$ is derived from the \emph{Pauli observables}. Define 
	\[ \sigma^X = \begin{pmatrix} 0 & 1 \\ 1 & 0 \end{pmatrix}\;,\qquad \sigma^Z = \begin{pmatrix} 1 & 0 \\ 0 & -1\end{pmatrix}\;,\]
	and more generally for $a,b\in \F_2^k$ let $\sigma^X(a) = \otimes_{i=1}^t (\sigma^X)^{a_i}$ and $\sigma^Z(b) = \otimes_{i=1}^t (\sigma^Z)^{b_i}$. We slightly abuse notation and write 
	\[ \sigma^X_a = \Es{\alpha\in\F_2^t} (-1)^{a\cdot \alpha} \sigma^X(\alpha)\qquad\text{and}\qquad\sigma^Z_b = \Es{\beta\in\F_2^t} (-1)^{b\cdot\beta} \sigma^Z(\beta)\]
	for the associated POVM elements. 
	
	
	\begin{definition}[Closeness]\label{def:close}
Let $\{A^i_a\}\subseteq \mM$ and $\{B^i_a\}\subseteq \mN$ be two families of projective measurements on  tracial algebras $(\mM,\tau^\mM)$ and $(\mN,\tau^\mN)$ respectively, indexed by the same set $i\in \mI$ and with the same set of outcomes $a,b\in\mA$. For $\delta\geq0$ and $\mu$ a measure on $\mI$ we say that $\{A^i\}$ and $\{B^i\}$ are $(\delta,\mu)$-close if there exists a projection $P\in\mM_\infty$ of finite trace such that $\mN=P\mM_\infty P$ and $\tau^\mN=\tau_\infty/\tau_\infty(P)$, and a partial isometry $w\in P \mM_\infty \Id_\mM$ such that 
\[ \Es{i\sim\mu} \sum_{a\in\mA} \big\| A^i_a - w^* B^i_a w \big\|_2^2 \,\leq\,\delta\]
and 
\[\max\big\{ \tau^\mM(\Id_\mM-w^*w)\,,\; \tau^\mN(P-ww^*)\big\} \,\leq\, \delta\;.\]
If the measure $\mu$ is omitted then it is understood to be the uniform measure on $\mI$.
\end{definition}
	
The following uses orthonormalization. 
	
\begin{lemma}\label{lem:pull-back}
Let  $(\mM,\tau^\mM)$ be a tracial von Neumann algebra, $P\in\mM_\infty$ a projection of finite trace, $\mN=P\mM_\infty P$ and $\tau^\mN=\tau_\infty/\tau_\infty(P)$, and $w\in P \mM_\infty \Id_\mM$ a partial isometry. Let 
\[ \eps = 1-\min\big\{ \tau^\mM\big(\Id_\mM - w^* w\big)\,,\;\tau^\mN\big( P- w w^*\big)\big\}\;.\]
 Then for any projective measurement $\{P_i\}$ on $\mN$, there is a projective measurement $\{Q_i\}$ on $\mM$ such that 
\[ \sum_i \big\| Q_i - w^* P_i w\big\|_2^2 \,=\, O(\eps)\;.\] 
\end{lemma}	

\begin{proof}
If $\eps\geq \frac{1}{2}$ the conclusion is trivial (for a suitably large implicit constant), so assume $\eps<\frac{1}{2}$. 
Define 
\[\tilde{Q}_i = w^* P_i w  + \frac{1}{|\mI|}\big(\Id_\mM - w^* w\big) \in \mM\;.\]
Then $\{\tilde{Q}_i\}$ is a POVM on $\mM$. Moreover, 
\begin{align*}
\sum_{i} \tau\big( (\tilde{Q}_i)^2 \big) &\geq \sum_{i} \tau\big( \big(w^* P_i w \big)^2 \big) \\
&= \sum_{i} \tau\big(  w^* P_i w w^*P_i w \big)\\
&= \sum_i \tau\big(  w^* P_i  P P_i w \big) - \sum_i \tau\big( w^* P_i  ( P - w w^*) P_i w \big)\\
&\geq 1 - \eps -  \tau_\infty\Big(\big( P - w w^*\big)\Big(\sum_i  P_i w w^* P_i\Big)\Big)\\ 
&\geq 1- \eps- \tau_\infty\big( P- w w^*\big)\;,
\end{align*}
where the fourth line uses that $P_iPP_i=P_i$, $\sum_i P_i = \Id_\mN$ and the definition of $\eps$ for the first term, and cyclicity of the trace for the second, and the last uses $\|ww^*\|,\|\sum_i P_i\|_\infty\leq 1$. By assumption, 
\begin{align*}
\tau_\infty\big( P- w w^*\big) \,\leq\, \eps\, \tau_\infty(P)\,\leq \frac{\eps}{1-\eps}\;.
\end{align*}
where the last inequality is because by definition, $\tau^N(P)=1$, thus
\[1-\eps \,\leq\, \tau^\mN(ww^*) \,=\, \frac{\tau_\infty(ww^*)}{\tau_\infty(P)}\,=\, \frac{\tau_\infty(w^*w)}{\tau_\infty(P)}  \,\leq\, \frac{1}{\tau_\infty(P)}\]
\hnote{why is $ww^* \leq 1_\mM$? The input to $w^*$ is from $P$, which belongs to ${\mM}_{\infty}$}\tnote{I had forgotten to use cyclicity once; better?}
since $\tau^\mM$ is normalized and $w^*w\leq 1_\mM$. Overall, 
\[ \sum_{i} \tau\big( (\tilde{Q}_i)^2 \big) \,\geq\, 1-\eps-\frac{\eps}{1-\eps}\,\geq\, 1-3\eps\;.\]
To conclude we apply~\cite[Theorem 1.2]{de2021orthogonalization} to obtain a projective measurement $\{Q_i\}$ on $\mM$ such that 
\begin{equation*}
\sum_i \big\|{Q}_i - \tilde{Q}_i \big\|^2_2 \,=\, O(\eps)\;.
\end{equation*}
Finally,
\begin{align*}
\sum_i \big\|{Q}_i - w^*{P}_i w\big\|^2_2 &= \sum_i \Big\|{Q}_i - \tilde{Q}_i  + \frac{1}{|\mI|}\big(1_\mM - w^* w\big) \Big\|^2_2\\
&\leq  \sum_i 2\big\|{Q}_i - \tilde{Q}_i\big\|_2^2  + 2\frac{1}{|I|}\big\|1_\mM - w^* w\big\|_2^2 \\
&=  O(\eps)\;,
\end{align*}
where the second line is by the triangle inequality. 
\end{proof}

We end with a couple lemma that will prove useful. 

\begin{lemma}\label{lem:l1-l2}
Let $\{P_i\}$ and $\{Q_i\}$ be two projective measurements that are $(\eps,\mu)$-close. Then 
\begin{equation}\label{eq:l1-l2}
\sum_i \tau\big(|P_i-w^*Q_iw|\big) \,\leq\, \eps+ 2 \sqrt{\eps}\;.
\end{equation}
\end{lemma}

\begin{proof}
Using the triangle inequality, 
\begin{align*}
 \sum_i \tau\big(|P_i - w^* Q_i w|\big) &\leq\, \sum_i \Big(\tau\big(|(P_i-P_i^2)|\big) + \tau\big(|P_i(P_i-w^*Q_iw)|\big) + \tau\big(|(P_i-w^*Q_iw)w^*Q_iw|\big) \\
&\qquad +\tau\big(| (w^*Q_iw^*wQ_iw-w^*Q_iw)|\big)\Big)\;.
\end{align*}
The first term on the right-hand side is zero, because $P$ is assumed projective. The terms in the middle are bounded using H\"older's inequality:
\begin{align*}
\sum_i  \tau\big(|P_i(P_i-w^*Q_iw)|\big) &\leq  \|P_i\|_2 \|P_i-w^*Q_iw\|_2\\
&\leq \Big(\sum_i \|P_i\|^2_2\Big)^{1/2}\Big( \sum_i  \|P_i-w^*Q_iw\|_2^2 \Big)^{1/2}\\
&\leq \sqrt{\eps}
\end{align*}
by closeness, and similarly for the second term. 
Finally the last term can be bounded as 
\begin{align*}
\sum_i \tau\big(| (w^*Q_iw^*wQ_iw-w^*Q_iw)|\big) &= \sum_i \tau\big( (w^*Q_i(1-w^*w)Q_iw)\big)\\
&=  \tau\Big( (1-w^*w)\Big(\sum_i Q_iww^* Q_i\Big)\Big)\\
&\leq \tau(1-w^*w) \Big\|\sum_i Q_iww^* Q_i\Big\|\\
&\leq \eps\;.
\end{align*}
\end{proof}

\begin{lemma}[Data-processing]\label{lem:dp}
Let $\{P_i\}$ and $\{Q_i\}$ be two POVM on $(\mM,\tau)$ with the same index set. Then for any $f:\mI\to \mJ$, 
\begin{equation}\label{eq:dp}
\sum_{j\in \mJ} \Big\| \sum_{i\in f^{-1}(j)} (P_i-Q_i) \Big\|_2^2 \,\leq\, \sum_{i\in \mI} \big\| P_i-Q_i\big\|_2^2\;.
\end{equation}
\end{lemma}

\begin{proof}
This follows by expending the left-hand side and using $\tau(P_iQ_j)\geq 0$ for all $i\neq j$. 
\end{proof}
	
\section{Quantum soundness of linear codes}

\subsection{Classical definitions}

\begin{definition}[Code]
For $q$ a prime power and $n,k,d$ integer
a $[n,k,d]_q$ linear code is a $k$-dimensional subspace $\code\subseteq \F_q^n$ such that for every $u\in \code$ such that $u\neq 0$, $|u|\geq d$. Equivalently, we think of $\code$ as a linear map $\code:\F_q^k\to \F_q^n$. We write $G_\code \in \F_q^{k\times n}$ for the generator matrix defined by $ G_\code^T e_i =\code(e_i)$ for all $i\in\{1,\ldots,k\}$, where $\{e_1,\ldots,e_k\}$ is the standard basis of $\F_q^k$. 
\end{definition}

When the subscript $q$ is omitted, it is implicitly taken as $q=2$, i.e.\ the code is assumed to be binary. 

%\begin{definition}
%Let $\code$ be an $[n,k,d]_q$ linear code. Define $\ind:\{1,\ldots,n\}\to \field^k$ such that for every $a\in \field^k$ and $i\in\{1,\ldots,n\}$, $a\cdot \ind(i) = \code(a)_i$.
%\end{definition}

%\begin{definition}[Oracle machine]
%A (non-adaptive) \emph{randomized oracle machine} $M$ is a Turing machine that has read access to two special tapes, the randomness tape and the proof tape. $M$ operates in two stages. In the first stage, $M$ accesses the randomness tape in state $\rand$ and returns a set $S=M(\rand)\subseteq\{1,\ldots,n\}$ of  indices of the proof tape. In the second stage $M$ reads the corresponding entries from the state $\pi$ of the proof tape and returns a value in $\{0,1\}$, which only depends on $S$ and $\pi_S$, not on $\rand$. We write $M(\rand,\pi)$ for the decisions taken by $M$ when the randomness tape is $\rand$ and the proof tape is $\pi$. We also write $M(S,\pi_S)\in\{0,1\}$ for the decision taken by $M$ when the chosen entries are $S$ and their values is $\pi_S$.   
%\end{definition}

\begin{definition}[Codeword test]\label{def:code-test}
Let $\code$ be an $[n,k,d]_q$ linear code, $\delta:[0,1]\to[0,1]$, and $r\in \N$.
An $r$-local $\delta$-tester for $\code$ is a distribution $\nu$ over subsets $S\subseteq [n]$ of size at most $r$, and for each $S$ in the support of $\nu$ a predicate $M(S,\cdot):\F_q^S\to\{0,1\}$, such that the following hold:
\begin{itemize} 
\item (Completeness:) For any $u\in \code$, $\Pr_{S\sim \nu}( M(S,u_S)=1)=1$.
\item (Soundness:) For any $\eps\geq 0$ and any $u\in \F_q^n$ that is within (Hamming) distance at least  $\eps$ from $\code$, $\Pr_{S\sim\nu}(M(S,u_{|S})=0)\geq \delta(\eps)$. 
\end{itemize}
\end{definition}

%\begin{definition}[Local quantum presentation]
%Let $\code$ be an $[n,k,d]_q$ linear code and $r\in \N$. An \emph{$r$-local quantum presentation of $\code$} is specified by
%\begin{itemize}
%\item A tracial von Neumann algebra $(\mM,\tau)$;
%\item For each $i\in\{1,\ldots,n\}$, a $q$-outcome projective measurement $\{A^i_a\}_{a\in \field} \subseteq \mM$;
%\item For each $S\subseteq \{1,\ldots,n\}$ such that $|S|\leq r$, a projective measurement $\{P^S_a\}_{a\in \field^S}\subseteq \mM$.
%\end{itemize}
%\end{definition}

\subsection{Quantum soundness}
\label{sec:q-soundness}

\begin{definition}[Code representation]
Let $\code$ be an $[n,k,d]$ linear code. A \emph{representation} of $\code$ is a collection of projective measurements $\{A^i_a\}_{a\in\F_q} \subseteq\mM$ for $i\in\{1,\ldots,n\}$ such that the $\{A^i_a\}$ pairwise commute and $\sum_{c\in\code} \prod_{i=1}^n A^i_{c_i}=\Id$.\footnote{This implies that whenever $u\notin \code$, $\prod_i A^i_{u_i}=0$.} 
\end{definition}



\begin{definition}[$\eps$-local presentation]
Let $\code$ be an $[n,k,d]_q$ linear code and $M$ an $r$-local tester for $\code$ with distribution $\nu$. An \emph{$\eps$-local presentation of $(\code,M)$} is a collection of projective measurements $\{A^i_a\}_{a\in\F_q} \subseteq\mM$ for $i\in\{1,\ldots,n\}$ such that for any $S\subseteq\{1,\ldots,n\}$ of size $|S|\leq r$ there are $|S|$ pairwise commuting projective measurements $\{B^{S,i}_b\}_{b\in \F_q}$, for $i\in S$, such that 
\begin{itemize}
\item $\sum_{a\in \F^S} 1_{M(S,a)=1} \prod_{i\in S} B^{S,i}_{a_i}= \Id$
\item The two families $\{A^i\}_{S, i\in S}$ and $\{B^{S,i}\}_{S, i\in S}$ are $(\eps,\nu')$-close, where $\nu'((S,i))=\nu(S)1_{i\in S}\frac{1}{|S|}$.
\end{itemize} 
\hnote{The $1_{M(S,a)=1}$ notation conflicts with the $1_{\mM}$ notation above. Also sometimes $I$ is used, othertimes $1$ is used.}\tnote{fixed, now always using $\Id$ for the identity}
\end{definition}


\begin{definition}[Quantum soundness]\label{def:q-sound}
Let $\code$ be an $[n,k,d]_q$ linear code and $M$ a $r$-local tester for $\code$. Let $\delta:[0,1]\to[0,1]$ be such that $\delta(0)=0$. We say that $M$ has \emph{quantum soundness $\delta(\eps)$} if the following holds. For any $\eps$-local presentation $(\mM,A)$ of $(\code,M)$
% such that 
%\[ \Es{S\leftarrow M(\cdot)} \Es{i\in S} \sum_{a\in \field^s} \big[ \tau\big(  P^S_a  A^i_{a_i}\big) \big] \,\geq \,1-\eps\;,\] 
there is a representation $(\mN,B)$ of $\code$ such that $(\mM,A)$ and $(\mN,B)$ are $\delta(\eps)$-close. 
\end{definition}

Definition~\ref{def:q-sound} can be reformulated in the language of nonlocal games, see Proposition~\ref{prop:sound-game}. 

\begin{remark}
A necessary condition for a code to be quantum sound is that any $0$-local quantum presentation $(\mM,A)$ is  a representation of $\mC$. Not all codes satisfy this condition, see e.g.~\cite[Example 2.16]{paddock2022arkhipov}. If $(\mC,M)$ satisfies this condition, then we say that it is \emph{Abelian}. Thus, Abelian codes have, by definition, quantum soundness $\overline{\delta}$ where $\overline{\delta}(0)=0$ and $\overline{\delta}(x)=1$ for $x\in(0,1]$. 
\end{remark}


		
		
\subsection{Example: the Hadamard code}
\label{sec:had}

We give an example of a family of quantum sound codes, the \emph{Hadamard codes}. This code can be defined for any  $t\geq 1$ and it is an $[T,t,T/2]_2$ linear code, where $T=2^t$. For simplicity we write  $\code_\had$ for it, omitting $t$. As a linear map, for $i\in\{1,\ldots,t\}$, $\code_\had(e_i) = (x_i)_{x\in \field^t} \in \field^T$, where we identify the index set $\{1,\ldots,T\}$ with the set $\field^t$ in an arbitrary way.  

It is shown in~\cite{blum1990self} that the following randomized oracle machine $M_\had$ is a $3$-local $\delta$-tester for $\code_\had$, where $\delta(\eps)=6\eps$. $M$ first uses its random tape to select $x,y\in\field^k$uniformly at random and returns the set $S=\{x,y,x+y\}$. For $\pi\in \field^T$, $M(S,\pi_S)$ accepts if and only if $\pi_x + \pi_y  = \pi_{x+y}$. 

The following appears as~\cite[Theorem 10]{natarajan2016robust}.\footnote{While~\cite{natarajan2016robust} only show the result for finite-dimensional $\mM$, the same proof works for any $(\mM,\tau)$.}
 
\begin{theorem}\label{thm:had-qsound}
$M_\had$ has quantum soundness $O(\eps)$.
\end{theorem}

%\subsection{Tensor codes}
%
%Fix a $t_0$-interpolable $[n_0,k_0,d_0]_q$ code $\code_0$, and $m\in \N$. Let $\code = \code_0^{\otimes m}$. Then $\code$ is an $[n,k,d]_q$ code where $n=n_0^m$, $k=k_0^m$ and $d=d_0^m$. 
%
%
%If $\code_0$ is a Reed-Solomon code with degree $s_0=t_0-1$, then $n_0=q$ and $d_0=n_0-s_0=n_0-t_0+1$.


\section{Nonlocal games}

\subsection{Preliminaries}

\begin{definition}[Game]
A game is a tuple $(\mX,\mu,\mA,D)$ where $\mX$ is a finite set, $\mu$ a distribution on $\mX\times \mX$, $\mA=(\mA(x))_{x\in\mX}$ a collection of finite sets, and 
\[ D: \big\{ (x,y,a,b) : (x,y)\in\text{supp}(\mu),a\in\mA(x),b\in\mA(y)\big\} \;\to\;\{0,1\}\]
such that $D$ is symmetric, i.e. $D(x,y,a,b)=D(y,x,b,a)$ whenever both terms are defined. We often abuse notation and write $\mu$ for the symmetrized marginal of $\mu$, i.e.\ 
\[\mu(x) := \sum_{x'\in \mX} \frac{1}{2}\big(\mu(x,x')+\mu(x',x')\big)\;.\]
\end{definition}
		
\begin{definition}[Synchronous strategy]
If $G=(\mX,\mu,\mA,D)$ is a game and $(\mM,\tau)$ a tracial von Neumann algebra, a \emph{synchronous strategy $\strategy$ for $G$ on $(\mM,\tau)$} is, for every $x\in \mX$, a projective measurement $(P^x_a)_{a\in \mA(x)}$ on $\mM$. The value of a strategy $\strategy$ in $G$ is 
\[ \omega(G;\strategy)\,=\, \sum_{(x,y)\in\mX\times\mX}\frac{1}{2}\big(\mu(x,y)+\mu(y,x)\big) \sum_{(a,b)\in\mA(x)\times\mA(y)} D(x,y,a,b)\, \tau\big(P^x_a \,P^y_b\big) \;.\footnote{Note the symmetrization of $\mu$. This is to avoid explicitly requiring $\mu$ to be permutation-invariant in the definition of a game. \hnote{Didn't quite understand... what would go wrong without this symmetrization?}\tnote{It's so that at some point we can claim that studying synchronous strategies is wlog. Maybe that point happens only outside this note though; so then we can say it's to keep with the formalism of synchronous games}}\]
We say that $\strategy$ is \emph{perfect} if $\omega(G;\strategy)=1$.
\end{definition}


The next lemma shows that close strategies have close value. 

\begin{lemma}\label{lem:close-value}
Let $G=(\mX,\mu,\mA,D)$ be a game and $\strategy=\{P^x_a\}$ and $\strategy'=\{Q^x_a\}$ strategies on $\mM$ and $\mN$ respectively such that $\{P^x_a\}$ and $\{Q^x_a\}$ are $(\eps,\mu)$-close. Then 
\[ \big|\omega(G;\strategy) - \omega(G;\strategy')\big|\,\leq\, O\big(\sqrt{\eps}\big)\;.\]
\end{lemma}

\begin{proof}
By definition,
\begin{align*}
\omega(G;\strategy') &= \Es{(x,y)\sim\mu} \sum_{a,b} D(a,b,x,y)  \tau\big( Q^x_a \, Q^y_b \big)\\
&=  \Es{(x,y)\sim\mu} \sum_{a,b} D(a,b,x,y)  \tau\big( w^* Q^x_a w\, w^* Q^y_b w\big)\\
&\quad+  \Es{(x,y)\sim\mu} \sum_{a,b} D(a,b,x,y)  \tau\big( w^* Q^x_a (P - w\, w^* ) Q^y_b w\big)
+ \Es{(x,y)\sim\mu} \sum_{a,b} D(a,b,x,y)  \tau\big( Q^x_a  Q^y_b (P-ww^*)\big)
\end{align*}
The second and third term on the right-hand side can be bounded as
\begin{align*}
\Big|\Es{(x,y)\sim\mu} \sum_{a,b} D(a,b,x,y)  \tau\big( w^* Q^x_a (P - w\, w^* ) Q^y_b w\big)\Big|
&=\Big| \tau\Big(  (P - w\, w^* )\Big( \Es{(x,y)\sim\mu} \sum_{a,b} D(a,b,x,y) Q^y_b ww^* Q^x_a \Big)\Big)\Big|\\
&\leq \tau(P-ww^*) \Big\|\Es{(x,y)\sim\mu} \sum_{a,b} D(a,b,x,y) Q^y_b ww^* Q^x_a \Big\|\\
&\leq \eps\;,
\end{align*}
where the first inequality is by H\"older's inequality and the last inequality uses the definition of closeness to bound the first term. To bound the second term on the right-hand side, we write it as $\|A\circ B\|$, with $\circ$ the Hadamard product, and use  $\|A\circ B\| \leq \max_{i,j} |A_{i,j}|\|B\|$. Here $A = \sum_{a,b} D(a,b) P_{a,b}$, with $P_{a,b} = \sum_{i,j} \ket{u_{i,a}} \bra{v_{j,b}}$ with $\ket{u_{i,a}}$ a basis for the range of $Q^x_a$ and $\ket{v_{j,b}}$ for the range of $Q^y_b$, and $B=\sum_{a,b} Q^y_b ww^* Q^x_a = ww^*$. 

For $x\in \mX$ such that $\mu(x)\neq 0$, denote by $\mu_x$ the (symmetrized) conditional distribution $\mu_x(y)=\frac{1}{2}(\mu(x,y)+\mu(y,x))/\mu(x)$. Then using the above
\begin{align*}
\big|\omega(G;\strategy) - \omega(G;\strategy')\big|
 &\leq 2\eps+ \Big|\Es{(x,y)\sim\mu} \sum_{a,b} D(a,b,x,y) \big( \tau\big( P^x_a \, P^y_b \big)-\tau\big( w^* Q^x_a w\, w^* Q^y_b w\big)\big)\Big|\\
&\leq  2\eps+ \Big|\Es{x\sim\mu} \sum_{a}  \tau\Big( \big(P^x_a-w^* Q^x_aw\big) \, \Big( \Es{y\sim \mu_x} \sum_b D(a,b,x,y) P^y_b \Big)\Big)\Big|\\
&\qquad+ \Big|\Es{y\sim\mu} \sum_{b}  \tau\Big(\Big( \Es{x\sim \mu_y} \sum_a D(a,b,x,y) w^* Q^x_a w\Big) \big(P^y_b-w^*Q^y_bw\big) \, \Big)\Big|\\
&\leq 2\eps+2\Es{x\sim\mu} \sum_{a}  \tau\big(\big| P^x_a-w^*Q^x_aw\big|\big) \\
&\leq 4\eps+4\sqrt{\eps}\;,
\end{align*}
where the third line uses $\tau(AB)\leq\tau(|A|)\|B\|$ and the last inequality is by~\eqref{eq:l1-l2} and Jensen's inequality.
\end{proof}

\subsection{Robustness}


\begin{definition}[Robust game]
Given a game $G$, a function $\delta:[0,1]\to[0,1]$, and a distribution $\nu$ on $\mX$ we say that $G$ is \emph{$(\delta,\nu)$-robust} if any synchronous strategy for $G$ that succeeds with probability at least $1-\eps$ in $G$, for some $\eps\geq 0$, is
 $(\delta(\eps),\nu)$-close to a perfect strategy.
\end{definition}

Note that in the definition $\nu$ need not be the marginal of the game distribution $\mu$. In particular, we will often consider $\nu$ that is supported on a strict subset of the support of $\mu$. 

Let $\code$ be an $[n,k,d]_q$ linear code and $M$ an $r$-local tester for $\code$. Consider the following game $G_{\code,M}$. We set 
\[\mX = \{ S\subseteq \{1,\ldots,n\},|S|\leq r\} \sqcup\{1,\ldots,n\}\quad\text{and}\quad \mu(S,i)=\frac{1_{i\in S}}{|S|}\nu(S)\;,\]
and for any $S,i\in\mX$, $\mA(S)=\F_q^S$ and $\mA(i)=\F_q$, and finally $D(S,i,a,b)=M(S,a)1_{a_i=b}$. 


\tnote{The next proposition sounds like a special case of some other lemma about LCS games, which is probably written somewhere}

\begin{proposition}\label{prop:sound-game}
Let $\code$ be an $[n,k,d]_q$ linear code and $M$ an $r$-local tester for $\code$. Let $\delta,\delta':[0,1]\to[0,1]$. Then the following hold:
\begin{enumerate}
\item If $M$ has quantum soundness $\delta$ then $G_{\code,M}$ \hnote{was this object defined yet?}\tnote{Yes it's just above, we could make it into a definition if you think it's too hidden} is $(\delta'',\nu)$-robust, for some $\delta''(\eps)=O({\delta(2\eps)})$ and $\nu$ the uniform distribution on $\{1,\ldots,n\}\subseteq \mX$. 
\item If $G_{\code,M}$ is $(\delta',\nu)$-robust for $\nu$ the uniform distribution on $\{1,\ldots,n\}\subseteq \mX$, and $(\mC,M)$ is Abelian, then $M$ has quantum soundness $\delta''$, for some $\delta''(\eps) = O(\delta'(\sqrt{\eps}))$. 
\end{enumerate}
\end{proposition}

\begin{proof}
We start with the second assertion. Suppose that $G_{\code,M}$ is $\delta'$-robust, for some function $\delta'$. Let $(\mM,A)$ be be an $r$-local $\eps$-presentation of $\code$, and for each $S\subseteq\{1,\ldots,n\}$ such that $|S|=r$, $\{B^{S,i}\}_{i\in S}$ the commuting family of projective measurements promised by the definition. Then for every $S$ there is a projection $P^S$ in $\mM_\infty$ and an isometry $w^S\in P^S \mM_\infty 1_\mM$ such that 
\[ \Es{i\in S} \sum_a \big\| A^i_a - (w^S)^* B^{S,i}_{a}w^S \big\|_2^2 \,\leq\,\eps_S\;,\]
for some $(\eps_S)$ such that $\Es{S\leftarrow M(\cdot)} \eps_S \leq \eps$. 
 For any $a\in\field^S$ let $\tilde{P}^S_a = \prod_{i\in S} B^{S,i}_{a_i}$. Then $\{\tilde{P}^S_a\}_{a\in\F_q^S}$ is a projective measurement such that
\begin{align}
\Es{S\leftarrow M(\cdot)} \Es{i\in S} \sum_{a\in \F_q^S} \tau\big(  (w^S)^* \tilde{P}^S_a w^S \, A^i_{a_i}\big) 
&= \Es{S\leftarrow M(\cdot)} \Es{i\in S} \sum_{b\in \F_q}\tau\big(  (w^S)^* B^{S,i}_b w^S\, A^i_{b}\big)\notag \\
&= \Es{S\leftarrow M(\cdot)} \Es{i\in S} \frac{1}{2}\Big(\tau\big((w^S)^*w^S\big) + \tau(1) - \sum_{b\in \F_q} \big\|  (w^S)^* B^{S,i}_b w^S - A^i_{b}\big\|_2^2\Big) \notag\\
&\geq \,\frac{1}{2}(1-\eps+1-\eps)\;.\label{eq:s-g-1}
\end{align}
Starting from the $\{\tilde{P}^S_a\}$, let $\{P^S_a\}$ be the family of projective measurements on $\mM$ that is promised by Lemma~\ref{lem:pull-back}. By a simple averaging argument,
\begin{equation}\label{eq:s-g-1b}
\Es{S\leftarrow M(\cdot)}\sum_a \big\| P^S_a - (w^S)^* \tilde{P}^S_a (w^S)\big\|_2^2 \,=\, O(\eps)\;. 
\end{equation}
We can write
\begin{align}
\Big|\Es{S\leftarrow M(\cdot)} \Es{i\in S} \sum_a \tau\big( P^S_a A^i_{a_i}\big)
&- \Es{S\leftarrow M(\cdot)} \Es{i\in S} \sum_a \tau\big( (w^S)^* \tilde{P}^S_a (w^S)A^i_{a_i}\big)\Big|\notag\\
&= \Big| \Es{S\leftarrow M(\cdot)} \Es{i\in S} \sum_a \tau\big(\big((w^S)^* \tilde{P}^S_a (w^S)-P^S_a\big)A^i_{a_i}\big)\Big|\notag\\
&\leq \Es{S\leftarrow M(\cdot)} \Es{i\in S} \sum_a \tau\big(\big|(w^S)^* \tilde{P}^S_a (w^S)-P^S_a\big|\big)\notag\\
&= O(\sqrt{\eps})\;,\label{eq:s-g-2}
\end{align}
where the second line uses $\|A^i_a\|\leq 1$ and the last is by Lemma~\ref{lem:l1-l2} and~\eqref{eq:s-g-1b}.
Together with~\eqref{eq:s-g-1} and using the definition of $G_{\code,M}$, we have shown that
\begin{equation}
 \sum_{(S,i)} \mu(S,i) \sum_{a\in \field^s}\sum_{b\in\field} D(S,i,a,b) \tau\big(  P^S_a  A^i_{b}\big) \,\geq \,1-O(\sqrt{\eps})\;.\label{eq:s-g-4}
\end{equation}
In other words $\omega(G_{\code,M};\strategy)\geq 1-\eps'$ for $\strategy$ the strategy with measurements $\{A^i_b\}$ and $\{P^{S}_a\}$ and some $\eps'=O(\sqrt{\eps})$. Using the definition of robustness we deduce that $(\mM,A,P)$ is $(\delta'(\eps'),\nu)$-close to a perfect strategy $(\mN,C,Q)$ for $G$. Since $\nu$ is uniform on $\{1,\ldots,n\}$ we may specialize to the $A$ measurements; using the definition of closeness there exists a projection $P\in \mM_\infty$ such that $\{C^i\}_a \subseteq P\mM_\infty P$ and $\Es{i}\sum_a \|A^i_a-w^* C^i_a w \|_2^2 \leq 2\delta'$, and moreover $\{C^i_a\}$ satisfy that for any $S$ in the support of $\mu$, $\{C^i_a\}$ pairwise commute for $i\in S$. Using the assumption that $(\code,M)$ is Abelian, all the $\{C^i_a\}$ commute and form a representation of $\code$.

Now we show the first assertion. Suppose that $M$ has quantum soundness $\delta$.
Let $\strategy=(\mM,A,P)$ be a strategy for $G_{\code,M}$ on $(\mM,\tau)$ that succeeds with probability at least $1-\eps$. This can be reformulated as 
\begin{equation}
 \sum_{(S,i)} \mu(S,i) \sum_{a\in \F_q^s}\sum_{b\in\F_q} D(S,i,a,b) \tau\big(  P^S_a  A^i_{b}\big) \,\geq \,1-\eps\;.\label{eq:s-g-3a}
\end{equation}
For each $S$ and $i\in S$, let $B^{S,i}_b = \sum_{a:a_i=b} P^S_a$. Then using the definition of $\mu$ and $D$, rewriting~\eqref{eq:s-g-3a} immediately gives
\begin{equation}
 \Es{S\leftarrow M(\cdot)} \Es{i\in S} \sum_{b\in\F_q} \big\| A^i_{b} - B^{S,i}_b\big\|_2^2 \,\leq \,2\eps\;.\label{eq:s-g-3}
\end{equation}
Thus $(\mM,A)$ is an $\eps'$-local presentation of code, for $\eps'=2\eps$. Using quantum soundness, there is a representation $(\mN,C)$ of $\code$ that is $\delta'$-close to $(\mM,A)$, for $\delta'=\delta(\eps')$. 

For every $S$, let $Q^S_a = \prod_{i\in S} C^i_{a_i}$, which is a projective measurement on $\mN$. It is easy to verify that $(\mN,C,Q)$ is a perfect strategy in $G_{\code,M}$. Using the definition of $\nu$, it follows that   $(\mN,B,Q)$ is $O({\delta'})$-close to $(\mM,A,P)$.

%Moreover,
%\begin{align}
%\Es{S\leftarrow M(\cdot)} \sum_{a\in\field^k} \big\| P^S_a - w^* Q^S_a w\big\|_2^2
%&=2 - 2\Es{S\leftarrow M(\cdot)} \sum_{a\in\field^k} \tau\Big( \prod_i B^{S,i}_{a_i} w^* \prod_i C^{i}_{a_i}w\big)\;.\label{eq:s-g-4}
%\end{align}
%Applying a simple inductive argument, using Lemma~\ref{lem:l1-l2} we get that \tnote{This needs a little more detail}
%\begin{align*}
%\Big|\Es{S\leftarrow M(\cdot)} \sum_{a\in\field^k} \tau\Big( \prod_i B^{S,i}_{a_i} w^* \prod_i C^{i}_{a_i}w\big) - \Es{S\leftarrow M(\cdot)} \sum_{a\in\field^k} \tau\Big( \prod_i B^{S,i}_{a_i}  \prod_i B^{S,i}_{a_i}\big)\Big| &= O\big(r\sqrt{\delta'}\big)\;.
%\end{align*}
%Since the second term on the left-hand side above equals $1$, plugging back into~\eqref{eq:s-g-4} it follows that   $(\mN,B,Q)$ is $O(r\sqrt{\delta'})$-close to $(\mM,A,P)$.
\end{proof}


\subsection{Commutation game}

We use the same game as in~\cite[Section 3.1]{de2022spectral}. For convenience we change the notation slightly and denote $x_{X,0}, x_{Z,0} \in \mX_{com}$ the two special questions, $x_{com,1}$ and $x_{com,2}$ respectively. 

\subsection{Anti-commutation game}

We use the same game as in~\cite[Section 3.2]{de2022spectral}. For convenience we change the notation slightly and  denote $x_{X,1}, x_{Z,1} \in \mX_{anticom}$ the two special questions, $x_{anticom,1}$ and $x_{anticom,2}$ respectively. 

\subsection{Combined game}

Let $\code$ be an $[n,k,d]_2$ linear code and $M$ an $r$-local tester for $\code$. 
We use $G_{\dlS}$ to denote the game introduced in~\cite[Section 3.4]{de2022spectral}, specialized to the following choices. The group is $H=\field^k$. Let
\[ S_X=S_Z=\{G_\code e_i:\,i\in\{1,\ldots,n\}\}\subseteq \field^k\;,\]
and let $\mu_{\dlS}$ be the uniform distribution over $S_X\times S_Z$. Let $\Omega$ be the support of $\mu_{\dlS}$ and $\alpha,\beta$ the coordinate projections. Then $G_{\dlS}=(\mX,\mu_{\dlS},\mA,D)$ has question set $\mX = \{PX,PZ\} \cup (\mX_{com}\times \Omega_+) \cup (\mX_{anticom} \times \Omega_-)$ and is as described in~\cite[Section 3.4]{de2022spectral}. For clarity we recall the game, using our notation, in Figure~\ref{fig:dlS}. 

\begin{figure}[!htbp]
  \centering
  \begin{gamespec}
Let $S_X,S_Z\subseteq \field^k$.  Select $\omega = (\omega_X,\omega_Z)\in S_X \times S_Z $ uniformly at random. Let $\gamma = \omega_X \cdot \omega_Z \in \field$. Execute either of the following tests with probability $1/3$ each. 
    \begin{enumerate}
      \setlength\itemsep{1pt}
    \item (\textbf{Anti-commutation test}) 
		\begin{enumerate}
		\item If $\gamma=0$ then select a pair of questions $(x_c,y_c)$ as in the commutation game. Send $(x_c,\omega)$ to $\alice$ and $(y_c,\omega)$ to Bob. Accept if and only if their answers are accepted in the commutation game. 
		\item If $\gamma\neq 0$ then do the same but for the anti-commutation game. 
		\end{enumerate} 
		 \item (\textbf{$Z$-Consistency test}) Send $Z$ to $\alice$ and $(x_{Z,\gamma},\omega)$ to $\bob$. Receive $a\in \field^k$ and $b\in \field$ respectively. Accept if and only if $a\cdot \omega_Z=b$. 
		 \item (\textbf{$X$-Consistency test}) Send $X$ to $\alice$ and $(x_{X,\gamma},\omega)$ to $\bob$. Receive $a\in \field^k$ and $b\in \field$ respectively. Accept if and only if $a\cdot \omega_X=b$. 
    \end{enumerate}
  \end{gamespec}
  \caption{The game $G_{\dlS}$ checks (anti)commutation relations between two collections of observables.}
  \label{fig:dlS}
\end{figure}

\section{Main result}

In this section we focus on binary codes only, see Section~\ref{sec:pbt} for the case of fields over $\F_q$ for $q$ a power of $2$. Let $\code$ be an $[n,k,d]$ linear code and $M$ an $r$-local tester for $\code$. 


\begin{figure}[!htbp]
  \centering
  \begin{gamespec}
Let $M$ be an $r$-local tester for the $[n,k,d]$ code $\code$.  Execute either of the following tests with probability $1/3$ each. 
    \begin{enumerate}
      \setlength\itemsep{1pt}
    \item (\textbf{Code test}) Select $W\in \{X,Z\}$ uniformly at random, and $\rand$ a setting for $M$'s random tape at random. Let $S=M(\rand)\subseteq\{1,\ldots,n\}$. Sample $i\in S$ uniformly at random. Send $(W,S)$ to $\alice$ and $(W,i)$ to $\bob$. Receive $a\in \field^S$ from $\alice$ and $b\in \field$ from $\bob$. Accept if and only if $M(S,a)=1$ and $a_i = b$.  
    \item (\textbf{Anti-commutation test}) Select $(i_X,i_Z)\in \{1,\ldots,n\}^2 $ uniformly at random. Let $\omega = (G_\code e_{i_X}, G_\code e_{i_Z})$ and $\gamma =  (G_\code e_{i_X}) \cdot(G_\code e_{i_Z}) \in \field$.
		\begin{enumerate}
		\item If $\gamma=0$ then select a pair of questions $(x_c,y_c)$ as in the commutation game. Send $(x_c,\omega)$ to $\alice$ and $(y_c,\omega)$ to Bob. Accept if and only if their answers are accepted in the commutation game. 
		\item If $\gamma\neq 0$ then do the same but for the anti-commutation game. 
		\end{enumerate} 
		 \item (\textbf{Consistency test}) Select $(i_X,i_Z)\in \{1,\ldots,n\}^2 $ and $W\in \{X,Z\}$ uniformly at random. Let $\omega=(G_\code e_{i_X}, G_\code e_{i_Z})$ and $\gamma = (G_\code e_{i_X}) \cdot(G_\code e_{i_Z}) \in \field$. Send $(W,i_W)$ to $\alice$ and $(x_{W,\gamma},\omega)$ to $\bob$. Receive $a\in \field$ and $b\in \field$ respectively. Accept if and only if $a=b$. 
    \end{enumerate}
  \end{gamespec}
  \caption{The braiding test over $\code$ verifies that the players respond consistently with a uniformly random codeword from $\code$, obtained by measuring $k$ shared EPR pairs in the standard or Hadamard basis and encoding the resulting $k$-bit string.}
  \label{fig:braiding-test}
\end{figure}


\begin{definition}
Let $k\in \N$ and $\delta:[0,1]\to\R_+$. 
A $(k,\delta(\eps))$-qubit test is a synchronous game $G=(\mX,\mu,\mA,D)$ such there are two sets $S_X,S_Z\subseteq \field^k$ and an injection $\phi:(\{X\}\times S_Z) \cup (\{Z\}\times S_Z) \to \mX$ such that $\mA(\phi({X},a))=\mA(\phi({Z},b))=\field$ for all $a\in S_X$, $b\in S_Z$ and such that the following holds:
\begin{itemize}
\item (Completeness:) There is a synchronous strategy for $G$ on $\mM=M_{2^{k}}(\C)$ that succeeds with probability $1$ in $G$ and is such that $\widehat{P^{\phi({W},a)}} = \sigma^W(a)$ \hnote{overloading $\tau$}\tnote{Fixed; since the Paulis are binary we can just use $\sigma$ everywhere} for every $W\in\{X,Z\}$ and $a\in S_W$.
\item (Soundness:) Any synchronous strategy in $(\mM,\tau)$ for $G$ that succeeds with probability $1-\eps$ for some $\eps\geq 0$ is $\delta(\eps)$-close to a strategy on an algebra of the form $(M_{2^{k}}(\C)\otimes \mN,\tr\otimes \tau')$ and such that
\[\widehat{P^{\phi({W},a)}} = \sigma^W(a)\otimes \Id_\mN\;.\]
\end{itemize}
\end{definition}

\begin{remark}
The definition allows ``trivial'' settings for the sets $S_X,S_Z$, e.g.\ $S_X=S_Z=\emptyset$. In this case, a strategy for $G$ may not give direct access (through specific questions) to any Pauli observable and the soundness condition is trivial. As soon as e.g. $S_X,S_Z$ both contain the basis elements $e_1,\ldots,e_k \in \field^k$ then the definition is non-trivial. Moreover, the notion of closeness provided by the soundness statement is averaged over the marginal of the game distribution $\mu$. Therefore, the measurements associated with questions in the range of $\phi$ are close to the ideal measurements only inasmuch the corresponding questions have non-trivial probability of being asked in $G$. 
\end{remark}

\begin{theorem}\label{thm:main}
Let $\code$ be an $[n,k,d]$ linear code and $M$ an $r$-local tester for $\code$ that is $\delta(\eps)$-quantum sound and such that $(\code,M)$ is Abelian. Then the braiding test over $\code$ is a $(k,\delta')$-qubit test, where $S_X=S_Z=\{G_\code e_i:\,i\in\{1,\ldots,n\}\}\subseteq \field^k$, $\phi(W,G_\code e_i)=(W,i)$ and $\delta' = O(\delta^{1/2}(6\eps))$.\footnote{Assume $G_\code$ has no repeated columns.} 
\end{theorem}

\begin{proof}
\underline{Completeness:} We first verify completeness. For $W\in\{X,Z\}$ and $i\in\{1,\ldots,n\}$ let $P^{(W,i)}_a = (\tau^W(G_\code e_i))_a$ and $P^{(W,S)}_a = \prod_{i\in S} P^{(W,i)}_{a_i}$. For $(i_X,i_Z)\in \{1,\ldots,n\}^2 $ let $\omega=(G_\code e_{i_X}, G_\code e_{i_Z})$ and $\gamma =(G_\code e_{i_X}) \cdot(G_\code e_{i_Z}) $ we let $P^{x_{W,\gamma},\omega} = P^{(W,i_W)}$. 

These choices already ensure that the strategy succeeds with probability $1$ in the consistency test. We verify that it succeeds in the code test. Let $S\subseteq\{1,\ldots,n\}$ and $h\in \F_2^S$ any valid parity check for $\code$ with support in $S$. Then $\sum_{i\in S} h_i G_\code e_i = 0$, so $\prod_{i\in S}(\tau^W(G_\code e_i))^{h_i}=\Id$. This means that for any $a\in\field^S$ such that $P^{W,S}_a\neq 0$, we have that $h\cdot a =0$, i.e.\ $a$ satisfies the parity check. Using the completeness property of $M$ it follows that $M$ must accept any $a$ in the support of $P^{W,S}$, which shows that the strategy succeeds in the code test with probability $1$. 

Remains the anti-commutation test. For this we observe that 
% To define $P^{(x_c,\omega)}$ and $P^{(x_a,\omega)}$ for $x_c\notin \{x_{Z,0},x_{Z,0}\}$ and $x_a \notin \{x_{X,1},x_{Z,1}\}$ we observe that 
the binary observables 
\[ U=\widehat{ P^{x_{X,\gamma},\omega}} \quad\text{and}\quad V= \widehat{P^{x_{Z,\gamma},\omega} } \]
commute in case $\gamma=0$ and anti-commute in case $\gamma=1$. This is because by definition $U=\tau^X(G_\code i_X)$ and $V=\tau^W(G_\code i_W)$, and the definition of $\gamma$. Hence the pair $(U,V)$ can be completed to a perfect strategy for the commutation game (if $\gamma=0)$ or anti-commutation game (if $\gamma=1)$. This defines the measurements $P^{(x,\omega)}$ for $x\notin \{x_{W,\gamma},W\in\{X,Z\},\gamma\in\{0,1\}\}$. 

\bigskip 

\underline{Soundness:} Next we show soundness. Let $\strategy$ be a synchronous strategy for $G$  in  $(\mM,\tau)$ that succeeds with probability at least $1-\eps$. For $W\in\{X,Z\}$ let $\strategy^W$ be the strategy in $G_{\code,M}$ that is obtained by restricting $\strategy$ to the relevant measurements, i.e.\ the $P^{W,S}$ and $P^{W,i}$.  Then $\strategy^W$ succeeds with probability at least $1-3\eps$ in $G_{\code,M}$. Using the assumption that $M$ is $\delta(\cdot)$-quantum sound and the first item from Proposition~\ref{prop:sound-game} it follows that there is a $\delta_1 = O({\delta(6\eps)})$ such that for each $W\in\{X,Z\}$, $\strategy^W$ is $(\delta_1,\nu)$ close to a perfect strategy $\tilde{\strategy}^W$ on $(\mN^W,\tau^W)$, where $\nu$ is uniform on $\{1,\ldots,n\}$. This strategy has measurement operators $\{ \tilde{P}^{W,S}_a\}$ and $\{\tilde{P}^{W,i}_b\}$ associated with questions of the form $S$ and $i$ in $G_{\code,M}$. Using the definition of $\nu$, these satisfy that 
\begin{equation}\label{eq:main-0}
\Es{i\in\{1,\ldots,n\}}\sum_b \big\|P^{W,i}_b - (w^W)^* \tilde{P}_b^{W,i} w^W \big\|^2_2 \,=\, O(\delta_1)\;.
\end{equation}
\hnote{What is $\tilde{P}^W_i$, where did that come from?}\tnote{Sorry I put the $i$ at the wrong place, fixed}
Furthermore, since $(\code,M)$ is Abelian, there is a  POVM $\{\tilde{P}^W_u\}_{u\in \field^n}$ such that $\sum_{u\in \code}\tilde{P}^W_u = \Id$ and for each $i\in \{1,\ldots,n\}$, $\tilde{P}^{W,i}_b = \sum_{a:a_i=b} \tilde{P}^W_a$.% where  $u_i = G_\code e_i$.
 
Applying Lemma~\ref{lem:pull-back}, we obtain projective measurements $\{Q^W_u\}$ on $\mM$ such that 
\begin{equation}\label{eq:main-1}
\sum_u \big\|Q^W_u - (w^W)^* \tilde{P}_u^W w^W \big\|^2_2 \,=\, O(\delta_1)\;.
\end{equation}
For any $i\in\{1,\ldots,n\}$ let $Q^{W,i}_b = \sum_{a:\,a_i=b}  Q^W_a$. Then by Lemma~\ref{lem:dp},
\begin{align*}
\Es{i}\sum_b \tau\big( Q^{W,i}_b (w^W)^*\tilde{P}^{W,i}_b(w^W) \big)
&\geq \Es{i}\sum_a \tau\big( Q^{W}_a (w^W)^*\tilde{P}^{W}_a(w^W) \big)\\
&\geq 1-O(\delta_1)\;.
\end{align*}
For $W\in\{X,Z\}$ and $b\in \F^k$ let $R^W_b = Q^W_{G_\mC^T b}$, where by definition $G_\mC^T b\in \mC$. 

We now define a strategy $\strategy'$ for the game $G_{\dlS}$. On question $W\in \{X,Z\}$ the projective measurement is $\{R^W_b\}$. On question of the form $(x_c,\omega)$ for $x_c$ a question in the commutation game, the projective measurement is $\{P^{x_c,\omega}\}$, i.e.\ the same projective measurement as used in $\strategy$. Similarly, on a question of the form $(x_{ac},\omega)$ for $x_{ac}$ a question in the anti-commutation game, the projective measurement is $\{P^{x_{ac},\omega}\}$.

To conclude we show that this strategy succeeds in the game $G_{\dlS}$ with probability $1-O(\sqrt{\delta_1})$. Assuming that this has been shown, by~\cite[Corollary 3.9]{de2022spectral} the strategy $\strategy'$ is $O(\sqrt{\delta_1})$-close to a strategy $\strategy''$ on an algebra of the form $(M_{2^{k}}(\C)\otimes \mN,\tr\otimes \tau')$ such that $P^W_b = \tau^W_b\otimes 1_\mN$. By definition of $R$, 
\begin{equation}\label{eq:main-3}
 Q^{W,i}_b \,=\,  \sum_{a:\,a_i=b}  Q^W_a \,=\, \sum_{c: (G_\mC^T c)_i=b}  R^W_c \;,
\end{equation}
hence
\begin{equation*}
 \widehat{Q^{W,i}}\,=\, \sum_c (-1)^{c\cdot (G_\mC e_i)} R^W_c \,=\, \widehat{R^W}(G_\mC e_i)\;.
\end{equation*}
Using the definition of the game distribution, closeness of $\strategy'$ and $\strategy''$ thus implies that
\begin{equation*}
\Es{i\in\{1,\ldots,n\}} \big\|\widehat{Q^{W,i}} - (w'')^* {\sigma^W}(G_\mC e_i) (w'') \big\|_2^2 \,=\,O(\sqrt{\delta_1})\;.
\end{equation*}
Combining with~\eqref{eq:main-0} and~\eqref{eq:main-1}, this shows the theorem. 

It remains to verify that $\strategy'$ succeeds in the game $G_{\dlS}$ with probability $1-O(\sqrt{\delta_1})$. By definition $\strategy'$ succeeds in the (anti)-commutation test with probability $1-O(\eps)$. It remains to check the $W$-consistency test, for $W\in\{X,Z\}$. Because $\strategy$ succeeds with probability $1-O(\eps)$ in the consistency test, 
\begin{equation*}
\Es{i_X,i_Z\in\{1,\ldots,n\}} \sum_b \tau\big( P^{W,i}_b P^{x_{W,\gamma},\omega}_b\big) \,\geq\, 1-O(\eps)\;,\
\end{equation*}
where $\omega$ and $\gamma$ are defined as in Figure~\ref{fig:braiding-test}. Using~\eqref{eq:main-0},~\eqref{eq:main-1}
and Lemma~\ref{lem:close-value} it follows that 
\begin{equation*}
\Es{i_X,i_Z\in\{1,\ldots,n\}} \sum_b \tau\big( Q^{W,i}_b P^{x_{W,\gamma},\omega}_b\big) \,\geq\, 1-O(\sqrt{\delta_1})\;,\
\end{equation*}
Using~\eqref{eq:main-3}, this can be rewritten as 
\begin{equation}\label{eq:main-4}
\Es{i_X,i_Z\in\{1,\ldots,n\}} \sum_{b,c: (G_\mC^T c)_i=b} \tau\big( R^{W}_c P^{x_{W,\gamma},\omega}_b\big) \,\geq\, 1-O(\sqrt{\delta_1})\;.
\end{equation}
Since $\omega_W = G_\mC e_i$, $(G_\mC^T c)_i = c\cdot \omega_W$. Thus~\eqref{eq:main-4} shows that $\strategy'$ succeeds with probability $1-O(\sqrt{\delta_1})$ in the $W$-consistency test, as desired. 
\end{proof}


\section{Application: the Pauli braiding test}
\label{sec:pbt}


For $q=2^t$ and $a\in \field$ we let $\kappa(a)\in\F_2^t$ denote the binary representation of $a$, taken in a fixed but usually left implicit self-dual basis of $\F_2^t$ over $\F_2$. We extend $\kappa$ to vectors over $\field$ coordinate-wise. We let $\tr(\cdot):\F_q\to\F_2$ denote the trace over $\F_2$. Because we chose a self-dual basis for the binary representation, the trace satisfies $\tr(ab)=\kappa(a)\cdot\kappa(b)$. 
	
%	We use $\sigma^W(a)$, for $a\in F_q^k$, to denote the tensor product of $k$ Pauli $W$ observables over $\field$. We use $\{\sigma^W_a\}$ to denote the associated projective measurement. 


%\subsection{Codeword tests over $\F_q$}
%
%We broaden Definition~\ref{def:code-test} to allow tests defined over any finite field $\F_q$.
%
%\begin{definition}[Codeword test over $\F_q$]
%Let $\code$ be an $[n,k,d]_q$ linear code, $\delta:[0,1]\to[0,1]$, and $r\in \N$.
%An $r$-local $\delta$-tester for $\code$ is a distribution $\nu$ over subsets $S\subseteq [n]$ of size at most $r$, and for each $S$ in the support of $\nu$ a predicate $M(S,\cdot):\F_q^S\to\{0,1\}$, such that the following hold:
%\begin{itemize} 
%\item (Completeness:) For any $u\in \code$, $\Pr_{S\sim \nu}( M(S,u_S)=1)=1$.
%\item (Soundness:) For any $\eps\geq 0$ and any $u\in \F_q^n$ that is within (Hamming) distance at least  $\eps$ from $\code$, $\Pr_{S\sim\nu}(M(S,u_{|S})=0)\geq \delta(\eps)$. 
%\end{itemize}
%\end{definition}
%
%The notion of code representation, $\eps$-local presentation and quantum soundness from Section~\ref{sec:q-soundness} are extended to codes over $\F_q$ in the natural way. 

\subsection{Code composition}
\label{sec:code-comp}

Let $q=2^t$ and $\code$ an $[n,k,d]_q$ linear code. Let $\code_{\Had}$ be the Hadamard code over $\F_2^t$ (see Section~\ref{sec:had}). Let $T=2^t$.
Let $\mC'$ be the $[Tn,tk,d']$ linear code over $\F_2$ defined as follows. Given $a\in (\F_2^t)^{k}$, first map $a\mapsto a'=\kappa^{-1}(a) \in \F_q^{k}$. Then encode $a'$ to $b'=\code_\RM(a')\in \F_q^n$. Finally, return $b=\code_\Had(\kappa(b'))\in(\F_2^T)^n$, where $\code_\Had$ is applied component-wise. Using that $\code_\Had$ has relative distance $\frac{1}{2}$, it is easy to verify that this code has distance $d'\geq dT/2$.

Given an $r$-local $\delta$-tester $M$ for $\code$, there is a natural $rq$-local tester $M'$ for $\code'$ which can be described as follows. Index coordinates of $\code'$ by pairs $(i,\alpha)\in [n]\times\F_2^t$, fixing a bijection between $[Tn]$ and $[n]\times \F_2^t$. Then $\nu'$ is the uniform mixtures of two distributions, $\nu'_1$ and $\nu'_2$. To sample from $\nu'_1$, sample $S\sim\nu$ and return the set $S\times\F_2^t$. To sample from $\nu'_2$, select $i\sim[n]$ uniformly at random and $x,y\in\F_2^t$ uniformly at random, and return $\{i\}\times\{x,y,x+y\}$. The decision predicate $M'$ executes $M$ for all sets of the form $S\times \F_2^t$, and the tester for the Hadamard code (Section~\ref{sec:had}) for sets of the form $\{i\}\times\{x,y,x+y\}$.


\begin{proposition}\label{prop:q-to-2}
Suppose that $M$ is an $r$-local tester for $\code$ with quantum soundness $\delta$. Then $M'$ is an $rq$-local tester for $\code'$ with quantum soundness $\delta'$ such that $\delta'(\eps)=\delta(O(\eps))$. 
\end{proposition}

\begin{proof}
Let $\{A^{(i,\alpha)}\}$ and $\{B^{S,(i,\alpha)}\}$ be an $\eps$-local presentation of $\code'$. By definition of the measure $\nu'$, there are $\{\eps_i\}$ such that $\Es{i} \eps_i \leq 2\eps$ and for every $i$, the collections $\{A^{(i,\alpha)}\}$, for $\alpha \in \F_2^t$, and $\{B^{\{i\}\times \{x,y,x+y\},(i,x)}\}$, for $x,y\in\F_2^t$, form an $\eps_i$-local presentation of $\code_\Had$. By quantum soundness of $\code_\Had$ (Theorem~\ref{thm:had-qsound}) $\{i\}\times\{x,y,x+y\}$), for each $i$ there exists commuting $\{\hat{A}^{(i,\alpha)}\}$ that are $O(\eps_i)$-close to the $\{A^{(i,\alpha)}\}$ and moreover are a representation of $\code_{\Had}$. Let $w^{(i)}$ be the implied isometry. For every $a\in \F_2^t$, define $\hat{A}^{i}_{a}=\Es{\alpha}(-1)^{a\cdot \alpha} \hat{A}^{(i,\alpha)}$. Then by linearity this is a projective measurement:
\begin{align*}
\big(\hat{A}^{i}_{a}\big)^2 &=\Big(\Es{\alpha}(-1)^{a\cdot \alpha} \hat{A}^{(i,\alpha)}\Big)^2\\
&= \Es{\alpha,\alpha' }(-1)^{a\cdot (\alpha+\alpha')} \hat{A}^{(i,\alpha)}\hat{A}^{(i,\alpha')}\\
&=\Es{\alpha,\alpha' }(-1)^{a\cdot (\alpha+\alpha')} \hat{A}^{(i,\alpha+\alpha')}\\
&=\hat{A}^{i}_{a}\;,
\end{align*}
where the third line uses that  $\{A^{(i,\alpha)}\}$ are a representation of $\code_{\Had}$. Moreover, $\sum_a \hat{A}^{i}_{a} = \hat{A}^{(i,0)}=\Id$. Hence using Lemma~\ref{lem:pull-back}, for every $i$ we obtain a projective measurement $\{\tilde{A}^{i}_{a}\}$ on $\cM$ such that
\begin{equation}\label{eq:qto2-1a}
 \sum_a \big\| \tilde{A}^i_a - (w^{(i)})^* \hat{A}^i_a (w^{(i)}) \big\|_2^2 \,=\, O(\eps_i)\;.
\end{equation}
%Moreover,
%\begin{equation}\label{eq:qto2-1}
%\Es{i} \Es{\alpha}  \Big\| A^{(i,\alpha)} - \sum_a (-1)^{a\cdot\alpha}(w^{(i)})^* \hat{A}^i_a (w^{(i)})\Big\|^2
%\,=\, \Es{i} \Es{\alpha}  \big\| A^{(i,\alpha)} - (w^{(i)})^* \hat{A}^{i,\alpha} (w^{(i)})\big\|^2 \,=\, O(\eps)\;,
%\end{equation}
%by closeness. Combining~\eqref{eq:qto2-1a} with~\eqref{eq:qto2-1} through the triangle inequality gives
We then get
\begin{align}
 \Es{i} \sum_a \Big\| \Es{\alpha} (-1)^{a\cdot \alpha} A^{(i,\alpha)} -  \tilde{A}^i_a \Big\|_2^2 
&\leq  2\Es{i} \sum_a \Big\| \Es{\alpha} (-1)^{a\cdot \alpha} A^{(i,\alpha)} -  (w^{(i)})^* \hat{A}^i_a (w^{(i)}) \Big\|_2^2 + O(\eps)\notag\\
&=  2\Es{i} \sum_a \Big\| \Es{\alpha} (-1)^{a\cdot \alpha} A^{(i,\alpha)} -   \Es{\alpha} (-1)^{a\cdot \alpha} (w^{(i)})^* \hat{A}^{i,\alpha} (w^{(i)}) \Big\|_2^2+ O(\eps) \notag\\
&= 2\Es{i} \Es{\alpha} \big\| A^{(i,\alpha)} -    (w^{(i)})^* \hat{A}^{i,\alpha} (w^{(i)}) \big\|_2^2+ O(\eps)\notag\\
&= O(\eps)\;, \label{eq:qto2-2}
\end{align}
where the first line uses the triangle inequality and~\eqref{eq:qto2-1a}, the second line uses the definition of $\hat{A}^i_a$, the third line is Parseval's identity and the last is by closeness. 

Now for $b\in \F_q$ define $\tilde{B}^{S,i}_b= \Es{\alpha} (-1)^{b\cdot \alpha} B^{S\times \F_2^t,(i,\alpha)}$, which for the same reasons as earlier is a projective measurement. 
To conclude we show that $\{\tilde{A}^{i}\}$ and $\{\tilde{B}^{S,i}\}$ form an $O(\eps)$-presentation of $\code$. The fact that the $\{\tilde{B}^{S,i}\}$ satisfy the constraints imposed by $M$ is clear, because $\{B^{S,(i,\alpha)}\}$ satisfy those of $M'$. For the closeness condition, we have
\begin{align*}
\Es{S}\Es{i} \sum_b \big\| \tilde{A}^i_b - \tilde{B}^{S,i}_b \big\|_2^2
&\leq 2\Es{S}\Es{i} \sum_b \Big\| \Es{\alpha} (-1)^{b\cdot \alpha} {A}^{(i,\alpha)} - \Es{\alpha} (-1)^{b\cdot \alpha} B^{S\times \F_2^t,(i,\alpha)}\Big\|_2^2 + O(\eps)\\
&= 2\Es{S}\Es{i} \Es{\alpha}\big\|  {A}^{(i,\alpha)} -  B^{S\times \F_2^t,(i,\alpha)}\big\|_2^2 + O(\eps)\\
&\leq 4\eps + O(\eps)\;,
\end{align*}
where the first inequality is by~\eqref{eq:qto2-2} and the triangle inequality, the second line by Parseval's formula and the last is by assumption. Thus quantum soundness of $\code'$ follows from quantum soundness of $\code$. 
\end{proof}






\subsection{The Reed-Muller code over $\F_q$}

Fix integers $m,t \in \N$ and let $q=2^t$ and $M = 2^m$. Let $\mP(q,m,d)$ be the vector space over $\F_q$ that consists of all $m$-variate polynomials $f$ over $\F_q$ of individual degree at most $d$, that is all functions of the form
\[
	f(x_1,\ldots,x_m) = \sum_{\alpha \in \{0,1,\ldots,d\}^m} c_\alpha\,
  x_1^{\alpha_1} \cdots x_m^{\alpha_m}\;,
\]
where $\{c_\alpha\}$ is a collection of coefficients in $\F_q$. It is easy to verify that $\mP(q,m,d)$ has dimension $D = (d+1)^m$ over $\F_q$. It follows that the map $\mC_\RM: (c_\alpha) \mapsto f$ defines a $[q^m,(d+1)^m,D]_q$ linear code over $\F_q$, where $D\geq (1-md/q)q^m$ follows from the Schwartz-Zippel lemma: 

\begin{lemma}[Schwartz-Zippel lemma~\cite{Sch80,Zip79}]
  \label{lem:schwartz-zippel}
  Let $f, g: \F_q^m \to \F_q$ be two unequal polynomials with total degree at most $d$. Then
  \begin{equation*}
    \Pr_{x \sim \F_q^m}\big(f(x) = g(x)\big) \leq \frac{d}{q}\;.
  \end{equation*}
\end{lemma}


We define a tester $M_{\RM}$ for the code $\mC_\RM$ over $\F_q$, see Figure~\ref{fig:RM-tester}. The second test applied by the tester, the subcube commutation test, may seem superfluous, because it always accepts. However, the test is important to show that the code is robust. Note that including the test imposes a non-trivial constraint of pairwise approximate commutation on representations of $\mC_\RM$, and hence also on $\eps$-local presentations. In particular, due to the presence of this test the pair $(\code_\RM,M_\RM)$ is trivially Abelian. 


\begin{figure}[!htbp]
  \centering
  \begin{gamespec}
Perform one of the following tests with probability~$\tfrac{1}{2}$ each. 
\begin{enumerate}
	\item \textbf{Axis-parallel lines test:}
		Let $u \sim \F_q^m$ be a uniformly random point, $j\sim \{1,\ldots,m\}$ chosen uniformly at random,
		and let $\ell = \{ (u_1,\ldots,u_{j-1},s,u_{j+1},\ldots,u_m) \in \F_q^m : s \in \F_q \}$
		be the axis-parallel line passing through $u$ in the $j$-th direction. Read the entries indexed by $\ell$ and accept if and only if they match a degree-$d$ polynomial. 
	\item \textbf{Subcube commutation test:}
	Select $j \sim \{1,\ldots,m\}$ uniformly at random, and select $x_{m-j+2},\ldots,x_{m} \sim \F_q$ uniformly at random. Select $u,v$ independently and uniformly at random from $\F_q^m$, conditioned on the last $(j-1)$ coordinates of both points being $x_{m-j+2},\ldots,x_m$. Read the entries indexed by $u$ and $v$ and accept. 	
    \end{enumerate}
  \end{gamespec}
  \caption{A local test for $\code_{\RM}$}
  \label{fig:RM-tester}
\end{figure}



\begin{theorem}\label{thm:mrm-sound}
$M_\RM$ has quantum soundness $\delta(\eps)=\poly(m,d)\cdot \poly(\eps,n^{-1})$.
\end{theorem}

\begin{proof}
In~\cite{ji2022quantum} it is shown that the game $G_{\code_\RM,M_\RM}$ (played using $\F_q$ as the base field) is $(\delta,\nu)$-robust, where $\nu$ is the uniform distribution over $\F_q^m \subseteq \mX$ and $\delta$ satisfies $\delta(\eps)=\poly(m,d)\cdot \poly(\eps,n^{-1})$. The theorem follows by the second item of Proposition~\ref{prop:sound-game}. 
\end{proof}


\subsection{The Pauli braiding test}

By applying Proposition~\ref{prop:q-to-2} to the tester $M_\RM$, which is quantum sound by Theorem~\ref{thm:mrm-sound}, we deduce that $M'_\RM$, defined from $M_\RM$ as in Section~\ref{sec:code-comp}, is an $O(\delta)$-sound $rq$-local tester for the binary code $\code'_{\RM}$, where $\delta$ is as in Theorem~\ref{thm:mrm-sound}. This allows us to apply Theorem~\ref{thm:main} to obtain a concrete instantiation of the braiding test. We call it the Pauli braiding test. For completeness, we give a description of the test in its entirety in Figure~\ref{fig:pauli-braiding}. To formulate the test, we make explicit the sets $S_X$ and $S_Y$. For the rows of the generating matrix $G_{\code_\RM}$ we take an arbitrary basis of all individual degree-$d$ polynomials, e.g. for $\alpha \in \{0,\ldots,d\}^m$, 
\[ \ind_{\alpha}(x) = \prod_{i} x_i^{\alpha_i} (1-x_i)^{d-\alpha_i}\;.\]
Then the $x$-th column of $G_{\code_\RM}$, $G_{\code_{\RM}}e_x$ for $x\in \F_q^m$, has entries $\ind(x)=(\ind_{\alpha}(x))_{\alpha\in\{0,\ldots,d\}^m}$.

\begin{figure}[!htbp]
  \centering
  \begin{gamespec}
Let $S_X,S_Z\subseteq \field^k$.  Select $u_X,u_Z \in \F_q^m$ and $r_X,r_Z\in \F_q$ uniformly at random. Let $\omega_X = \ind(u_X)$, $\omega_Z=\ind(u_Z)$, and  $\omega = (\omega_X,\omega_Z,r_X,r_Z)\in (\F_q^{(d+1)^m})^2 \times (\F_q)^2$. Let 
$\gamma = \tr((r_X\omega_X) \cdot (r_Z\omega_Z)) \in \F_2$.
 Execute either of the following tests with probability $1/3$ each. 
\begin{enumerate}
      \setlength\itemsep{1pt}
    \item (\textbf{Code test}) Select $W\in \{X,Z\}$ uniformly at random, and $(S,T)\subseteq \F_q^{(d+1)^m} \times \F_2^t$ from $\nu'_\RM$. Sample $(i,\alpha)\in (S,T)$ uniformly at random. Send $(W,(S,T))$ to $\alice$ and $(W,(i,\alpha))$ to $\bob$. Receive $a\in \F_2^{S\times T}$ from $\alice$ and $b\in \F_2$ from $\bob$. Accept if and only if $M'_\RM((S,T),a)=1$ and $a_{(i,\alpha)} = b$.  
    \item (\textbf{Anti-commutation test}) 
		\begin{enumerate}
		\item If $\gamma=0$ then select a pair of questions $(x_c,y_c)$ as in the commutation game. Send $(x_c,\omega)$ to $\alice$ and $(y_c,\omega)$ to Bob. Accept if and only if their answers are accepted in the commutation game. 
		\item If $\gamma\neq 0$ then do the same but for the anti-commutation game. 
		\end{enumerate} 
		 \item (\textbf{Consistency test}) Select $W\in \{X,Z\}$ uniformly at random. Send $(W,(\omega_W,r_W))$ to $\alice$ and $(x_{W,\gamma},\omega)$ to $\bob$. Receive $a\in \F_2$ and $b\in \F_2$ respectively. Accept if and only if $a=b$. 
    \end{enumerate}
  \end{gamespec}
  \caption{The Pauli braiding test.}
  \label{fig:pauli-braiding}
\end{figure}




\begin{theorem}
The braiding test is a $(t(d+1)^m,\delta')$-qubit test, for some $\delta'(\eps)=\poly(m,d)\cdot \poly(\eps,n^{-1})$.
\end{theorem}

\begin{proof}
Follows from Theorem~\ref{thm:main} and Theorem~\ref{thm:mrm-sound}.
\end{proof}

As a corollary we obtain quantum soundness for the variant of the Pauli braiding test which is described in~\cite[Section 7.3]{ji2020mip}. This is because any (synchronous) strategy for the latter can be mapped into a strategy for the former. The only difference is that in the variant from~\cite{ji2020mip}, some of the answers, specifically the ones from the Code test associated with $\code_\RM$, are specified over $\F_q$. Of course this is equivalent to returning the binary representation of all $\F_q$ elements, which is what is required in the Code test here, for the cases where the set $T=\F_q$. \tnote{This part is loose; I didn't check things formally and I'm not sure we want to either}


\bibliography{qld}

\notesendofpaper

\end{document}
