\documentclass[11pt]{article}
\usepackage{booktabs}
\usepackage{fullpage}
\usepackage{titlesec}
%\newcommand{\sectionbreak}{\clearpage}
\usepackage{amsmath,amsfonts,amsthm,mathrsfs,xspace,graphicx}
\usepackage[backref,colorlinks,citecolor=blue,bookmarks=true]{hyperref}
\usepackage{mathpazo}
\usepackage{endnotes}
\usepackage{color}
\usepackage{float}
\usepackage{xcolor}
\usepackage{mdframed}
\usepackage{bbm}
\usepackage{suffix} % for *-version commands
\usepackage{times}
\usepackage{tabularx}
\usepackage{makecell}
\usepackage{amssymb,latexsym}
%\usepackage{IEEEtrantools}
\usepackage[capitalize]{cleveref}
\usepackage{enumitem}
\usepackage{tikz}
\usepackage{tikz-cd}
\usepackage{multirow}
\usepackage[section]{placeins}
\usepackage[affil-it]{authblk}


\mdfdefinestyle{figstyle}{ %
  linecolor=black!7, %
  backgroundcolor=black!7, %
  innertopmargin=10pt, %
  innerleftmargin=25pt, %
  innerrightmargin=25pt, %
  innerbottommargin=10pt %
}

\newtheorem{theorem}{Theorem}[section]
\newtheorem{proposition}[theorem]{Proposition}
\newtheorem{conjecture}[theorem]{Conjecture}
\newtheorem{lemma}[theorem]{Lemma}
\newtheorem{claim}[theorem]{Claim}
\newtheorem{fact}[theorem]{Fact}
\newtheorem{corollary}[theorem]{Corollary}

\newtheorem{remark}[theorem]{Remark}

\theoremstyle{definition}
\newtheorem{definition}[theorem]{Definition}
\newtheorem{example}[theorem]{Example}

\newcommand{\beq}{\begin{eqnarray}}
\newcommand{\eeq}{\end{eqnarray}}

\newcommand{\code}{\mathcal{C}}
\newcommand{\strategy}{\mathscr{S}}
\newcommand{\algebra}{\mathscr{A}}

\newcommand{\ket}[1]{|#1\rangle}
\newcommand{\bra}[1]{\langle#1|}
\newcommand{\ketbra}[2]{\ket{#1}\!\bra{#2}}
\newcommand{\ip}[2]{\langle #1 \! | #2 \rangle}
\newcommand{\proj}[1]{\ket{#1}\!\bra{#1}}
\newcommand{\Tr}{\mbox{\rm Tr}}
\newcommand{\Id}{\ensuremath{I}}
\DeclareMathOperator*{\Expectation}{\mathbb{E}}
\newcommand{\Es}[1]{\Expectation_{#1}}

\newcommand{\reg}[1]{{\textsf{#1}}}
\newcommand{\ol}[1]{\overline{#1}}

\newcommand{\C}{\ensuremath{\mathbb{C}}}
\newcommand{\N}{\ensuremath{\mathbb{N}}}
\newcommand{\bbN}{\ensuremath{\mathbb{N}}}
\newcommand{\complex}{\ensuremath{\mathbb{C}}}
\newcommand{\real}{\ensuremath{\mathbb{R}}}
%\newcommand{\natural}{\ensuremath{\mathbb{N}}}

\newcommand{\bij}{\pi}
\newcommand{\qp}{\tau}

\newcommand{\F}{\ensuremath{\mathbb{F}}}
\newcommand{\M}{\ensuremath{\mathbb{M}}}
\newcommand{\ot}{\otimes}
\newcommand{\Fp}{\F_p}
\newcommand{\Fq}{\F_q}
\newcommand{\BH}{\textsc{BH}}
\newcommand{\ld}{\textsc{ld}}
\newcommand{\downsize}{\kappa}
\newcommand{\tobin}{\flat}
\newcommand{\downsizem}{\chi}

\newcommand{\K}{\ensuremath{\mathbb{K}}}
\newcommand{\R}{\ensuremath{\mathbb{R}}}
\newcommand{\Z}{\ensuremath{\mathbb{Z}}}

\newcommand{\mA}{\ensuremath{\mathcal{A}}}
\newcommand{\mB}{\ensuremath{\mathcal{B}}}
\newcommand{\mC}{\ensuremath{\mathcal{C}}}
\newcommand{\mE}{\ensuremath{\mathcal{E}}}
\newcommand{\mD}{\ensuremath{\mathcal{D}}}
\newcommand{\mF}{\ensuremath{\mathcal{F}}}
\newcommand{\mG}{\ensuremath{\mathcal{G}}}
\newcommand{\mH}{\ensuremath{\mathcal{H}}}
\newcommand{\mK}{\ensuremath{\mathcal{K}}}
\newcommand{\mM}{\ensuremath{\mathcal{M}}}
\newcommand{\mI}{\ensuremath{\mathcal{I}}}
\newcommand{\cM}{\ensuremath{\mathcal{M}}}
\newcommand{\mQ}{\ensuremath{\mathcal{Q}}}
\newcommand{\mR}{\ensuremath{\mathcal{R}}}
\newcommand{\mS}{\ensuremath{\mathcal{S}}}
\newcommand{\mT}{\ensuremath{\mathcal{T}}}
\newcommand{\mX}{\ensuremath{\mathcal{X}}}
\newcommand{\mY}{\ensuremath{\mathcal{Y}}}

\newcommand{\Inv}{\ensuremath{\textsc{Inv}}}
\newcommand{\GEN}{\ensuremath{\textsc{GEN}}}
\newcommand{\SAMP}{\ensuremath{\textsc{SAMP}}}
\newcommand{\epr}{\ensuremath{\textsc{epr}}}

\newcommand{\Alg}{\mathcal{A}}
\newcommand{\ind}{\ensuremath{\mathrm{ind}}}


\newcommand{\setft}[1]{\mathrm{#1}}
\newcommand{\Density}{\setft{D}}
\newcommand{\Pos}{\setft{Pos}}
\newcommand{\Proj}{\setft{Proj}}
\newcommand{\Channel}{\setft{C}}
\newcommand{\Unitary}{\setft{U}}
\newcommand{\Herm}{\setft{Herm}}
\newcommand{\Lin}{\setft{L}}
\newcommand{\Trans}{\setft{T}}
\DeclareMathOperator{\poly}{poly}
\DeclareMathOperator{\negl}{negl}
\newcommand{\dset}{G}

\newcommand{\val}{\ensuremath{\mathrm{val}}}
\newcommand{\valco}{\ensuremath{\mathrm{val}^{\mathrm{co}}}}
\newcommand{\ia}{\Id_\alice}
\newcommand{\ib}{\Id_\bob}

\newcommand{\desc}[1]{\overline{\cal{#1}}}
\newcommand{\supp}{\textsc{Supp}}
\newcommand{\Gen}{\textsc{Gen}}
\newcommand{\Enc}{\textsc{Enc}}
\newcommand{\Dec}{\textsc{Dec}}

\newcommand{\GenTrap}{\textsc{GenTrap}}
\newcommand{\Invert}{\textsc{Invert}}
\newcommand{\lossy}{\textsc{lossy}}

\newcommand{\rand}{\textrm{rand}}
\newcommand{\had}{\textsc{Had}}


\newcommand{\eps}{\varepsilon}
\newcommand{\ph}{\ensuremath{\varphi}}


\newcommand{\ac}{\textsc{ac}}
\newcommand{\GX}{\textsc{Gap-Maxcut}}
\newcommand{\GNI}{\textsc{Graph Non-Isomorphism}}


\newcommand{\Acc}{\textsc{Acc}}
\newcommand{\Samp}{\textsc{Samp}}
\newcommand{\Ext}{\ensuremath{\text{Ext}}}

\newcommand{\BD}{\mathbb{QB}}
\newcommand{\DD}{\mathbb{D}}
\newcommand{\DDb}{\mathbb{D'}}
\newcommand{\Pot}{\Phi}
\newcommand{\inj}{J}
\newcommand{\mZ}{\mathcal{Z}}
\newcommand{\mN}{\mathcal{N}}
\newcommand{\vs}{\vspace{2mm}~\newline\noindent}
\newcommand{\vb}{\vspace{3mm}\noindent}
\newcommand{\sX}{\mathcal{X}}
\newcommand{\sA}{\mathcal{A}}
\newcommand{\sB}{\mathcal{B}}
\newcommand{\sY}{\mathcal{Y}}
\newcommand{\sR}{\mathcal{R}}


\newcommand{\trnq}[1]{\left[ {#1} \right]_q}

\DeclareMathOperator{\polylog}{polylog}
\newcommand{\mx}[1]{\mathbf{{#1}}}
\newcommand{\vc}[1]{\mathbf{{#1}}}
\newcommand{\abs}[1]{\left\vert {#1} \right\vert}
\newcommand{\norm}[1]{\left\| {#1} \right\|}
\newcommand{\for}{\text{for }}

\DeclareMathOperator{\arcsinh}{arcsinh}
\DeclareMathOperator{\tr}{tr}

\newcommand{\E}{\mathop{\mathbb{E}}\displaylimits} % Expectation

\newcommand{\unif}{\mathcal{U}}
\newcommand{\pt}{\textrm{pt}}
\newcommand{\sample}{\textrm{sample}}
\newcommand{\test}{\textrm{test}}
\newcommand{\free}{\mathcal{F}}
\newcommand{\plane}{\mathcal{P}}
\newcommand{\lines}{\mathcal{L}}
\newcommand{\clines}{\mathcal{CL}}
\newcommand{\pl}{\mathbf{p}}
\newcommand{\individual}{\textrm{individual}}
\newcommand{\blocks}{\textrm{blocks}}
\newcommand{\liness}{\textrm{lines}}
\newcommand{\lp}{\mathcal{LP}}
\newcommand{\Pl}{\ensuremath{\mathrm{Pl}}}
\newcommand{\Ln}{\ensuremath{\mathrm{Lines}}}
\newcommand{\mode}{\mathfrak{m}}
\newcommand{\ECC}{\ensuremath{\textsc{ECC}}}
\newcommand{\EC}{\ensuremath{\textsc{EC}}}
\newcommand{\ENC}{\ensuremath{\textsc{ENC}}}
\newcommand{\cktval}{\ensuremath{\textsc{CKTVAL}}}


\newcommand{\GL}{\mathrm{GL}}
\newcommand{\Matrix}{\mathrm{M}}
\newcommand{\End}{\mathrm{End}}
\newcommand{\Aut}{\mathrm{Aut}}

\newcommand{\game}{\mathfrak{G}}
\newcommand{\sampler}{\mathcal{S}}
\newcommand{\decider}{\mathcal{D}}
\newcommand{\verifier}{\mathcal{V}}


\newcommand{\type}{\mathcal{T}}
\newcommand{\lt}{\mathcal{L}}
\newcommand{\rt}{\mathcal{R}}
\newcommand{\checker}{\mathcal{C}}


\newcommand{\gamestyle}[1]{\ensuremath{\textsc{#1}}\xspace}
\newcommand{\qld}{\gamestyle{QLD}}
\newcommand{\ms}{\gamestyle{MS}}
\newcommand{\pauli}{\gamestyle{Pauli}}
%\newcommand{\bp}{\gamestyle{BP}}
\newcommand{\ora}{\gamestyle{Orac}}
\newcommand{\pcp}{\gamestyle{PCP}}
\newcommand{\ar}{\gamestyle{AR}}
\newcommand{\intro}{\gamestyle{Intro}}

\newcommand{\labelstyle}[1]{\ensuremath{\textsc{#1}}\xspace}
\newcommand{\EPR}{\labelstyle{EPR}}
\newcommand{\aux}{\labelstyle{aux}}
\newcommand{\ancilla}{\labelstyle{anc}}
\newcommand{\msc}{\labelstyle{MC}}
\newcommand{\msv}{\labelstyle{MV}}
\newcommand{\vertex}[1]{\labelstyle{V#1}}
\newcommand{\edge}[1]{\labelstyle{N#1}}
\newcommand{\basis}{\labelstyle{W}}
\newcommand{\xpt}{\labelstyle{X}}
\newcommand{\zpt}{\labelstyle{Z}}
\newcommand{\rxpt}{\labelstyle{R}_\xpt}
\newcommand{\rzpt}{\labelstyle{R}_\zpt}
\newcommand{\dir}[1]{\labelstyle{V#1}}
\newcommand{\coord}{\labelstyle{I}}
\newcommand{\intercept}{\labelstyle{U}}
\newcommand{\plf}{\labelstyle{Pl}}
\newcommand{\lnf}{\labelstyle{Ln}}
\newcommand{\ptf}{\labelstyle{Pt}}
\newcommand{\full}{\labelstyle{full}}
\newcommand{\opt}{\labelstyle{opt}}
\newcommand{\partition}{\mathcal{B}}

\newcommand{\tvarstyle}[1]{\mathsf{#1}}
\newcommand{\tvar}{\ensuremath{\tvarstyle{t}}}
\newcommand{\lvar}{\ensuremath{\tvarstyle{u}}}
\newcommand{\rvar}{\ensuremath{\tvarstyle{v}}}
\newcommand{\pvar}{\ensuremath{\tvarstyle{p}}}
\newcommand{\ovar}{\ensuremath{\tvarstyle{o}}}
\newcommand{\trole}{\ensuremath{v}} % used in intro types

\newcommand{\types}{\labelstyle{T}}

\newcommand{\decode}{\labelstyle{Decode}}

%\newcommand{\alice}{\labelstyle{Alice}}
%\newcommand{\bob}{\labelstyle{Bob}}
\newcommand{\alice}{\labelstyle{A}}
\newcommand{\bob}{\labelstyle{B}}

\newcommand{\oracle}{\labelstyle{Oracle}}
\newcommand{\ab}{\{\alice, \bob\}}

\newcommand{\typestyle}[1]{\ensuremath{\textsc{#1}}\xspace}
\newcommand{\Type}{\typestyle{Type}}
\newcommand{\Plane}{\typestyle{Plane}}
\renewcommand{\line}{\mathbf{\ell}}
\newcommand{\Llane}{\typestyle{Line}}
\newcommand{\Point}{\typestyle{Point}}
\newcommand{\HPoint}{\typestyle{HPoint}}
\newcommand{\Line}{\typestyle{Line}}
\newcommand{\ALine}{\typestyle{ALine}}
\newcommand{\DLine}{\typestyle{DLine}}
\newcommand{\Pair}{\typestyle{Pair}}
\newcommand{\Constraint}{\typestyle{Constraint}}
\newcommand{\Variable}{\typestyle{Variable}}
\newcommand{\Pauli}{\typestyle{Pauli}}
\newcommand{\Sample}{\typestyle{Sample}}
\newcommand{\Read}{\typestyle{Read}}
\newcommand{\MeasureX}{\typestyle{MeasureX}}
\newcommand{\Hide}[1]{\typestyle{Hide}_{#1}}
\newcommand{\HideX}[1]{\typestyle{HideX}_{#1}}
\newcommand{\Target}[1]{\typestyle{Target}_{#1}}
\newcommand{\Oracle}{\typestyle{Oracle}}
\newcommand{\Introspect}{\typestyle{Intro}}
\newcommand{\Intro}{\typestyle{Intro}}
\newcommand{\Simple}{\typestyle{Simple}}
\newcommand{\Eval}{\typestyle{Eval}}
\newcommand{\Agg}{\typestyle{Agg}}
\newcommand{\Input}{\typestyle{Input}}
\newcommand{\Skip}{\typestyle{Skip}}
\newcommand{\Alice}{\typestyle{Alice}}
\newcommand{\Bob}{\typestyle{Bob}}
\newcommand{\Edge}{\typestyle{Alice}}
\newcommand{\Vertex}{\typestyle{Bob}}
\newcommand{\Anchor}{\typestyle{Anchor}}
\renewcommand{\Game}{\typestyle{Game}}
\newcommand{\AB}{\{\alice, \bob\}}
\newcommand{\ctrl}{\labelstyle{c}}
\newcommand{\target}{\labelstyle{t}}

\newcommand{\abc}[1][\delta]{\otimes I_\bob \simeq_{#1} I_\alice \otimes}

\newcommand{\ldc}{k} % number of copies of classical ld tests

\newcommand{\class}[1]{\ensuremath{\mathsf{#1}}\xspace}
\newcommand{\NP}{\class{NP}} %
\newcommand{\IP}{\class{IP}} %
\newcommand{\EXP}{\class{EXP}} %
\newcommand{\NEXP}{\class{NEXP}} %
\newcommand{\QMA}{\class{QMA}} %
\newcommand{\QMIP}{\class{QMIP}} %
\WithSuffix\newcommand\QMIP*{\ensuremath{\class{QMIP}^*}} %
\newcommand{\PSPACE}{\class{PSPACE}} %
\newcommand{\PCP}{\class{PCP}} %
\newcommand{\MIP}{\class{MIP}} %
\newcommand{\MIPco}{\class{MIP}^{\mathrm{co}}} %
\newcommand{\RE}{\class{RE}} %
\newcommand{\coRE}{\class{coRE}}
\newcommand{\NEEXP}{\class{NEEXP}} %
\newcommand{\NEEEXP}{\class{NEEEXP}}
\WithSuffix\newcommand\MIP*{\ensuremath{\class{MIP}^*}} %
\newcommand{\QIP}{\class{QIP}} %


\newcommand{\Ent}{\mathscr{E}}
\newcommand{\compr}{\textsc{Compr}}
\newcommand{\halt}{\textsc{Halt}}
\newcommand{\machine}{\cal{M}}
\renewcommand{\cal}[1]{\mathcal{#1}}
\newcommand{\Kleene}{\cal{K}}
\newcommand{\qldparams}{\mathsf{qldparams}}
\mathchardef\mhyphen="2D
\newcommand{\Fqldparams}{\F_2\mhyphen\mathsf{qldparams}}
\newcommand{\introparams}{\mathsf{introparams}}
\newcommand{\ldparams}{\mathsf{ldparams}}
\newcommand{\tmldparams}{\mathsf{tmldparams}}
\newcommand{\pcpparams}{\mathsf{pcpparams}}

\newcommand{\TMtoSAT}{\mathrm{TMtoSAT}}
\newcommand{\TMtoLD}{\mathrm{TMtoLD}}
\newcommand{\BoundedHalting}{\mathrm{BH}}
\newcommand{\timecomplexity}{\mathsf{TIME}}
\newcommand{\TIME}{\mathsf{TIME}}
\newcommand{\answer}{\mathsf{ANS}}
\newcommand{\MS}{\mathrm{MS}}

\newcommand{\accept}{\typestyle{Accept}}
\newcommand{\reject}{\typestyle{Reject}}

\newcommand{\anch}{\gamestyle{Anch}}
\newcommand{\ans}{\gamestyle{ANS}}
%%%%%%%self testing macros%%%%%%%%%%

\newcommand{\local}{\mathrm{local}}
%\newcommand{\aux}{\mathrm{aux}}


\newcommand{\G}{\mG}
\newcommand{\XZ}{\mathcal{B}}
\newcommand{\hilb}{\mathcal{H}}


%\newcommand{\tmstyle}[1]{\ensuremath{\textsf{#1}}}
\newcommand{\tmstyle}[1]{\ensuremath{\mathsf{#1}}}
\newcommand{\Compress}{\tmstyle{Compress}}
\newcommand{\ComputeRepetitions}{\tmstyle{ComputeRepetitions}}
\newcommand{\ComputeSampler}{\tmstyle{ComputeSampler}}
\newcommand{\RawIntroSampler}{\tmstyle{RawIntroSampler}}
\newcommand{\ComputeIntroSampler}{\tmstyle{IntroSampler}}
\newcommand{\RawIntroDecider}{\tmstyle{RawIntroDecider}}
\newcommand{\ComputeIntroDecider}{\tmstyle{IntroDecider}}
\newcommand{\ComputeIntroVerifier}{\tmstyle{IntroVerifier}}
\newcommand{\ComputeOracleVerifier}{\tmstyle{OracleVerifier}}
\newcommand{\ComputeAnsVerifier}{\tmstyle{AnsRedVerifier}}
\newcommand{\ComputeParrepVerifier}{\tmstyle{RepeatedVerifier}}
\newcommand{\ComputePCPVerifier}{\tmstyle{PCPVerifier}}
\newcommand{\ComputeFixedPoint}{\tmstyle{ComputeFixedPoint}}
\newcommand{\detype}{\tmstyle{Detype}}

\newenvironment{gamespec}{
  \begin{mdframed}[style=figstyle]}{
  \end{mdframed}}

\newcommand{\zero}{\mathrm{zero}}

%%%%%%%From NW19:%%%%%%%%%%
\newcommand{\polymeas}[3]{\mathrm{PolyMeas}(#1,#2,#3)}
\newcommand{\simulpolymeas}[4]{\mathrm{PolyMeas}(#1,#2,#3, #4)}

\newcommand{\eval}{\mathrm{eval}}

%\newcommand{\coin}{o}
\newcommand{\succinctdecider}{\ensuremath{\mathsf{SuccinctDecider}}}
\newcommand{\circuit}{\mathcal{C}}
\newcommand{\formula}{\mathcal{F}}
\newcommand{\bin}{\mathrm{binary}}
\newcommand{\pcpeval}{\Xi}
\newcommand{\pcpverifier}{\mathcal{M}_\ar}
\newcommand{\qlen}{Q}
\DeclareMathOperator{\ev}{eval}

\newcommand{\coded}{\mathrm{Dec}}
\newcommand{\hx}{\hat{x}}
\newcommand{\hz}{\hat{z}}
\newcommand{\htvar}{\hat{\tvar}}
\newcommand{\soundness}{\mathrm{sound}}

\newcommand{\rep}{\gamestyle{Rep}}
\newcommand{\sep}{\gamestyle{Sep}}

\newcommand{\binary}[1]{\mathrm{binary}_{#1}}
\newcommand{\num}[1]{\mathrm{number}_{#1}}
\newcommand{\canbasis}[1]{\mathrm{basis}(#1)}
\newcommand{\canH}[3]{H_{\mathrm{canon}, #1, #2, #3}}
\newcommand{\canlilh}[3]{h_{\mathrm{canon}, #1, #2, #3}}
\newcommand{\canin}[3]{\pi_{\mathrm{canon},#1,#2,#3}}
\newcommand{\canenc}[4]{g_{\mathrm{canon},#1,#2,#3,#4}}


% \usepackage{showlabels}
% \renewcommand{\showlabelfont}{\tiny\ttfamily\color{red}}

\bibliographystyle{alpha}

\newif\ifnotes\notestrue
%\newif\ifnotes\notesfalse


% MARGIN NOTES

\ifnotes
\usepackage{color}
\definecolor{mygrey}{gray}{0.50}
\newcommand{\notename}[2]{{\textcolor{mygrey}{\footnotesize{\bf (#1:} {#2}{\bf ) }}}}
\newcommand{\noteswarning}{{\begin{center} {\Large WARNING: NOTES ON}\endnote{Warning: notes on}\end{center}}}
\newcommand{\notesendofpaper}{{\theendnotes}}

\newcommand{\pnote}[1]{\textcolor{blue}{\small {\textbf{(MLN:} #1\textbf{)
      }}}}
\newcommand{\tnote}[1]{\textcolor{magenta}{\small {\textbf{(Thomas:} #1\textbf{)
      }}}}
\newcommand{\mnote}[1]{\textcolor{red}{\small {\textbf{(Michael:} #1\textbf{) }}}}
\newcommand{\hnote}[1]{\textcolor{olive}{\small {\textbf{(Henry:} #1\textbf{) }}}}
\newcommand{\ftnote}[1]{\footnote{\textcolor{magenta}{\small {\textbf{(Thomas:} #1\textbf{) }}}}}
\newcommand{\tdnote}[1]{\textcolor{blue}{\small {\textbf{(TODO:} #1\textbf{) }}}}

\else

\newcommand{\notename}[2]{{}}
\newcommand{\noteswarning}{{}}
\newcommand{\notesendofpaper}{}
\newcommand{\pnote}[1]{}

\newcommand{\tnote}[1]{}
\newcommand{\jnote}[1]{}
\newcommand{\anote}[1]{}
\newcommand{\znote}[1]{}
\newcommand{\hnote}[1]{}
%\newcommand{\ftnote}[1]{\footnote{\textcolor{magenta}{\small {\textbf{(Thomas:} #1\textbf{) }}}}}
%\newcommand{\tdnote}[1]{\textcolor{blue}{\small {\textbf{(TODO:} #1\textbf{) }}}}

\fi


\begin{document}

\title{Qubit tests from classical codes}

\author{Thoams Vidick}
\date{\today}
\maketitle

\noteswarning


\begin{abstract}

\end{abstract}

	\section{Preliminaries}
	
	For $q=2^t$ and $a\in \F_q$ we let $\kappa(a)$ denote the binary representation of $a$, where the representation is taken in a self-dual normal basis of $\F_q$ over $\F_2$. We extend $\kappa$ to vectors over $\F_q$ coordinate-wise.
	
	We use $\tau^W(a)$, $a\in F_q^k$, to denote the tensor product of $k$ Pauli $W$ observables over $\F_q$. We use $\tau^W_a$ to denote the associated projective measurement. 
	
	Given a projective measurement $\{P_a\}_{a\in \F_q^k}$ and $b\in \F_q^k$ we let $\widehat{P}(b) = \sum_a \omega^{\tr(a\cdot b)} P_a$ be the associated observable. With this notation, $\widehat{\tau^W}(b)=\tau^W(\kappa(b))$.
	
	\subsection{Tracial von Neumann algebras}
	
	We write $\|x\|_2^2=\tau(x^*x)$ for the $2$-norm on $\mM$.
	
\section{Quantum soundness of linear codes}

\begin{definition}[Code]
A $[n,k,d]_q$ linear code is a $k$-dimensional subspace $\code\subseteq \F_q^n$ such that for every $c\in \code$ such that $c\neq 0$, $|c|\leq d$. Equivalently, we think of $\code$ as a linear map $\code:\F_q^k\to \F_q^n$. 
\end{definition}

\begin{definition}
Let $\code$ be an $[n,k,d]_q$ linear code. Define $\ind:\{1,\ldots,n\}\to \F_q^k$ such that for every $a\in \F_q^k$ and $i\in\{1,\ldots,n\}$, $a\cdot \ind(i) = \code(a)_i$.
\end{definition}

\begin{definition}[Oracle machine]
A \emph{randomized oracle machine} $M$ is a randomized Turing machine that has query access to a string $\pi$. For $\rand$ a setting of $M$'s random tape, we write $S=M(\rand)\subseteq\{1,\ldots,n\}$ for the set of (at most) $r$ queries to $\pi$ made by $M$ on tape $\rand$, and $M(\rand,a)$ or $M(S,a)$ for the decision made by $M$ when the answer to the queries made on tape $\rand$ is $a=\pi_S\in \F_q^S$. We also write $M^\pi(\rand)$ for the same quantity.  
\end{definition}

\begin{definition}[Codeword test]
A $r$-local $\delta(\eps)$-tester for $\code$ is a randomized oracle machine $M$ that makes at most $r$ non-adaptive queries to its oracle and satisfies the following conditions:
\begin{itemize} 
\item (Completeness:) For any $c\in \code$, $\Pr_\rand( M^c(\rand)=1)=1$.
\item (Soundness:) For any $u$ that is within (Hamming) distance at least  $\eps$ from $\code$, $\Pr_\rand(M^u(\rand)=0)\geq \delta(\eps)$. 
\end{itemize}
\end{definition}

\begin{definition}[Local quantum presentation]
Let $\code$ be an $[n,k,d]_q$ linear code and $r\in \N$. An \emph{$r$-local quantum presentation of $\code$} is specified by
\begin{itemize}
\item A tracial von Neumann algebra $(\mM,\tau)$;
\item For each $i\in\{1,\ldots,n\}$, a $q$-outcome projective measurement $\{A^i_a\}_{a\in \F_q} \subseteq \mM$;
\item For each $S\subseteq \{1,\ldots,n\}$ such that $|S|\leq r$, a projective measurement $\{P^S_a\}_{a\in \F_q^S}\subseteq \mM$.
\end{itemize}
\end{definition}

\begin{definition}[Code representation]
Let $\code$ be an $[n,k,d]_q$ linear code. A \emph{representation} of $\code$ is a collection of projective measurements $\{B^i_a\}_{a\in\F_q} \subseteq\mM$ for $i\in\{1,\ldots,n\}$ such that the $\{B^i_a\}$ pairwise commute and $\sum_{c\in\code} \prod_{i=1}^n B^i_{c_i}=\Id$.\footnote{This implies that whenever $u\notin \code$, $\prod_i B^i_{u_i}=0$.} 
\end{definition}

\begin{definition}[Closeness]\label{def:close}
Let $\{A^i_a\}\subseteq \mM$ and $\{B^i_b\}\subseteq \mN$ be two families of projective measurements, indexed by the same sets $i\in \mI$ and $a\in \mA(I)$ but on different tracial algebras $(\mM,\tau^\mM)$ and $(\mN,\tau^\mN)$. We say that $\{A^i\}$ and $\{B_i\}$ are $(\delta,\mu)$-close if there exists a projection $P\in\mM_\infty$ of finite trace such that $\mN=P\mM_\infty P$ and $\tau^\mN=\tau^\mM/\tau^\mM_\infty(P)$, and a partial isometry $w\in P \mM_\infty 1_\mM$ such that 
\[ \Es{i\sim\mu} \sum_a \big\| A^i_a - w^* B^i_a w \big\|_2^2 \,\leq\,\delta\]
and 
\[ \tau^\mM(1-w^*w)\,,\; \tau^\mN(P-w^*w) \,\leq\, \delta\;.\]
\end{definition}

\begin{definition}[Quantum soundness]\label{def:q-sound}
Let $\code$ be an $[n,k,d]_q$ linear code and $M$ a $r$-local tester for $\code$. We say that $M$ has \emph{quantum soundness $\delta(\eps)$} if the following holds. For any $r$-local quantum presentation $(\mN,A,P)$ of $\code$ such that 
\[ \Es{S\leftarrow M(\cdot)} \Es{i\in S} \sum_{a\in \F_q^s} \big[ \tau\big(  P^S_a  A^i_{a_i}\big) \big] \,\geq \,1-\eps\;,\] 
there is a representation $(\mM,B)$ of $\code$ such that $(\mM,A)$ and $(\mM,B)$ are $\delta$-close. 
\end{definition}

Definition~\ref{def:q-sound} can be reformulated in the language of nonlocal games, see Lemma~\ref{lem:sound-game}. 


		
		
\section{Examples}

\subsection{Hadamard code}

The Hadamard code $\code_\had$ is an $[K,k,K/2]_2$ linear code, for any $k\geq 1$ and $K=2^k$. The following randomized oracle machine $M_\had$ is a $3$-local $\delta(\eps)=6\eps$-tester for $\code_\had$: $M$ selects $x,y\in\{1,\ldots,K\}$ uniformly at random and interprets them as $k$-bit strings. $M$ accepts if and only if $\pi_x + \pi_y  = \pi_{x+y}$. Using~\cite[Theorem 10]{natarajan2016robust} it follows that $M$ has quantum soundness $O(\sqrt{\eps})$. 

\subsection{Tensor codes}

Fix a $t_0$-interpolable $[n_0,k_0,d_0]_q$ code $\code_0$, and $m\in \N$. Let $\code = \code_0^{\otimes m}$. Then $\code$ is an $[n,k,d]_q$ code where $n=n_0^m$, $k=k_0^m$ and $d=d_0^m$. 


If $\code_0$ is a Reed-Solomon code with degree $s_0=t_0-1$, then $n_0=q$ and $d_0=n_0-s_0=n_0-t_0+1$.


\section{Nonlocal games}



\begin{definition}[Game]
A game is a tuple $(\mX,\mu,\mA,D)$ where $\mX$ is a finite set, $\mu$ a distribution on $\mX\times \mX$, $\mA=(\mA(x))_{x\in\mX}$ a collection of finite sets, and 
\[ D: \big\{ (x,y,a,b) : (x,y)\in\text{supp}(\mu),a\in\mA(x),b\in\mA(y)\big\} \;\to\;\{0,1\}\]
such that $D$ is symmetric, i.e. $D(x,y,a,b)=D(y,x,a,b)$ whenever both terms are defined. 
\end{definition}
		
\begin{definition}[Synchronous strategy]
If $G=(\mX,\mu,\mA,D)$ is a game and $(\mM,\tau)$ a tracial von Neumann algebra, a \emph{synchronous strategy $\strategy$ for $G$ on $(\mM,\tau)$} is, for every $x\in \mX$, a projective measurement $(P^x_a)_{a\in \mA(x)}$ on $\mM$. The value of a strategy $\strategy$ in $G$ is 
\[ \omega(G;\strategy)\,=\, \sum_{(x,y)\in\mX\times\mX}\mu(x,y) \sum_{(a,b)\in\mA(x)\times\mA(y)} D(x,y,a,b) \tau(P^x_a P^y_b) \;.\]
\end{definition}

\begin{definition}[Robust game]
Given a game $G$, we say that $G$ is \emph{$\delta(\eps)$-robust} if any synchronous strategy for $G$ that succeeds with probability at least $1-\eps$ in $G$ is $(\delta(\eps),\mu_\alice)$-close to a perfect strategy, where we used $\mu_\alice$ to denote the marginal of $\mu$ on $\mX$.  
\end{definition}

Let $\code$ be an $[n,k,d]_q$ linear code and $M$ an $r$-local tester for $\code$. Consider the following game $G_{\code,M}$. We set 
\[\mX = \{ S\subseteq \{1,\ldots,n\},|S|\leq r\} \sqcup\{1,\ldots,n\}\quad\text{and}\quad \mu(S,i)=\frac{1_{i\in S}}{|S|}\Pr_\rand\big(M(\rand)=S\big)\;,\]
$\mA(S)=\F_q^S$ and $\mA(i)=\F_q$, and $D(S,i,a,b)=M(S,a)1_{a_i=b}$. 

\tnote{The next lemma sounds like a special case of some other lemma about LCS games, which is probably written somewhere}

\begin{lemma}\label{lem:sound-game}
Let $\code$ be an $[n,k,d]_q$ linear code and $M$ an $r$-local tester for $\code$. 
If $M$ has quantum soundness $\delta(\eps)$, then $G$ is $O(\delta(\eps))$-robust. Conversely, if $G$ is $\delta'(\eps)$-robust then $M$ has quantum soundness $O(r\delta'(\eps))$. 
\end{lemma}

\begin{proof}[Proof sketch]
Suppose first that $G$ is $\delta'(\eps)$-robust, for some function $\delta'$. Let $(\mM,A,P)$ be be an $r$-local presentation of $\code$ such that 
\begin{equation}\label{eq:s-g-1}
\Es{S\leftarrow M(\cdot)} \Es{i\in S} \sum_{a\in \F_q^s} \big[ \tau\big(  P^S_a  A^i_{a_i}\big) \big] \,\geq \,1-\eps\;.
\end{equation}
Using the definition of $G_{\code,M}$ this equation can be rewritten as
\[ \sum_{(S,i)} \mu(S,i) \sum_{a\in \F_q^s}\sum_{b\in\F_q} D(S,i,a,b) \tau\big(  P^S_a  A^i_{b}\big) \big] \,\geq \,1-\eps\;,\]
or in other words $\omega(G_{\code,M};\strategy)\geq 1-\eps$. Using the definition of robustness we deduce that $(\mM,A,P)$ is $(\delta'(\eps),\mu_\alice)$-close to a perfect strategy for $G$. Specializing to the $A$ measurements, using the definition of closeness there exists a projection $P\in \mM_\infty$ and $\{B^i\}_a \subseteq P\mM_\infty P$ such that $\Es{i}\sum_a \|A^i_a-w^* B^i_a w \|_2^2 \leq 2\delta'$, and moreover $\{B^i_a\}$ satisfy that for any $S$ in the support of $\mu$, $\{B^i_a\}$ pairwise commute for $i\in S$. So all the $\{B^i_a\}$ commute and form a representation of $\code$. 

Conversely, suppose that $M$ has quantum soundness $\delta(\eps)$, for some function $\delta$.
Let $\strategy=(\mM,A,P)$ be a strategy for $G_{\code,M}$ on $(\mM,\tau)$ that succeeds with probability at least $1-\eps$. As above, $(\mM,A,P)$ is automatically an $r$-local presentation of code such that~\eqref{eq:s-g-1} holds. Using quantum soundness, there is a representation $(\mN,B)$ that is $\delta(\eps)$-close. Now, we let $Q^S_a = \prod_{i\in S} B^i_{a_i}$ which is a projective measurement on $\mN$. It is easy to verify that $(\mN,B,Q)$ is a perfect strategy in $G_{\code,M}$, and moreover $(\mN,B,P)$ is $O(r\delta)$-close to $(\mM,A,P)$.\tnote{This last estimate is a guess and needs a calculation}
\end{proof}

\subsection{Commutation game}

See~\cite[Section 3.1]{de2022spectral}. We denote $x_{X,0}$ and $x_{Z,0}$ the two special questions, $x_{com,1}$ and $x_{com,2}$ respectively. 

\subsection{Anti-commutation game}

See~\cite[Section 3.2]{de2022spectral}. We denote $x_{X,1}$ and $x_{Z,1}$ the two special questions, $x_{anticom,1}$ and $x_{anticom,2}$ respectively. 

\section{Main result}

Let $q=2^t$, $\code$ an $[n,k,d]_q$ linear code and $M$ an $r$-local tester for $\code$. 


\begin{figure}[!htbp]
  \centering
  \begin{gamespec}
Let $M$ be an $r$-local tester for the $[n,k,d]_q$ code $\code$. 
    \begin{enumerate}
      \setlength\itemsep{1pt}
    \item (\textbf{Code test}) Select $W\in \{X,Z\}$ uniformly at random, and $\rand$ a setting for $M$'s random tape at random. Let $S=M(\rand)\subseteq\{1,\ldots,n\}$. Sample $i\in S$ uniformly at random. Send $(W,S)$ to $\alice$ and $(W,i)$ to $\bob$. Receive $a\in \F_q^S$ from $\alice$ and $b\in \F_q$ from $\bob$. Accept if and only if $M(\rand,a)=1$ and $a_i = b$.  
    \item (\textbf{Anti-commutation test}) Select $\omega=(i_X,i_Z,r_X,r_Z)\in \{1,\ldots,n\}^2 \times \F_q^2$ uniformly at random. Let $\gamma = \tr( (r_X\ind(i_X))\cdot  (r_Z \ind(i_Z)))$.
		\begin{enumerate}
		\item If $\gamma=0$ then select a pair of questions $(x_c,y_c)$ as in the commutation game. Send $(x_c,\omega)$ to $\alice$ and $(y_c,\omega)$ to Bob. Accept if and only if their answers are accepted in the commutation game. 
		\item If $\gamma\neq 0$ then do the same but for the anti-commutation game. 
		\end{enumerate} 
		 \item (\textbf{Consistency test}) Select $\omega=(i_X,i_Z,r_X,r_Z)\in \{1,\ldots,n\}^2 \times \F_q^2$ and $W\in \{X,Z\}$ uniformly at random. Let $\gamma = \tr( (r_X\ind(i_X))\cdot  (r_Z \ind(i_Z)))$. Send $(W,i_W)$ to $\alice$ and $(x_{W,\gamma},\omega)$ to $\bob$. Receive $a\in \F_q$ and $b\in \F_2$ respectively. Accept if and only if $\tr(a r_W)=b$. 
    \end{enumerate}
  \end{gamespec}
  \caption{The braiding test over $\code$ verifies that the players respond consistently with a uniformly random codeword from $\code$, obtained by measuring $k\log q$ shared EPR pairs in the standard or Hadamard basis.}
  \label{fig:braiding-test}
\end{figure}


\begin{definition}
A $(k,q,\delta(\eps))$-qubit test is a game $G=(\mX,\mu,\mA,D)$ such there are two sets $S_X,S_Z\subseteq \F_q^k$ and an injection $\phi:\{X\}\times S_Z \cup \{Z\}\times S_Z \to \mX$ such that $\mA(\phi({X},a))=\mA(\phi({Z},b))=\F_q$ for all $a\in S_X$, $b\in S_Z$ and such that the following holds:
\begin{itemize}
\item (Completeness:) There is a synchronous strategy for $G$ on $\mM=M_{2^{tk}}(\C)$ that succeeds with probability $1$ in $G$ and is such that $\widehat{P^{\phi({X},a)}}(r) = \tau^X(ra) $, $\widehat{P^{\phi({Z},b)}}(r) = \tau^Z(rb)$ for every $a\in S_X$, $b\in S_Z$ and $r\in \F_q$.
\item (Soundness:) Any synchronous strategy in $(\mM,\tau)$ for $G$ that succeeds with probability at least $1-\eps$ is $\delta(\eps)$-close to a strategy on an algebra of the form $(M_{2^{tk}}(\C)\otimes \mN,\tr\otimes \tau')$ where
\[\widehat{P^{\phi({X},a)}}(r) = \tau^X(ra)\otimes \Id_\mN\;,\qquad \widehat{P^{\phi({Z},b)}}(r) = \tau^Z(rb)\otimes\Id_\mN\;,\]
on average over $a\in S_X$, $b\in S_Z$, and $r\in \F_q$. 
\end{itemize}
\end{definition}

\begin{remark}
The definition allows ``trivial'' settings for the sets $S_X,S_Z$, e.g.\ $S_X=S_Z=\emptyset$. In this case, a strategy for $G$ may not give direct access (through specific questions) to any Pauli observable and the soundness condition is trivial. As soon as e.g. $S_X,S_Z$ both contain the basis elements $e_1,\ldots,e_k \in \F_q^k$ then the definition is non-trivial. 
\end{remark}

\begin{theorem}
Let $\code$ be an $[n,k,d]_q$ linear code and $M$ an $r$-local tester for $\code$ that is $\delta(\eps)$-quantum sound. Then the braiding test over $\code$ is an $(k,q,\delta)$-qubit test, where $S_X=S_Z=\{\ind(i):\,i\in\{1,\ldots,n\}\}\subseteq \F_q^k$ and $\phi(W,\ind(i))=(W,i)$. 
\end{theorem}

\begin{proof}
Completeness: We first verify completeness. For $W\in\{X,Z\}$ let $P^{(W,i)}_a = \sum_{c:\,c\cdot\ind(i)=a} \tau^W_{c}$ and $P^{(W,S)}_a = \prod_{i\in S} P^{(W,i)}_{a_i}$. Let $\omega=(i_X,i_Z,r_X,r_Z)\in \{1,\ldots,n\}^2 \times \F_q^2$ and $\gamma = \tr( (r_X\ind(i_X))\cdot  (r_Z \ind(i_Z)))$. Then $P^{x_{W,\gamma},\omega} = P^{(W,i_W)}$. 

With these choices the strategy succeeds with probability $1$ in the code test and in the consistency test. To define $P^{(x_c,\omega)}$ and $P^{(x_a,\omega)}$ for $x_c\notin \{x_{Z,0},x_{Z,0}\}$ and $x_a \notin \{x_{X,1},x_{Z,1}\}$ we observe that the binary observables 
\[ U = \sum_a (-1)^{\tr(ar_X)} P^{x_{X,\gamma},\omega}_a\quad\text{and}\quad V= \sum_a (-1)^{\tr(a r_Z)} P^{x_{Z,\gamma},\omega}_a \]
commute in case $\gamma=0$ and anti-commute in case $\gamma=1$. This is because 
\begin{align*}
U &=  \sum_a (-1)^{\tr(ar_X)} \sum_{c:\,c\cdot\ind(i_X)=a} \tau^X_{c}\\
&= \sum_c (-1)^{\tr(r_X (c\cdot\ind(i_X)))} \tau^X_{c}\\
&= \tau^X(r_X\ind(i_X))
\end{align*}
and similarly for $V$. As a consequence, the pair $(U,V)$ can be completed to a perfect strategy for the commutation game (if $\gamma=0)$ or anti-commutation game (if $\gamma=1)$. This defines the measurements $\tau^{(x,\omega)}$ for $x\notin \{x_{W,\gamma},W\in\{X,Z\},\gamma\in\{0,1\}\}$. 

Next we show soundness \tnote{todo}. 
\end{proof}


\bibliography{qld}

\notesendofpaper

\end{document}
