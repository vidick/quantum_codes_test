\documentclass[11pt]{article}
\usepackage{booktabs}
\usepackage{fullpage}
\usepackage{titlesec}
%\newcommand{\sectionbreak}{\clearpage}
\usepackage{amsmath,amsfonts,amsthm,mathrsfs,xspace,graphicx}
\usepackage[backref,colorlinks,citecolor=blue,bookmarks=true]{hyperref}
\usepackage{mathpazo}
\usepackage{endnotes}
\usepackage{color}
\usepackage{float}
\usepackage{xcolor}
\usepackage{mdframed}
\usepackage{bbm}
\usepackage{suffix} % for *-version commands
\usepackage{times}
\usepackage{tabularx}
\usepackage{makecell}
\usepackage{amssymb,latexsym}
%\usepackage{IEEEtrantools}
\usepackage[capitalize]{cleveref}
\usepackage{enumitem}
\usepackage{tikz}
\usepackage{tikz-cd}
\usepackage{multirow}
\usepackage[section]{placeins}
\usepackage[affil-it]{authblk}


\mdfdefinestyle{figstyle}{ %
  linecolor=black!7, %
  backgroundcolor=black!7, %
  innertopmargin=10pt, %
  innerleftmargin=25pt, %
  innerrightmargin=25pt, %
  innerbottommargin=10pt %
}

\newtheorem{theorem}{Theorem}[section]
\newtheorem{proposition}[theorem]{Proposition}
\newtheorem{conjecture}[theorem]{Conjecture}
\newtheorem{lemma}[theorem]{Lemma}
\newtheorem{claim}[theorem]{Claim}
\newtheorem{fact}[theorem]{Fact}
\newtheorem{corollary}[theorem]{Corollary}

\newtheorem{remark}[theorem]{Remark}

\theoremstyle{definition}
\newtheorem{definition}[theorem]{Definition}
\newtheorem{example}[theorem]{Example}

\newcommand{\beq}{\begin{eqnarray}}
\newcommand{\eeq}{\end{eqnarray}}

\newcommand{\code}{\mathcal{C}}
\newcommand{\strategy}{\mathscr{S}}
\newcommand{\algebra}{\mathscr{A}}

\newcommand{\ket}[1]{|#1\rangle}
\newcommand{\bra}[1]{\langle#1|}
\newcommand{\ketbra}[2]{\ket{#1}\!\bra{#2}}
\newcommand{\ip}[2]{\langle #1 \! | #2 \rangle}
\newcommand{\proj}[1]{\ket{#1}\!\bra{#1}}
\newcommand{\Tr}{\mbox{\rm Tr}}
\newcommand{\Id}{\ensuremath{I}}
\DeclareMathOperator*{\Expectation}{\mathbb{E}}
\newcommand{\Es}[1]{\Expectation_{#1}}

\newcommand{\reg}[1]{{\textsf{#1}}}
\newcommand{\ol}[1]{\overline{#1}}

\newcommand{\field}{\mathbb{F}_2}
\newcommand{\C}{\ensuremath{\mathbb{C}}}
\newcommand{\N}{\ensuremath{\mathbb{N}}}
\newcommand{\bbN}{\ensuremath{\mathbb{N}}}
\newcommand{\complex}{\ensuremath{\mathbb{C}}}
\newcommand{\real}{\ensuremath{\mathbb{R}}}
%\newcommand{\natural}{\ensuremath{\mathbb{N}}}

\newcommand{\bij}{\pi}
\newcommand{\qp}{\tau}
\newcommand{\dlS}{\ensuremath{\rm dlS}}

\newcommand{\F}{\ensuremath{\mathbb{F}}}
\newcommand{\M}{\ensuremath{\mathbb{M}}}
\newcommand{\ot}{\otimes}
\newcommand{\Fp}{\F_p}
\newcommand{\Fq}{\field}
\newcommand{\BH}{\textsc{BH}}
\newcommand{\ld}{\textsc{ld}}
\newcommand{\downsize}{\kappa}
\newcommand{\tobin}{\flat}
\newcommand{\downsizem}{\chi}

\newcommand{\K}{\ensuremath{\mathbb{K}}}
\newcommand{\R}{\ensuremath{\mathbb{R}}}
\newcommand{\Z}{\ensuremath{\mathbb{Z}}}

\newcommand{\mA}{\ensuremath{\mathcal{A}}}
\newcommand{\mB}{\ensuremath{\mathcal{B}}}
\newcommand{\mC}{\ensuremath{\mathcal{C}}}
\newcommand{\mE}{\ensuremath{\mathcal{E}}}
\newcommand{\mD}{\ensuremath{\mathcal{D}}}
\newcommand{\mF}{\ensuremath{\mathcal{F}}}
\newcommand{\mG}{\ensuremath{\mathcal{G}}}
\newcommand{\mH}{\ensuremath{\mathcal{H}}}
\newcommand{\mK}{\ensuremath{\mathcal{K}}}
\newcommand{\mM}{\ensuremath{\mathcal{M}}}
\newcommand{\mI}{\ensuremath{\mathcal{I}}}
\newcommand{\mJ}{\ensuremath{\mathcal{J}}}
\newcommand{\cM}{\ensuremath{\mathcal{M}}}
\newcommand{\mP}{\ensuremath{\mathcal{P}}}
\newcommand{\mQ}{\ensuremath{\mathcal{Q}}}
\newcommand{\mR}{\ensuremath{\mathcal{R}}}
\newcommand{\mS}{\ensuremath{\mathcal{S}}}
\newcommand{\mT}{\ensuremath{\mathcal{T}}}
\newcommand{\mU}{\ensuremath{\mathcal{U}}}
\newcommand{\mX}{\ensuremath{\mathcal{X}}}
\newcommand{\mY}{\ensuremath{\mathcal{Y}}}

\newcommand{\Inv}{\ensuremath{\textsc{Inv}}}
\newcommand{\GEN}{\ensuremath{\textsc{GEN}}}
\newcommand{\SAMP}{\ensuremath{\textsc{SAMP}}}
\newcommand{\epr}{\ensuremath{\textsc{epr}}}
\newcommand{\RM}{\ensuremath{\textsc{RM}}}
\newcommand{\Had}{\ensuremath{\textsc{Had}}}
\newcommand{\HRM}{\ensuremath{\textsc{HRM}}}


\newcommand{\Alg}{\mathcal{A}}
\newcommand{\ind}{\ensuremath{\mathrm{ind}}}


\newcommand{\setft}[1]{\mathrm{#1}}
\newcommand{\Density}{\setft{D}}
\newcommand{\Pos}{\setft{Pos}}
\newcommand{\Proj}{\setft{Proj}}
\newcommand{\Channel}{\setft{C}}
\newcommand{\Unitary}{\setft{U}}
\newcommand{\Herm}{\setft{Herm}}
\newcommand{\Obs}{\setft{Obs}}
\newcommand{\Lin}{\setft{L}}
\newcommand{\Trans}{\setft{T}}
\DeclareMathOperator{\poly}{poly}
\DeclareMathOperator{\negl}{negl}
\newcommand{\dset}{G}

\newcommand{\val}{\ensuremath{\mathrm{val}}}
\newcommand{\valco}{\ensuremath{\mathrm{val}^{\mathrm{co}}}}
\newcommand{\ia}{\Id_\alice}
\newcommand{\ib}{\Id_\bob}

\newcommand{\desc}[1]{\overline{\cal{#1}}}
\newcommand{\supp}{\textsc{Supp}}
\newcommand{\Gen}{\textsc{Gen}}
\newcommand{\Enc}{\textsc{Enc}}
\newcommand{\Dec}{\textsc{Dec}}

\newcommand{\GenTrap}{\textsc{GenTrap}}
\newcommand{\Invert}{\textsc{Invert}}
\newcommand{\lossy}{\textsc{lossy}}

\newcommand{\rand}{\textrm{rand}}
\newcommand{\had}{\textsc{Had}}


\newcommand{\eps}{\varepsilon}
\newcommand{\ph}{\ensuremath{\varphi}}


\newcommand{\ac}{\textsc{ac}}
\newcommand{\GX}{\textsc{Gap-Maxcut}}
\newcommand{\GNI}{\textsc{Graph Non-Isomorphism}}


\newcommand{\Acc}{\textsc{Acc}}
\newcommand{\Samp}{\textsc{Samp}}
\newcommand{\Ext}{\ensuremath{\text{Ext}}}

\newcommand{\BD}{\mathbb{QB}}
\newcommand{\DD}{\mathbb{D}}
\newcommand{\DDb}{\mathbb{D'}}
\newcommand{\Pot}{\Phi}
\newcommand{\inj}{J}
\newcommand{\mZ}{\mathcal{Z}}
\newcommand{\mN}{\mathcal{N}}
\newcommand{\vs}{\vspace{2mm}~\newline\noindent}
\newcommand{\vb}{\vspace{3mm}\noindent}
\newcommand{\sX}{\mathcal{X}}
\newcommand{\sA}{\mathcal{A}}
\newcommand{\sB}{\mathcal{B}}
\newcommand{\sY}{\mathcal{Y}}
\newcommand{\sR}{\mathcal{R}}


\newcommand{\trnq}[1]{\left[ {#1} \right]_q}

\DeclareMathOperator{\polylog}{polylog}
\newcommand{\mx}[1]{\mathbf{{#1}}}
\newcommand{\vc}[1]{\mathbf{{#1}}}
\newcommand{\abs}[1]{\left\vert {#1} \right\vert}
\newcommand{\norm}[1]{\left\| {#1} \right\|}
\newcommand{\for}{\text{for }}

\DeclareMathOperator{\arcsinh}{arcsinh}
\DeclareMathOperator{\tr}{tr}

\newcommand{\E}{\mathop{\mathbb{E}}\displaylimits} % Expectation

\newcommand{\unif}{\mathcal{U}}
\newcommand{\pt}{\textrm{pt}}
\newcommand{\sample}{\textrm{sample}}
\newcommand{\test}{\textrm{test}}
\newcommand{\free}{\mathcal{F}}
\newcommand{\plane}{\mathcal{P}}
\newcommand{\lines}{\mathcal{L}}
\newcommand{\clines}{\mathcal{CL}}
\newcommand{\pl}{\mathbf{p}}
\newcommand{\individual}{\textrm{individual}}
\newcommand{\blocks}{\textrm{blocks}}
\newcommand{\liness}{\textrm{lines}}
\newcommand{\lp}{\mathcal{LP}}
\newcommand{\Pl}{\ensuremath{\mathrm{Pl}}}
\newcommand{\Ln}{\ensuremath{\mathrm{Lines}}}
\newcommand{\mode}{\mathfrak{m}}
\newcommand{\ECC}{\ensuremath{\textsc{ECC}}}
\newcommand{\EC}{\ensuremath{\textsc{EC}}}
\newcommand{\ENC}{\ensuremath{\textsc{ENC}}}
\newcommand{\cktval}{\ensuremath{\textsc{CKTVAL}}}


\newcommand{\GL}{\mathrm{GL}}
\newcommand{\Matrix}{\mathrm{M}}
\newcommand{\End}{\mathrm{End}}
\newcommand{\Aut}{\mathrm{Aut}}

\newcommand{\game}{\mathfrak{G}}
\newcommand{\sampler}{\mathcal{S}}
\newcommand{\decider}{\mathcal{D}}
\newcommand{\verifier}{\mathcal{V}}


\newcommand{\type}{\mathcal{T}}
\newcommand{\lt}{\mathcal{L}}
\newcommand{\rt}{\mathcal{R}}
\newcommand{\checker}{\mathcal{C}}


\newcommand{\gamestyle}[1]{\ensuremath{\textsc{#1}}\xspace}
\newcommand{\qld}{\gamestyle{QLD}}
\newcommand{\ms}{\gamestyle{MS}}
\newcommand{\pauli}{\gamestyle{P}}
%\newcommand{\bp}{\gamestyle{BP}}
\newcommand{\ora}{\gamestyle{Orac}}
\newcommand{\pcp}{\gamestyle{PCP}}
\newcommand{\ar}{\gamestyle{AR}}
\newcommand{\intro}{\gamestyle{Intro}}

\newcommand{\labelstyle}[1]{\ensuremath{\textsc{#1}}\xspace}
\newcommand{\EPR}{\labelstyle{EPR}}
\newcommand{\aux}{\labelstyle{aux}}
\newcommand{\ancilla}{\labelstyle{anc}}
\newcommand{\msc}{\labelstyle{MC}}
\newcommand{\msv}{\labelstyle{MV}}
\newcommand{\vertex}[1]{\labelstyle{V#1}}
\newcommand{\edge}[1]{\labelstyle{N#1}}
\newcommand{\basis}{\labelstyle{W}}
\newcommand{\xpt}{\labelstyle{X}}
\newcommand{\zpt}{\labelstyle{Z}}
\newcommand{\rxpt}{\labelstyle{R}_\xpt}
\newcommand{\rzpt}{\labelstyle{R}_\zpt}
\newcommand{\dir}[1]{\labelstyle{V#1}}
\newcommand{\coord}{\labelstyle{I}}
\newcommand{\intercept}{\labelstyle{U}}
\newcommand{\plf}{\labelstyle{Pl}}
\newcommand{\lnf}{\labelstyle{Ln}}
\newcommand{\ptf}{\labelstyle{Pt}}
\newcommand{\full}{\labelstyle{full}}
\newcommand{\opt}{\labelstyle{opt}}
\newcommand{\partition}{\mathcal{B}}

\newcommand{\tvarstyle}[1]{\mathsf{#1}}
\newcommand{\tvar}{\ensuremath{\tvarstyle{t}}}
\newcommand{\lvar}{\ensuremath{\tvarstyle{u}}}
\newcommand{\rvar}{\ensuremath{\tvarstyle{v}}}
\newcommand{\pvar}{\ensuremath{\tvarstyle{p}}}
\newcommand{\ovar}{\ensuremath{\tvarstyle{o}}}
\newcommand{\trole}{\ensuremath{v}} % used in intro types

\newcommand{\types}{\labelstyle{T}}

\newcommand{\decode}{\labelstyle{Decode}}

%\newcommand{\alice}{\labelstyle{Alice}}
%\newcommand{\bob}{\labelstyle{Bob}}
\newcommand{\alice}{\labelstyle{A}}
\newcommand{\bob}{\labelstyle{B}}

\newcommand{\oracle}{\labelstyle{Oracle}}
\newcommand{\ab}{\{\alice, \bob\}}

\newcommand{\typestyle}[1]{\ensuremath{\textsc{#1}}\xspace}
\newcommand{\Type}{\typestyle{Type}}
\newcommand{\Plane}{\typestyle{Plane}}
\renewcommand{\line}{\mathbf{\ell}}
\newcommand{\Llane}{\typestyle{Line}}
\newcommand{\Point}{\typestyle{Point}}
\newcommand{\HPoint}{\typestyle{HPoint}}
\newcommand{\Line}{\typestyle{Line}}
\newcommand{\ALine}{\typestyle{ALine}}
\newcommand{\DLine}{\typestyle{DLine}}
\newcommand{\Pair}{\typestyle{Pair}}
\newcommand{\Constraint}{\typestyle{Constraint}}
\newcommand{\Variable}{\typestyle{Variable}}
\newcommand{\Pauli}{\typestyle{Pauli}}
\newcommand{\Sample}{\typestyle{Sample}}
\newcommand{\Read}{\typestyle{Read}}
\newcommand{\MeasureX}{\typestyle{MeasureX}}
\newcommand{\Hide}[1]{\typestyle{Hide}_{#1}}
\newcommand{\HideX}[1]{\typestyle{HideX}_{#1}}
\newcommand{\Target}[1]{\typestyle{Target}_{#1}}
\newcommand{\Oracle}{\typestyle{Oracle}}
\newcommand{\Introspect}{\typestyle{Intro}}
\newcommand{\Intro}{\typestyle{Intro}}
\newcommand{\Simple}{\typestyle{Simple}}
\newcommand{\Eval}{\typestyle{Eval}}
\newcommand{\Agg}{\typestyle{Agg}}
\newcommand{\Input}{\typestyle{Input}}
\newcommand{\Skip}{\typestyle{Skip}}
\newcommand{\Alice}{\typestyle{Alice}}
\newcommand{\Bob}{\typestyle{Bob}}
\newcommand{\Edge}{\typestyle{Alice}}
\newcommand{\Vertex}{\typestyle{Bob}}
\newcommand{\Anchor}{\typestyle{Anchor}}
\renewcommand{\Game}{\typestyle{Game}}
\newcommand{\AB}{\{\alice, \bob\}}
\newcommand{\ctrl}{\labelstyle{c}}
\newcommand{\target}{\labelstyle{t}}

\newcommand{\abc}[1][\delta]{\otimes I_\bob \simeq_{#1} I_\alice \otimes}

\newcommand{\ldc}{k} % number of copies of classical ld tests

\newcommand{\class}[1]{\ensuremath{\mathsf{#1}}\xspace}
\newcommand{\NP}{\class{NP}} %
\newcommand{\IP}{\class{IP}} %
\newcommand{\EXP}{\class{EXP}} %
\newcommand{\NEXP}{\class{NEXP}} %
\newcommand{\QMA}{\class{QMA}} %
\newcommand{\QMIP}{\class{QMIP}} %
\WithSuffix\newcommand\QMIP*{\ensuremath{\class{QMIP}^*}} %
\newcommand{\PSPACE}{\class{PSPACE}} %
\newcommand{\PCP}{\class{PCP}} %
\newcommand{\MIP}{\class{MIP}} %
\newcommand{\MIPco}{\class{MIP}^{\mathrm{co}}} %
\newcommand{\RE}{\class{RE}} %
\newcommand{\coRE}{\class{coRE}}
\newcommand{\NEEXP}{\class{NEEXP}} %
\newcommand{\NEEEXP}{\class{NEEEXP}}
\WithSuffix\newcommand\MIP*{\ensuremath{\class{MIP}^*}} %
\newcommand{\QIP}{\class{QIP}} %


\newcommand{\Ent}{\mathscr{E}}
\newcommand{\compr}{\textsc{Compr}}
\newcommand{\halt}{\textsc{Halt}}
\newcommand{\machine}{\cal{M}}
\renewcommand{\cal}[1]{\mathcal{#1}}
\newcommand{\Kleene}{\cal{K}}
\newcommand{\qldparams}{\mathsf{qldparams}}
\mathchardef\mhyphen="2D
\newcommand{\Fqldparams}{\F_2\mhyphen\mathsf{qldparams}}
\newcommand{\introparams}{\mathsf{introparams}}
\newcommand{\ldparams}{\mathsf{ldparams}}
\newcommand{\tmldparams}{\mathsf{tmldparams}}
\newcommand{\pcpparams}{\mathsf{pcpparams}}

\newcommand{\TMtoSAT}{\mathrm{TMtoSAT}}
\newcommand{\TMtoLD}{\mathrm{TMtoLD}}
\newcommand{\BoundedHalting}{\mathrm{BH}}
\newcommand{\timecomplexity}{\mathsf{TIME}}
\newcommand{\TIME}{\mathsf{TIME}}
\newcommand{\answer}{\mathsf{ANS}}
\newcommand{\MS}{\mathrm{MS}}

\newcommand{\accept}{\typestyle{Accept}}
\newcommand{\reject}{\typestyle{Reject}}

\newcommand{\anch}{\gamestyle{Anch}}
\newcommand{\ans}{\gamestyle{ANS}}
%%%%%%%self testing macros%%%%%%%%%%

\newcommand{\local}{\mathrm{local}}
%\newcommand{\aux}{\mathrm{aux}}


\newcommand{\G}{\mG}
\newcommand{\XZ}{\mathcal{B}}
\newcommand{\hilb}{\mathcal{H}}


%\newcommand{\tmstyle}[1]{\ensuremath{\textsf{#1}}}
\newcommand{\tmstyle}[1]{\ensuremath{\mathsf{#1}}}
\newcommand{\Compress}{\tmstyle{Compress}}
\newcommand{\ComputeRepetitions}{\tmstyle{ComputeRepetitions}}
\newcommand{\ComputeSampler}{\tmstyle{ComputeSampler}}
\newcommand{\RawIntroSampler}{\tmstyle{RawIntroSampler}}
\newcommand{\ComputeIntroSampler}{\tmstyle{IntroSampler}}
\newcommand{\RawIntroDecider}{\tmstyle{RawIntroDecider}}
\newcommand{\ComputeIntroDecider}{\tmstyle{IntroDecider}}
\newcommand{\ComputeIntroVerifier}{\tmstyle{IntroVerifier}}
\newcommand{\ComputeOracleVerifier}{\tmstyle{OracleVerifier}}
\newcommand{\ComputeAnsVerifier}{\tmstyle{AnsRedVerifier}}
\newcommand{\ComputeParrepVerifier}{\tmstyle{RepeatedVerifier}}
\newcommand{\ComputePCPVerifier}{\tmstyle{PCPVerifier}}
\newcommand{\ComputeFixedPoint}{\tmstyle{ComputeFixedPoint}}
\newcommand{\detype}{\tmstyle{Detype}}

\newenvironment{gamespec}{
  \begin{mdframed}[style=figstyle]}{
  \end{mdframed}}

\newcommand{\zero}{\mathrm{zero}}

%%%%%%%From NW19:%%%%%%%%%%
\newcommand{\polymeas}[3]{\mathrm{PolyMeas}(#1,#2,#3)}
\newcommand{\simulpolymeas}[4]{\mathrm{PolyMeas}(#1,#2,#3, #4)}

\newcommand{\eval}{\mathrm{eval}}

%\newcommand{\coin}{o}
\newcommand{\succinctdecider}{\ensuremath{\mathsf{SuccinctDecider}}}
\newcommand{\circuit}{\mathcal{C}}
\newcommand{\formula}{\mathcal{F}}
\newcommand{\bin}{\mathrm{binary}}
\newcommand{\pcpeval}{\Xi}
\newcommand{\pcpverifier}{\mathcal{M}_\ar}
\newcommand{\qlen}{Q}
\DeclareMathOperator{\ev}{eval}

\newcommand{\coded}{\mathrm{Dec}}
\newcommand{\hx}{\hat{x}}
\newcommand{\hz}{\hat{z}}
\newcommand{\htvar}{\hat{\tvar}}
\newcommand{\soundness}{\mathrm{sound}}

\newcommand{\rep}{\gamestyle{Rep}}
\newcommand{\sep}{\gamestyle{Sep}}

\newcommand{\binary}[1]{\mathrm{binary}_{#1}}
\newcommand{\num}[1]{\mathrm{number}_{#1}}
\newcommand{\canbasis}[1]{\mathrm{basis}(#1)}
\newcommand{\canH}[3]{H_{\mathrm{canon}, #1, #2, #3}}
\newcommand{\canlilh}[3]{h_{\mathrm{canon}, #1, #2, #3}}
\newcommand{\canin}[3]{\pi_{\mathrm{canon},#1,#2,#3}}
\newcommand{\canenc}[4]{g_{\mathrm{canon},#1,#2,#3,#4}}


% \usepackage{showlabels}
% \renewcommand{\showlabelfont}{\tiny\ttfamily\color{red}}

\bibliographystyle{alpha}

\newif\ifnotes\notestrue
%\newif\ifnotes\notesfalse


% MARGIN NOTES

\ifnotes
\usepackage{color}
\definecolor{mygrey}{gray}{0.50}
\newcommand{\notename}[2]{{\textcolor{mygrey}{\footnotesize{\bf (#1:} {#2}{\bf ) }}}}
\newcommand{\noteswarning}{{\begin{center} {\Large WARNING: NOTES ON}\endnote{Warning: notes on}\end{center}}}
\newcommand{\notesendofpaper}{{\theendnotes}}

\newcommand{\pnote}[1]{\textcolor{blue}{\small {\textbf{(MLN:} #1\textbf{)
      }}}}
\newcommand{\tnote}[1]{\textcolor{magenta}{\small {\textbf{(Thomas:} #1\textbf{)
      }}}}
\newcommand{\jnote}[1]{\textcolor{violet}{\small {\textbf{(John:} #1\textbf{) }}}}
\newcommand{\anote}[1]{\textcolor{red}{\small {\textbf{(Anand:} #1\textbf{) }}}}
\newcommand{\znote}[1]{\textcolor{cyan}{\small {\textbf{(Zhengfeng:} #1\textbf{) }}}}
\newcommand{\hnote}[1]{\textcolor{olive}{\small {\textbf{(Henry:} #1\textbf{) }}}}
\newcommand{\ftnote}[1]{\footnote{\textcolor{magenta}{\small {\textbf{(Thomas:} #1\textbf{) }}}}}
\newcommand{\tdnote}[1]{\textcolor{blue}{\small {\textbf{(TODO:} #1\textbf{) }}}}

\else

\newcommand{\notename}[2]{{}}
\newcommand{\noteswarning}{{}}
\newcommand{\notesendofpaper}{}
\newcommand{\pnote}[1]{}

\newcommand{\tnote}[1]{}
\newcommand{\jnote}[1]{}
\newcommand{\anote}[1]{}
\newcommand{\znote}[1]{}
\newcommand{\hnote}[1]{}
%\newcommand{\ftnote}[1]{\footnote{\textcolor{magenta}{\small {\textbf{(Thomas:} #1\textbf{) }}}}}
%\newcommand{\tdnote}[1]{\textcolor{blue}{\small {\textbf{(TODO:} #1\textbf{) }}}}

\fi


\begin{document}

\title{Efficient stability for the Pauli group}

\author{Thomas Vidick and Henry Yuen}
\date{\today}
\maketitle

\noteswarning


\begin{abstract}

\end{abstract}


\section{Introduction}

\tnote{Make the connection with LCS}


	\section{Preliminaries}

\subsection{Notation}

When we write $\Es{i\in \mX}$ where $\mX$ is a finite set, we mean the expectation over $i$ chosen uniformly at random from $\mX$, i.e.\ $\frac{1}{|\mX|} \sum_{i\in \mX}$. For a vector $u \in \mX^n$ and a subset $S \subseteq [n]$, we write $u_S$ to denote the vector in $\mX^S$ which is the restriction of $u$ to $S$.

\subsection{Algebra}

  A \emph{tracial von Neumann algebra} is a pair $(\mM,\tau)$ of a von Neumann algebra $\mM$ together with a normal faithful tracial state $\tau$ on $\mM$, which we often refer to as the \emph{trace}. The main example of interest is $\mM=M_n(\C)$, the algebra $n\times n$ complex matrices, with $\tau$ the dimension-normalized trace, which we denote $\tr(M)=\frac{1}{n}\Tr(M)$. 	We write $\|x\|_2=\tau(x^*x)^{1/2}$ for the $2$-norm on $\mM$.
	
	Let $B(\ell_2)$ be the von Neumann algebra of bounded operators on $\ell_2$, the Hilbert space of convergent sequences in $\C^\Z$ equipped with the usual Euclidean norm (for which we let $(e_i)_{i \in \Z}$ denote the standard basis). We denote $\mM_\infty = \mM \overline{\otimes} B(\ell_2)$, where the overline denotes closure for the operator topology. $\mM_\infty$ is a von Neumann algebra equipped with the (infinite) trace $\tau_\infty = \tau \otimes \Tr$, with $\Tr(x)=\sum_{i\in \Z} e_i^T X e_i$ the trace on $B(\ell_2)$. We generally identify $\mM$ with the ``corner'' $\mM\otimes e_{1,1}\subset \mM_\infty$. 

	We let $\F$ denote a finite field, and $\field$ the field with two elements. For $u\in \F^n$ for some $n$, we write $|u|$ for the Hamming weight of $u$, i.e.\ the number of nonzero coordinates. For $a,b \in \F^k$, we write $a \cdot b$ to denote the inner product $\sum_{i=1}^k a_i b_i$. 
	
	
	\subsection{Measurements}
	\label{sec:measurements}
	
	A POVM in $\mM$ with outcome set $\mA$ is a finite collection of positive semidefinite operators $\{P_a\}_{a\in \mA}$ such that $\sum_a P_a = \Id_\mM$. A POVM is \emph{projective} if for all $a$, $P_a$ is a projection. 
	Given a projective measurement $\{P_a\}_{a\in \field^k}$ and $b\in \field^k$ we define the corresponding \emph{observable} 
	\[ \widehat{P}(b) = \sum_a (-1)^{a\cdot b} P_a\;,\]
	which is self-adjoint and unitary. If $k=1$, we often use the shorthand $\widehat{P}$ for $\widehat{P}(1) = P_0-P_1$.
	
	We define a specific family of projective measurements on $M_{2^k}(\C)$ which are derived from the \emph{Pauli observables}. Define
	\[ \sigma^X = \begin{pmatrix} 0 & 1 \\ 1 & 0 \end{pmatrix}\;,\qquad \sigma^Z = \begin{pmatrix} 1 & 0 \\ 0 & -1\end{pmatrix}\;,\]
	and more generally for $a,b\in \F_2^k$ let $\sigma^X(a) = \bigotimes_{i=1}^t (\sigma^X)^{a_i}$ and $\sigma^Z(b) = \bigotimes_{i=1}^t (\sigma^Z)^{b_i}$, which are observables in $M_{2^k}(\C)$. These are self-adjoint unitary operators called Pauli observables. Each observable $\sigma^X(a)$ (resp. $\sigma^Z(b)$) corresponds to the \emph{Pauli measurement} $\{ \sigma^X_a \}_{a \in \F_2^k}$ (resp. $\{ \sigma^Z_b \}_{b \in \F_2^k}$) where (in a slight abuse of notation)  
	%	We slightly abuse notation and write 
	\[ \sigma^X_a = \Es{\alpha\in\F_2^k} (-1)^{a\cdot \alpha} \sigma^X(\alpha)\qquad\text{and}\qquad\sigma^Z_b = \Es{\beta\in\F_2^k} (-1)^{b\cdot\beta} \sigma^Z(\beta).\]
	It is easy to verify that $\{\sigma^X_a\}_a$ and $\{\sigma^Z_b\}_b$ are projections summing to identity.	
	
	
\subsection{Nonlocal games}

We give standard definitions on nonlocal games. 

\begin{definition}[Game]
A game is a tuple $(\mX,\mu,\mA,D)$ where $\mX$ is a finite set, $\mu$ a distribution on $\mX\times \mX$, $\mA=(\mA(x))_{x\in\mX}$ a collection of finite sets, and 
\[ D: \big\{ (x,y,a,b) : (x,y)\in\text{supp}(\mu),a\in\mA(x),b\in\mA(y)\big\} \;\to\;\{0,1\}\]
such that $D$ is symmetric, i.e. $D(x,y,a,b)=D(y,x,b,a)$ whenever both terms are defined. We often abuse notation and write $\mu$ for the symmetrized marginal of $\mu$, i.e.\ 
\[\mu(x) := \sum_{x'\in \mX} \frac{1}{2}\big(\mu(x,x')+\mu(x',x')\big)\;.\]
\end{definition}

The interpretation of $G=(\mX,\mu,\mA,D)$ as a nonlocal game is the following. In the ``game'', a referee is imagined to sample a pair of ``questions'' $(x,y)\sim \mu$. The question $x$ is sent to a first player, ``Alice,'' and the question $y$ is sent to a second player, ``Bob.'' Each player is tasked with responding with an answer $a\in \mA(x)$ for Alice, and $b\in \mA(y)$ for Bob. The referee accepts the players' answers if and only if $D(x,y,a,b)=1$. 

Nonlocal games provide a framework to study different kinds of bipartite correlations: depending on the level of coordination allowed between Alice and Bob, they may have varying chances of success in the game. 

In quantum mechanics, a local strategy for the players (meaning that each player is required to determine their answer locally, without exchanging information with the other player) is specified by the following. 

\begin{definition}[Synchronous strategy]
If $G=(\mX,\mu,\mA,D)$ is a game and $(\mM,\tau)$ a tracial von Neumann algebra, a \emph{synchronous strategy $\strategy$ for $G$ on $(\mM,\tau)$} is, for every $x\in \mX$, a projective measurement $(P^x_a)_{a\in \mA(x)}$ on $\mM$. The value of a strategy $\strategy$ in $G$ is 
\[ \omega(G;\strategy)\,=\, \sum_{(x,y)\in\mX\times\mX}\frac{1}{2}\big(\mu(x,y)+\mu(y,x)\big) \sum_{(a,b)\in\mA(x)\times\mA(y)} D(x,y,a,b)\, \tau\big(P^x_a \,P^y_b\big) \;.\footnote{Note the symmetrization of $\mu$. This is to avoid explicitly requiring $\mu$ to be permutation-invariant in the definition of a game.}\]
We say that $\strategy$ is \emph{perfect} if $\omega(G;\strategy)=1$.
\end{definition}
	
The name \emph{synchronous} stems from the fact that whenever an identical pair $(x,x)$ is chosen, $\tau(P^x_a P^x_b)=0$ for $a\neq b$ due to the requirement that $\{P^x_a\}_a$ is a projective measurement. Thus a synchronous strategy always returns the same answer to the same question. More general strategies, which allow different operators $\{P^x_a\}$ and $\{Q^y_b\}$, do not automatically enforce the synchronicity condition, but we do not consider such strategies here. 
	
		
\section{Efficient stability}

\subsection{Approximate homomorphisms}

Given a set $S$, we let $\mF(S)$ denote the free group generated by $S$. We freely identify functions from $S$ $H$, where $H$ is any group, with homomorphisms from $\mF(S)$ to $H$. If $R$ is a subset of $\mF(S)$ then the quotient of $\mF(S)$ by the normal subgroup generated by $R$ is denoted $\langle S:R\rangle$. 

For convenience we adopt the following notation from~\cite{}. A \emph{group over $\Z_2$} is a pair $(G,J)$ of a finitely presented group $G$ and a central element $J$ of $G$ of order $2$. Any such group has a presentation $G=\langle S:R\rangle$ where $J\in S$ and $R$ includes the relations $J^2=e$ and $[J,s]=e$ for every $e\in S\backslash\{J\}$. We use the notation 
\[ \langle S:R\rangle_{\Z_2} \,=\, \langle S\cup\{J\} : R\cup\{[J,s]=e:s\in S\}\cup\{J^2=e\}\rangle\;.\]
In~\cite[Section 2]{hadwin2018stability} a notion of $eps$-\emph{almost homomorphism} from a finite group to a unital tracial $C^*$-algebra is introduced. Informally, an $\eps$-almost homomorphism of a finitely presented group $G=\langle S:R\rangle$ is a map from $S$ to $\mU(\mA)$ that approximately respects all relations in $R$. We give a variant of their definition that quantifies the error in an average sense. 

\begin{definition}
Let $G = \langle S:R\rangle $ be a finitely presented group, $\mu$ a distribution on $R$, and $(\mM,\tau)$ a tracial von Neumann algebra. An $(\eps,\mu)$-almost homomorphism of $G$ on $(\mM,\tau)$ is a homomorphism $\phi:\mF(S)\to\mU(\mN)$ such that
\[ \Es{r\sim \mu} \big\|  \phi(r) - \Id \|_\tau^2 \,\leq\, \eps\;.\]
\end{definition}

We note that this notion depends on the presentation of $G$, not only on the group itself. 
When the distribution $\mu$ is uniform over the set $R$, we simply write $\eps$-homomorphism. 

A stability result states that $\eps$-homomorphisms are close to exact hommorphisms. To measure the distance between homomorphisms into different algebras we make the following definition. 


\begin{definition}[Closeness]\label{def:close}
Let $\{U_i\}\subseteq \mM$ and $\{V_i\}\subseteq \mN$ be two families of unitaries on  tracial algebras $(\mM,\tau^\mM)$ and $(\mN,\tau^\mN)$ respectively, indexed by the same set $i\in \mI$. For $\delta\geq0$ and $\mu$ a measure on $\mI$ we say that $\{U_i\}$ and $\{V_i\}$ are $(\delta,\mu)$-close if there exists a projection $P\in\mM_\infty$ of finite trace such that $\mN=P\mM_\infty P$ and $\tau^\mN=\tau_\infty/\tau_\infty(P)$, and a partial isometry $w\in P \mM_\infty \Id_\mM$ such that 
\[ \Es{i\sim\mu} \big\| U_i - w^* V_i w \big\|_2^2 \,\leq\,\delta\]
and 
\[\max\big\{ \tau^\mM(\Id_\mM-w^*w)\,,\; \tau^\mN(P-ww^*)\big\} \,\leq\, \delta\;.\]
If the measure $\mu$ is omitted then it is understood to be the uniform measure on $\mI$.
\end{definition}

We now give our definition of stability. 

\begin{definition} 
Let $G = \langle S:R\rangle $ be a finitely presented group. Let $\mC$ be a class of tracial von Neumann algebras. Let $\mu_R$ be a distribution on $R$ and $\mu_S$ a distribution on $S$. For $\delta:[0,1]\to[0,1]$ such that $\lim_{t\to 0}\delta(t)=0$ we say that $(G,\mu_S,\mu_R)$ is $(\delta,\mC)$-stable if for every $(\cM,\tau)$ in $\mC$, every $(\eps,\mu_R)$-almost homomorphism of $G$ is $(\delta(\eps),\mu_S)$-close to a representation of $G$ on some $(\mN,\tau^\mN)\in \mC$. 
\end{definition}


\subsection{Examples}

For a finite group $G$ we can always write $G=\langle S:R\rangle$ where $S = G$ and $R=\{ g\cdot h \cdot (gh)^{-1} =e \}$. If we let $\mu_R$ and $\mu_S$ be the uniform distribution on $R$ and on $S$ respectively then we recover a widely used notion of \emph{flexible (Hilbert-Schmidt) stability}. In particular, for finite groups the following result is known~\cite{gh,thom}.

\begin{theorem}\label{thm:gh}[Theorem~1.4 in~\cite{de2022spectral}]
Let $G$ be a finite group and $\mC$ the class of all tracial von Neumann algebras. Let $U_R$ and $U_S$ be the uniform distribution on $R=\{ g\cdot h \cdot (gh)^{-1}=e \}$ and $S=G$ respectively. Then $(G,U_S,U_R)$ is $(O(\eps),\mC)$-stable. 
\end{theorem}

We spell out the application of this theorem to the case of $G=\Z_2^k$.

\begin{corollary}\label{cor:lin-test}
Let $\phi:\Z_2^k \to \Obs(\C^d)$ be such that 
\[ \Es{x,y\in \Z_2^k} \big\| \phi(x)\phi(y)-\phi(x+y) \big\|_F^2 \,\leq\,\eps\;.\]
Then there is a $d'=(1+O(\eps))d$, an isometry $w:\C^d \to \C^{d'}$ and a projective measurement $\{P_u\}_{u\in \Z_2^k}$ on $\C^{d'}$ such that 
\[ \Es{x\in \Z_2^k} \Big\| \phi(x) - w^* \Big(\sum_u (-1)^{u\cdot x} P_u\Big) w \Big\|_2^2 \,=\, O(\eps)\;.\]
\end{corollary} 

\begin{proof}
Any $\phi$ as in the corollary statement is an $(\eps,U_R)$-almost homomorphism of $\Z_2^k$ into $(M_d(\C),\frac{1}{d}\tr)$ for the exhaustive presentation. Applying Theorem~\ref{thm:gh}, $\phi$ is $O(\eps)$-close to a homomorphism into $M_{d'}(\C)$. Such a homomorphism is given by commuting Hermitian unitaries $(U_x)_{x\in\Z_2^k}$ on $\C^{d'}$. Then it is easy to verify that for $u\in \Z_2^k$, $P_u = \Es{x} (-1)^{u\cdot x} U_x$ is a projection such that the $\sum_u P_u=\Id$ and the $\{P_u\}$ satisfy the conclusion of the corollary. 
\end{proof}



\begin{lemma}[Lemma 3.8 in~\cite{slofstra2019set}]
For every $k$, there is a $\delta_k = O_k(\eps)$ such that the presentation~\eqref{eq:z2-efficient} is $\delta_k$-stable.
\end{lemma}

The dependence of $\delta_k$ on $k$ that comes out of the proof from~\cite{slofstra2019set} is exponential, i.e. $\delta_k(\eps)\leq C^k \eps$ for some $C>1$. In Section~\ref{} we give a presentation that has an intermediate trade-off. 

We end with the Pauli group. For an integer $k\geq 1$,  the Pauli group $\pauli_k$ can be defined in many ways. Let $\gamma: \Z_2^k\times \Z_2^k \to \{-1,1\}$ be given by $\gamma(a,b)=(-1)^{a\cdot b}$. Then $\pauli_k$ is the central extension of $\Z_2^k\times \Z_2^k$ by $\{-1,1\}$ given by $\gamma$. Alternatively, this is the group generated by the Pauli matrices 
\[ \sigma_X(a) = , \sigma_Z(b) =\]
for $a,b\in \Z_2^k$. It is also the same as the Heisenberg group $H_{2k+1} = \{\begin{pmatrix} 1 & a & c \\ 0 & 1_k & b \\ 0 & 0 & 1 \end{pmatrix} \subseteq GL_{k+2}(\F_2)$. Concretely, we label elements of the Pauli group as triples $(J,a,b)$ and let 
\[ \pauli_k \,=\, \langle (a,b), a,b\in \Z_2^k : (a,b)(a',b') = J^{a\cdot b'} (a',b')(a,b) \rangle_{\Z_2}\;.\] 
Applying Theorem~\ref{thm:gh} we obtain the following. 

\begin{corollary}[Pauli braiding test]
Let $\phi_X,\phi_Z:\Z_2^k \to \Obs(\C^d)$ be such that for all $W\in \{X,Z\}$,
\[ \Es{a,b\in \Z_2^k} \big\| \phi_W(a)\phi_W(b)-\phi_W(a+b) \big\|_F^2 \,\leq\,\eps\;,\]
and
\[ \Es{a,b\in \Z_2^k} \big\| \phi_X(a)\phi_Z(b)- (-1)^{a\cdot b} \phi_Z(b)\phi_X(a) \big\|_F^2 \,\leq\,\eps\;.\]
Then there is a $d'=(1+O(\eps))d2^{-k}$, an isometry $w:\C^d \to (\C^2)^{\otimes k} \otimes \C^{d'}$ such that for all $W\in \{X,Z\}$, 
\[ \Es{W\in \Z_2^k} \Big\| \phi_W(a) - w^* \big(\sigma_W(a)\otimes \Id\big) w \Big\|_2^2 \,=\, O(\eps)\;.\]
\end{corollary}

In the statement of the corollary, we implicitly defined $\phi$ as an $\eps$-almost homomorphism for the following presentation of $\pauli_k$:
\[ \pauli_k = \langle \cdots\rangle\]
This presentation is slightly simpler than the exhaustive presentation. We chose to use it for ``historical'' reasons, \tnote{...} 

\begin{proof}
\tnote{todo}
\end{proof}


\section{Efficient stability}



There is a more efficient presentation of $\Z_2^k$, given as

\begin{equation}\label{eq:z2-efficient}
 \Z_2^k = \langle x_1,\ldots,x_k : [x_i,x_j]=e, x_i^2=e \; \forall i\neq j \rangle_{\Z_2}\;.
\end{equation}



By adapting~\cite[Corollary 2.6]{dlS} we obtain the following. 


\begin{theorem}
For $i\in \{1,2\}$ let $\Z_2^k = \langle S_i:R_i\rangle$ be $\delta_i$-stable. Suppose further that the uniform measure on $S_i\subseteq \Z_2^k$ has spectral gap $\mu_i$. Then 
\[ P_k = \langle (a,b), a\in S_1,b\in S_2 :  J^{ab'+ab} (a,b)(a',b')(a,b)(a',b') \rangle\]
is $O(\delta_1+\delta_2)$-stable.
\end{theorem}

\begin{proof}

\end{proof}



\section{Nonlocal games}
	
For the relation to games, see~\cite[Proposition 4.4]{slofstra2018entanglement}. However, this is only for finite-dimensional reps.

Application of stability as robustness.


\section{Applications}

\subsection{The Pauli braiding test}

\subsection{Qubit tests}

\subsection{Dimension bounds}


Formulate the PBT as an LCS. Get soundness as a corollary. 





\bibliography{qld}

\notesendofpaper

\end{document}
